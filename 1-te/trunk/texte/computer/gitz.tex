\subsection{Gau"s-IT-Zentrum}

Das Rechenzentrum befindet sich in der Hans-Sommer-Str. 65, direkt hinter dem ,E-Tower'.
Wem der klassische Kommunikationsweg per Email \nurl{it-zentrum@tu-braunschweig.de} oder per Internet \nurl{http://www.tu-braunschweig.de/it} zu schwierig erscheint, kann auch per Telefon (0531-391 5555) oder pers"ohnlich das Rechenzentrum besuchen, "ahm das hei"st ja jetzt \textbf{G}aus-\textbf{IT}-\textbf{Z}entrum, dern wundersch"onen ,braunen W"urfel' findet Ihr z.B. im Standplan \nurl{http://stadtplan.braunschweig.de}. Wer den weiten weg nicht scheut, der findet au"ser den Linux-Distributionen noch viele weitere n"utzlige Features und Gadjets, die hin und wieder das Leben und Studium vereinfachen. Angefangen mit \textbf{A} wie Antworten zu Problemen rund um Euren Account (y-Nummer, etc.)
"uber \textbf{B} wie B"ucher "uber g"angige IT-Themen wie Betriebssysteme, Netze oder Programmiersprachen. Eine "ubersicht dieser sehr g"unstigen und oft guten Zusammenstellungen findet Ihr auf \nurl{http://www.tu-braunschweig.de/it/service-desk/rrzn-handbuecher}. Weiter geht es mit \textbf{K} wie Kurse \nurl{http://www.tu-braunschweig.de/it/aktuell/kurse} zu g"angigen Programmen wie zum Beispiel Maya, Photoshop oder auch AutoCAD und PHP, sowie C-Programmierung. Diese werden f"ur Studierende zumeist kostenlos vom GITZ angeboten. Am besten Ihr schaut einfach selber unter \textbf{S} wie Services \nurl{http://www.tu-braunschweig.de/it/services} was alles an Ger"aten und, Scannern, Software, Kursen im GITZ angeboten wird.
Der aufmerksame Leser der GITZ-Seiten ist bestimmt "uber den Abschnitt mit seinem Workspace gestolpert. Jeder Studi hat ungefaehr \unit[250]{MB} zur freien ferf"ugung, er kann sich auch einen Ordner anlegen, der im Netz erreichbar ist, also f"ur statische HTML-Seiten, oder per FTP Dateien von Zuhause auf den Uni-Account schieben, damit diese dann in der Uni abrufbar sind.\newline

DER ARTIKEL WIRD NOCH ERWEITERT... heute Nachmittag
