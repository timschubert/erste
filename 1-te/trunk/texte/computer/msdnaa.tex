\subsection{Microsoft Academic Alliance}
\label{msdnaa}
Die TU hat seit 2003 eine
Campuslizenz\nurlfootnote{http://msdn.microsoft.com/en-us/default.aspx}
von Microsoft erworben, in deren Rahmen du Microsoftprodukte kostenlos
beziehen kannst.\\
Zur Auswahl stehen die meisten Betriebssysteme, Entwicklungswerkzeuge und
diverse Serversoftware\footnote{\sloppy Eine komplette Liste der Software
findet sich unter \nurl{http://msdn.microsoft.com/en-us/subscriptions/downloads/default.aspx}}.
Die Office-Suite ist explizit \textbf{nicht} enthalten.
%geht wohl nicht mehr laut link
% kann aber separat
%gekauft werden.\footnote{Mehr Infos unter
%\nurl{http://www.tu-bs.de/it/services/software/lizenz/vertraege/ms-select/}}\\
Die Software darf zu nicht-kommerziellen Zwecken in Forschung und Lehre
eingesetzt werden, jedoch keine Infrastrukturaufgaben
erf"ullen\footnote{Die Nutzungsbedingungen sind nachzulesen unter
\nurl{http://msdn.microsoft.com/en-us/academic/bb250609.aspx}}.
Infos gibt es unter
\nurl{https://www.tu-braunschweig.de/it/service-interaktiv/software/doku/msdn-aa}.

%TODO hier muss wohl auch überarbeitet werden. XP ist nun echt nicht mehr aktuell, und kann dennoch auch heruntergeladen werden.

Etwas paradox ist dabei, dass du ein laufendes Windows brauchst, um Software 
(also auch Windows selbst) herunterzuladen. Du kannst Microsoft Windows XP 
aber auch bei den Operateuren im Rechenzentrum in \nroom{Raum 015}
f"ur eine Schutzgeb"uhr von \EUR{5} erwerben, die "ubrige Software kannst du
dort ausleihen oder unter
\nurl{https://www.tu-braunschweig.de/it/service-interaktiv/software/doku/msdn-aa}
downloaden.