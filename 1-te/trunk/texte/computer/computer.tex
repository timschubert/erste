\subsection{Computer und Informatik}
"`Informatik hat viel mit Computern zu tun!"' - Diesem Irrglauben erliegen zu
Anfang des Studiums einige. Zwingend notwendig ist ein Computer eigentlich nur
f"ur die Hausaufgaben, die in Programmieren aufgegeben werden.\\
Wie ihr euren Computer doch noch im Rahmen des Informatikstudiums nutzen
k"onnt, erkl"aren wir euch auf den folgenden Seiten.

\subsubsection{Vorlesungen Online}
Zu den meisten Vorlesungen kann man die Skripte im Internet finden. Es gibt auch
immer engagierte Studierende, die ihre Vorlesungsmitschriften online stellen.
Da diese sehr wahrscheinlich in deinem Semester sind, hilft es, wenn du dich
in den Vorlesungen umh"orst.\par
Hier die offiziellen Seiten zu den Vorlesungen:

\begin{description}
\item[Algorithmen und Datenstrukturen]~\\
{\footnotesize\nurl{http://www.ibr.cs.tu-bs.de/courses/ws0809/aud/index.html}}
\item[Diskrete Mathematik]~\\
{\footnotesize\nurl{http://www.mathematik.tu-bs.de/dm/mitarb/kemnitz/diskrmathinf.html}}
\item[Programmieren I]~\\
{\footnotesize\nurl{http://www.ips.cs.tu-bs.de/content/view/176/120/lang,german/}}
\item[Theoretische Informatik I]~\\
{\footnotesize\nurl{http://www.iti.cs.tu-bs.de/TI-INFO/milius/teaching/WS0809/THEOINF/}}
\item[Wissenschaftliches Arbeiten]~\\
{\footnotesize\nurl{http://ira.gaussschule-bs.de/WissArb/index.html}}
\end{description}

Desweiteren k"onnt ihr euch im Netz euren individuellen Stundenplan
zusammenstellen, in Erfahrung bringen, wann die n"achsten Klausuren
stattfinden, lesen, was es in der Mensa zu essen gibt, endlich herausfinden,
wann das Fachbereichssekretariat ge"offnet hat, den Seminarraum des
Rechenzentrums f"ur Lerngruppen reservieren, offene HiWi"~Stellen bei den
Instituten finden und vieles mehr.

\subsubsection{Computer an der Uni}
Es ist immer n"utzlich zu wissen, wo man mal schnell an einen Computer kann.

Zumindest ab und zu wirst du die Computer in der Uni benutzen, besonders
die Linuxarbeitspl"atze in \nroom{PK4.5} oder \nroom{PK4.8}, an denen du die Hausaufgaben f"ur
Programmieren abgeben musst.

{\small
An dieser Stelle gleich noch ein Tipp zu den Hausaufgaben in Programmieren:
Gib sie fr"uh ab, auch wenn du mehrere Wochen Zeit hast. In der letzten Woche
vor dem Abgabetermin ist der Raum immer so mit Studierenden und halbfertigen
Programmen "uberf"ullt, dass du oft sehr lange auf einen freien Computerplatz
warten musst - und noch l"anger auf den (dann schon total genervten) HiWi.
}

\begin{itemize}

\item[*] Im Erdgeschoss des Altbaus gibt es auf der rechten Seite zwei
Computerr"aume, einen weiter vorne (\nroom{PK4.6}) und einen genau in der Ecke
des Geb"audes (\nroom{PK4.5}). Zwei weitere R"aume (\nroom{PK4.8} und die
\nroom{"`Datenstation"'}) findest du im ersten Stock des Altbaus, auch wieder in
der rechten Ecke. Die Rechner in \nroom{PK4.5} und \nroom{PK4.8} sind mit Linux
ausgestattet. Im ersten Stock gibt es nun auch einen Windowsrechnerraum. Da kann man mal eine Word- oder
Powerpoint-Datei ausdrucken, wenn man denn muss.

\item[*] Reichlich Computer findet man schlie"slich im Gau"s-IT-Zentrum~(GITZ) an der
Hans-Sommer-Stra"se. Das ist der gedrungene, fast w"urfelf"ormige, dunkle
Klotz hinter dem Elektrotechnik-Hochhaus ("`E-Tower"'). Hier gibt es mehrere frei
zug"angliche R"aume mit Mappits\footnote{Das sind die ThinClients auf denen
Linux und Windows l"auft. Windows auf den Mappits befindet sich allerdings noch
im Teststadium.}, Linux PCs und auch einen gro"sen Windows"~Pool mit
Windows~2000 Rechnern.

\item[*] Seit 2005 stellt das IBR (Institut f�r Betriebssysteme und Rechnerverbund) im Raum G40 des Informatikzentrums einen Rechnerraum mit 24 Apple iMacs zur Verf�gung. Zu diesem CIP-Pool (Computer-Investitions-Programm) bekommt man mit seiner y-Nummer Zutritt. Anmeldung unter: \nurl{http://www.ibr.cs.tu-bs.de/passwd/rz.html}

\end{itemize}

WLAN wird vom Rechenzentrum in vielen H"ors"alen (wie dem \nroom{Audimax} und
\nroom{SN19.1}), in der Universit"atsbibliothek (UB) und im GITZ angeboten.
Notebookbesitzer k"onnen.\par

[BEARBEITEN AB HIER, HOWTO]

Aber Vorsicht beim kabellosen Vergn"ugen. Unverschl"usselt "ubertragene
Passw"orter (z.B. bei ftp, http, pop3 und imap) k"onnen alle WLAN
Benutzer in deinem Umkreis mith"oren. Also verwende immer "uber SSL
gesicherte Protokolle, wenn du sensible Daten "ubertr"agst.

Wer etwas schneller unterwegs sein will, dem sei das normale Ethernet ans
Herz gelegt. Ein Kabel dazu musst du dir selbst mitbringen. Dosen zum
Anschlie"sen gibt es in der Uni"~Bibliothek (z.T. versteckt unter runden
Klappen im Boden, z.T. an der Fensterseite frei liegend) und im
Rechenzentrum (im Laptopraum \nroom{R003} und in \nroom{R001} zwischen den
Mappits).

\subsubsection{Hilfe, UNIX!}
Du wirst es sicher schon bemerkt haben, die meisten Rechner an der Uni
laufen nicht unter Windows, sondern unter UNIX"~artigen Betriebssystemen.
Wenn du schon Linuxguru bist, kannst du diesen Absatz wahrscheinlich
"uberspringen, aber sollte dir SSH fremd sein, dann lies ruhig
weiter.\par
Um vom heimischen PC aus Zugriff auf deinen Uniaccount zu haben, kannst
du von Linux aus ssh benutzen. F"ur Windowsbenutzer gibt es zwei nette
kleine Tools, Putty und WinSCP. Deinen Uniaccount erreichst du "uber
den Server \nurl{rzlx00xx.rz.tu-bs.de} (xx geht von 01 bis 12).

\begin{description}
\item[Putty] stellt dir eine Shell auf dem UNIX"~Rechner bereit. Damit
kannst du so auf deinem Rechner arbeiten, als w"urdest du direkt auf
dem Server arbeiten (tust du ja auch). Um auch grafische Programme
starten zu k"onnen, musst du noch einen X"~Server f"ur Windows installieren,
z.B. X-Deep32.
\item[WinSCP] ist ein Tool, das einem FTP"~Client "ahnelt. Mit diesem
kannst du Dateien von und zu deinem Uniaccount kopieren. Der Vorteil
ist, dass die "Ubertragung verschl"usselt ist und Passw"orter somit
nicht abgeh"ort werden k"onnen.
\end{description}

Zu allen in diesem Text angesprochenen und noch zu vielen anderen
Computerproblemen mehr gibt es Informationen im Heft "`Don't Panic"',
das kostenlos im Rechenzentrum erh"altlich ist. Nimm es dir gleich mit, wenn
du deine y"~Nummer beantragst.

\begin{description}
\item[Allgemeines Vorlesungsverzeichnis:] ~\\
{\footnotesize\url{http://www.tu-braunschweig.de/studium/lehrveranstaltungen/}}
\item[Uni-Bibliothek:] ~\\
{\footnotesize\url{http://www.biblio.tu-bs.de/}}
\item[Druckkosten:] ~\\
{\footnotesize\url{http://www.tu-braunschweig.de/it/services/drucken/kosten}}
\item[Don't Panic online] ~\\
{\footnotesize\url{http://rz-cgi.tu-braunschweig.de/download/dokumente/gitz/dontpanic.pdf}}
\item[Putty Homepage] ~\\
{\footnotesize\url{http://www.chiark.greenend.org.uk/~sgtatham/putty/}}
\item[WinSCP Homepage] ~\\
{\footnotesize\url{http://winscp.net/eng/index.php}}
\end{description}
