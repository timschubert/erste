
\subsection{Stunden- und Semesterplan}
 
Im Vergleich zum Bachelorstudium an der TU-BS bietet der Master
verblüffende Freiheiten in der Fächerwahl. Auch wer die Redewendung
bisher dumm fand, lernt spätestens hier was "`die Qual der Wahl"'
bedeutet. Daher sind dieses und die folgenden Unterkapitel auch in erster
Linie für Masterstudenten gedacht. Allerdings werden mit einigen
Themen auch Bachelorstudenten in Berührung kommen, aber erst ab den
3. Semester. Bei Unterschieden zwischen den Studiengängen gehen wir
kurz darauf ein.\\\\
Zu Beginn jedes Semesters muss man sich selbständig entscheiden, welche Fächer man belegen möchte (und kann) und sich danach den Stundenplan zusammenstellen. Im Gegensatz zu den alteingessenen Bacheloranden aus Braunschweig steht man recht unvorbereitet vor einer ganz und gar nicht trivialen Aufgabe, die man am besten noch vor dem ersten Studientag erledigen sollte. Vielleicht hatte dein Bachelor kaum Wahlmöglichkeiten und das Zusammenstellen eines eigenen Stundenplans ist für dich ungewohnt. Und selbst wenn du diese Freiheiten schon im Bachelor hattest, so sind die Rahmenbedinungen und Feinheiten hier sicherlich ganz andere.

Und anders als im Bachelor, bei dem die meisten Vorlesungen erst in der zweiten Woche beginnen, starten viele Mastervorlesungen schon in der ersten Woche, schlimmstenfall schon Montag früh. Die erste Vorlesung zu verpassen ist nun auch nicht so schlimm, aber wenn mans vermeiden kann\ldots Es ist übrigens auch nicht ganz leicht, den Vorlesungsstart eines Faches herauszufinden, somit wirst du in der ersten (und evtl. auch zweiten) Woche oft vor verschlossenen Türen oder leeren Räumen stehen.

\subsection{Wie viele Credit Points?}
Standardmäßig ist vorgesehen, pro Semester 30 Credit Points zu erlagen - so hat man nach 4 Semestern den Master in der Tasche. Man ist dann aber auch zeitlich sehr ausgelastet, und für Urlaub, Familie und Nebenjob bleibt nicht unbedingt Zeit. Wenn man außerdem mit Zulassungsauflagen gesegtnet ist, sind dies bis zu 15 weitere Credit Points, die man irgendwie auf die ersten beiden Semester aufteilen muss, und so lohnt es sich durchaus frühzeitig darüber nachzudenken, wie viele Semester man wirklich studieren möchte und wie viele Credit Points man pro Semester ableisten möchte und kann. Du wirst hier und da noch Gerüchte hören, dass man mindestens 15 CP pro Semester schaffen muss. Das war früher mal so, wurde aber glücklicherweise nun abgeschafft, also lass dich von solchen Aussagen nicht allzusehr beeinflussen.

Dann steht ja dem entspannten Studium (fast) nur noch die Finanzierung im Wege. BAFöG-Höchstförderungsdauer, Langzeitstudiengebühren, sowie das Ende von Kindergeld, Kindesunterhalt und Famlienversicherung bei der Krankenkasse sind hier die relevanten Stichwörter, die viele Masterstudenten irgendwann ereilen. Hiwi-Jobs, Studienkredite und Stipendien verschaffen vielleicht Linderung.

Was auch immer du nun denkst, wie viele CP du im kommenden Semester belegen möchte, plane vielleicht ein paar Reserve-Punkte ein, also zusätzliche Fächer, die du belegst. Du kannst dann immernoch im laufenden Semester mache Vorlesungen "`kicken"' wenn es doch nicht so spannend ist wie zuerst gedacht bzw. du kommst noch auf die angepeilte Punktezahl, auch wenn du durch ein oder zwei Prüfungen fällst. Durchfallen ist weder eine Schande noch ein großes Problem, da die Prüfungsordnung dir erlaubt, bis zu drei Fächer, bei denen du durchgefallen bist, so abzuwählen als hättest du sie nie belegt. Dennoch sollte man es vielleicht mit den Reservefächern nicht übertreiben. Versuche einfach, den folgenden plumpen Witz auf diese Situation zu übertragen, und frag dich, ob das zielführend ist:

Kunde: Ich hätte gerne 20 Brötchen. 
Bäckerin: So viele? Davon wird ihnen doch die Hälfte trocken bevor sie die gegessen haben!
Kunde: Oh, das hab ich nicht bedacht. Dann nehm ich doch lieber 40 Stück.

\subsection{Nebenfach und Studienrichtung}
Im Bachelor musst du, im Master kannst du ein Nebenfach wählen.
%Laut Prüfungsordnung steht es dir frei, ob du im Master ein Nebenfach
%wählen möchtest oder "`reine"' Informati%k studierst.
 Die Nebenfach-Enscheidung (ob und welches) will gut überlegt sein, denn wenn man
 erstmal "`drin"' ist (d.h. zwei Prüfungen im Nebenfach bestanden hat)
 kommt man nicht mehr raus. Dies gilt für beide Studiengänge, man kann
 das Nebenfach dann auch nicht mehr wechseln. \\\\
Die Studienrichtung ist  optional, aber im Gegensatz zum Nebenfach
geht man damit keinerlei Verpflichtung ein. Am Ende des Studiums wird
einfach geschaut, ob man 50 (Bachelor) oder 70 (Master) Credit Points
aus einem artverwanden Bereich gemacht hat und bekommt dann auf Wunsch
ein Sonderprädikat aufs Zeugnis. 
Beide Entscheidungen musst du nicht im ersten Semester treffen, sondern kannst dich auch später (aber am besten nicht zu spät) spezialisieren.

\subsection{Mentoren und Beratungsgespräche}
Laut Studienordnung bekommst du auch einen Mentor zugewiesen - das ist Professor aus der Informatik, der dich bei Entscheidungen zum Studium im persönlichen Gespräch beraten soll. Gerade wenn du weißt, dass du dich spezialisieren möchtest, oder wenn du zumindest mit dem Gedanken spielst, solltest du einen Mentor haben, der aus der jeweiligen Fachrichtung kommt. Wird dir zu Beginn ein völlig fachfremder Mentor zugewiesen, dann kannst du recht formlos darum bitten, diesen zu wechseln. Gespräche mit dem Mentor sind weder verpflichtend noch planmäßig vorgesehen, es liegt also an dir, dich um einen Termin zu kümmen, wenn du beraten werden möchtest. Manche Mentoren veranstalten auch im Dezember ein großen Treffen mit all ihren Schützlichen, was dich aber nicht davon abhalten sollte, schon vorher das Gespräch zu suchen.

Außer dem dir zugewiesenen Mentor gibt es noch weitere Ansprechpartner für verschiedenste Anlässe. Die wichtigsten haben wir für dich unter \url{http://fginfo.cs.tu-bs.de/index.php/kontakt/ansprechpartner/} zusammengefasst.

\subsection{Modularten}
Den Hauptteil deiner 120 Credit Points sammelst du durch den Besuch von Vorlesungen und durch Bestehen der damit zusammenhängenden Prüfungen. Es ist schon schwer genug, sich für eine Menge von Vorlesungen zu entscheiden, aber hinzu kommen noch diverse andere Arten, sich Punkte zu verdienen. Das meiste davon gilt auch schon im Bachelor, siehe dazu den entsprechenden Abschnitt auf Seite ??. Für den Master kommt noch die Projektarbeit hinzu. Dies ist ein dicker, optionaler Brocken der euch 14 CP einbringt, und der in einem eigenständig erstellten Software-Projekt und schriftlicher Ausarbeitung besteht. Wenn man solch ein Projekt einbringt, dann überlicherweise direkt vor der Masterarbeit, es wird dich daher im ersten Semester noch nicht direkt interessieren, ist aber der vollständigkeit halber erwähnt.

\subsection{Mündliche Prüfungen}
Übrigens wirst du feststellen, dass fast alle Masterfächer mit einer
 mündlichen Prüfung abschließen. 
 Wer keine Pflichtauflagen hat, kann eventuell durch das gesamte
 Masterstudium ohne eine einzige schriftliche Klausur kommen! 
Im Bachelor sind hingen nahezu alle Prüfungen schrifltich, laut
 Prüfungsordnung sind aber drei mündliche Prüfungen abzulegen.
Im Gegensatz zu Klausurabschriften ist es generell geduldet, Prüfungsprotokolle zu verteilen. Das sind Gedächtnisprotokolle dessen, wie eine Prüfung abgelaufen ist, also was für Fragen gestellt werden, wie der Prof auf falsche Antworten reagiert, etc. Du findest diese Protokolle auf den Seiten der Fachgruppe.

\subsection{Studienplan und Reihenfolge}
Mindestens so frei wie die Wahl der Fächer ist auch die Wahl der Reihenfolge, in der man diese belegt. Es kursieren Studienpläne für den Master, die vorschlagen, bestimmte Module im ersten Semester zu machen (z.B. das Seminar), und andere im zweiten und dritten, etc. All dies sind höchstens gut gemeinte Tipps, letzlich muss ja nichtmals die Masterarbeit unbedingt am Ende stehen, und zur Idee, das Seminar im ersten Semester zu belegen, siehe auch den nächsten Absatz\ldots

\subsection{Seminar}
Außerdem musst du auch ein so genanntes Seminar einbringen, das ist eine Ausarbeitung zu einem Thema, die meist in einem Vortrag und einer mehrseitigen schriftlichen Ausarbeitung mündet. Anders als für alle anderen Modularten muss man sich für das Seminar inklusive Themenwahl schon im Vorraus anmelden. Die Frist für das jetzt beginnende Wintersemester war irgendwann Anfang August, daher wird wohl kein jetzt hinzugezogener Masterstudent sein Seminar im ersten Semester belegen. Keine Sorge, du hast noch mindestens drei weitere Semester, in denen du dieses Modul unterbringen kannst. Halte einfach kurz vor Vorlesungsende Ausschau nach der Ankündigung der Seminar-Info-Veranstaltung, z.B. auf der cs-studs Mailingliste.

\subsection{Welche Fächer gibt es?}
Die Liste der Fächer ist groß und ständig im Wandel. Offiziell
festgelegt sind diese im Modulhandbuch (MHB), und anders als der Name
vermuten lässt, präsentiert sich dieses nicht etwa als handliches Buch
für die linke Jackeninnentasche, sondern als recht unübersichtliche
Webanwendung. Unter
\url{https://mhb.tu-bs.de/mhb1011ws/studiengangAbstract.do?id=176&call=veranstaltungAbstract.do%3Fid%3D1818} 
findest du eine Liste sämtlicher Module, die du im Master einbringen kannst. Und da so ein kryptischer Link weder angenehm zu tippen ist, noch garantiert ist, dass er am Tage nach dem Druck dieses Heftes noch erreichbar ist, kannst du auch \url{https://mhb.tu-bs.de/mhb1011ws/} aufrufen und dann über "`Studiengänge ansehen"' navigieren um dann nach "`Informatik Master"' Ausschau zu halten.

All diese Fächer kannst du als Masterstudent belegen - aber längst nicht jedes davon wird in diesem Semester angeboten. Nach einem Klick auf ein Fach siehst du die Details. Dort steht dann auch alles weitere zum Modul, und manches davon ist verbindlich und stets aktuell und korrekt. Die Information, ob ein Fach im Winter oder Sommer angeboten wird, gehört definitiv nicht dazu, was uns zur nächsten Informationsquelle bringt\ldots

\subsection{Der generelle Stundenplan}
Unter
\url{http://www.cs.tu-bs.de/stundenplan/ws1011/wochenplan.do-20100723.html}
findest du den aktuellen Plan für das Wintersemester. Dort sind -
theoretisch - alle Veranstaltungen der Informatikmodule eingetragen,
allerdings ohne die Nebenfächer und den Schlüsselqualifikations-Pool (siehe entsprechender Abschnitt weiter unten). Der Stundenplan enthält sowohl Bachelor- als auch Masterfächer. Der Plan ist nicht getrennt, da nämlich die Bachelorstudenten auch ein paar Master-Fächer einbringen dürfen - andersrum gilt das aber nicht. Also musst du für jedes Fach, was du hier findest, erstmal verifizieren ob du dessen Punkte überhaupt im Master einbringen kannst. Wie du dir vielleicht schon denken kannst, wird dein persönlicher Stundenplan eine Untermenge dieses Mammut-Plans, erweitert um ein paar Veranstaltungen die selbst hier nicht stehen.

Wenn etwas darauf hindeutet, dass eine bestimmte Vorlesung im Semester angeboten wird, diese aber im Stundenplan nicht auftaucht, dann hilft eine Suche auf den Institusseiten, und wenn selbst das nicht hilft, eventuell eine Mail an den verantwortlichen Professor. Das gleiche gilt, wenn irgendwas komisch wirkt, z.B. wenn im Stundenplan zu einem Fach 5 Übungstermine und kein Vorlesungstermin stehen, was nun durchaus nicht das erste Mal wäre.

\subsection{Schlüsselqualifikationen/Mathe-Wahlpflicht}
Zu deinem Master-Abschluss gehören auch 8 bis 10 Punkte aus dem
Mathe-Wahlpflicht Schlüsselqualifikations-Pool. Im Bachelor sind dort
10 Credits einzubringen. Das ganze heißt deshalb Pool, weil darin
Vorlesungen von allen Fachbereichen und Instituten der Uni schwimmen,
nicht nur Informatikbezogene. Da wir hier von 100 bis 300 angebotenen
Fächern je Semester reden, sind diese nicht im Modulhandbuch und im
Informatik-Studenplan vermerkt, sondern können irgendwo auf
\url{http://www.tu-braunschweig.de/studium/lehrveranstaltungen/fb-uebergreifend}
gefunden werden. Außerdem können hier noch Wahlpflichtfächer aus der
Mathematik eingebracht werden.  Zu beachten ist dabei, dass man dabei
nur Fächer belegen darf, die nicht aus den Nebenfach kommen. Man kann
also z.B mit den Nebenfach Mathe nicht Mathewahlpflichtfächer
belegen. Diese Regelungen sind im Bachelor im Wesentlichen die
Gleichen. Allerdings ist dort der Mathe-Wahlpflichtbereich ein eigener
Bereich von 10 Credits und gehört nicht zu den Schlüsselqualifikationen.