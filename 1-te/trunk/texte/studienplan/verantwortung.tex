\subsection{Verantwortung}
\textit{Große Macht bringt große Verantwortung mit sich!} - Das wusste schon Ben Parker, der Onkel von Spiderman. Was du wissen solltest: Du hast die Macht und die Verantwortung über deinen Studienfortgang. Das geht los bei der Entscheidung, überhaupt zu studieren (dich hat doch - hoffentlich - keiner gezwungen?), über die Wahl des Faches und unserer schönen Universität, erstreckt sich über die Wahl, welche Fächer du hörst und wann du das tust, bis hin zur Einflussnahme auf den gesamten Studiengang (Siehe dazu das gesonderte Kapitel ab Seite \pageref{politik}).

Es besteht aber auch die \textit{Chance} (oder eher \textit{Gefahr}?) diese Verantwortung abzugeben. Es gibt einen Studienplan der dir \textit{vorschlägt} wie du deine Fächer wählen und anordnen kannst, um somit in Regelstudienzeit fertig zu werden. Für den Bachelor sieht dieser Plan sehr konkret aus, für den Master ist er abstrakter gehalten aber deckt immernoch nicht die gesamten Wahlmöglichkeiten ab. Das kann und soll er auch nicht - es handelt sich um zwei von unendlich vielen Möglichkeiten, zum Studienabschluss zu kommen. Vom vorgeschlagenen Plan abzuweichen ist noch lange kein Akt der anarchischen Rebellion - also keine Angst davor.

Wenn du komplett nach Plan studieren möchtest, so brauchst du vergleichsweise wenig über die internen Mechaniken des Studiums zu wissen. Um die Freiheiten, die dir geboten werden, auch nutzen zu können, solltest du allerdings einiges wissen, was wir die auf den folgenden Seiten vermitteln werden. Selbst wenn deine ursprüngliche Entscheidung dem Musterplan entspricht, so kann doch immer etwas dazwischen kommen, was einen kurzzeitig aus der Bahn wirft. Anders als in der Schule muss man dann kein ganzes Jahr wiederholen, das Studium geht einfach weiter. Um dann möglichst schnell wieder rein zu kommen, sollte man sich der diversen Flexiblitäten bewusst sein.