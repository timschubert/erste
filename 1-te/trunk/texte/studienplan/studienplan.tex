\subsubsection{Studienplan}

Wie ihr wahrscheinlich bereits in eurem Stundenplan festgestellt habt, m"usst ihr im ersten Semester f"unf "`Pflichtveranstaltungen'" h"oren.
Doch die Bezeichnung Pflichtverantstaltung sagt blo"s aus, dass ihr die Veranstaltung \emph{irgendwann} einmal h"oren m"usst, um euren Bachelor abschlie"sen zu k"onnen.
Die zeitliche Abfolge der Veranstaltungen d"urft ihr aber selbst festlegen.
Der von Herrn Dr. Struckmann bereit gestellte Musterstudienplan (s. n�chste Seite) bietet hier eine gute Orientierungsm"oglcihkeit.
Ihr m"usst euch aber nicht daran halten. Niemand zwingt euch eine Veranstaltung zu h"oren oder h"alt euch davon ab.
Ihr k"onnt euch eigentlich in jede Vorlesung setzten auch ohne hinterher an der Pr"ufung teilnehmen zu m�ssen.
Hier bieten sich zum Beispiel Module aus dem Wahlplichtbereich Informatik an, die eventuell nur alle 2 Jahre angeboten werden und "uber mehrere Semester gehen.
Bei den (Pflicht-)Modulen der Informatik m"usst ihr jedoch beachten, dass einige Module auf anderen aufbauen.
Zum Beispiel sollten Programmierengrundlagen in den ersten zwei Semestern erarbeitet werden und mit Theoretische Informatik II werdet ihr euch schwer tun, wenn ihr TheoInf I nicht geh�rt habt.

Damit sich euer Studium nicht unn"otig verl"angert, solltet ihr aber darauf achten, in jedem Semester 30 Leistungspunkte zu erwerben. 

