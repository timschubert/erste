
\subsection{Ersti Checkliste}

Hier wird zusammengefasst, was ihr in den ersten Tagen des Studiums
unbedingt erledigen solltet:

\subsubsection{BAf\"oG}

Wer BAf\"oG beantragen m\"ochte, sollte sich am besten gr\"undlich
informieren. Sehr zu empfehlen ist da \url{http://www.bafoeg.bmbf.de/}.

F\"orderungsantr\"age gibt es zum Download oder in Papierform im
Erdgeschoss des BAf\"oG-Amtes, Norstra\ss e 11. Am besten so fr\"uh
wie m\"oglich beantragen, denn BAf\"oG wird nichtr\"uckwirkend bezahlt.

\subsubsection{Mailingliste}

Es gibt eine Mailingliste f\"ur die Studierenden der Informatik.
Sie heißt \emph{cs-studs} und ist \emph{die} Informationsquelle.
Hier werden Ank\"undigungen zu Lehrveranstaltungen gemacht, eure
Fachgruppe k\"undigt hier Spiele- und Grillabende an und es gibt
oft Angebote zu Hiwistellen oder offenen Teamprojekten,
Bachelorarbeiten etc. und selbstverst\"andlich ist dies auch ein
guter Ort, um Fragen zum Studium loszuwerden.

Anmelden k\"onnt ihr euch unter
\url{http://www.cs.tu-bs.de/mailinglisten.html}.

\subsubsection{IRC-Channel}

Viele Studierenden der Informatik, Nebenfachh\"orer und
Fachgruppenmitglieder sind im IRC-Channel \texttt{\#\#cs-studs}
(ja, der zweite ,,\#'' ist korrekt) auf \texttt{irc.freenode.net}
unterwegs. Auch hier ist ein guter Ort, Fragen zu stellen.

\subsubsection{Mensa-Card}

Ihr braucht unbedingt eine Mensa-Card (eine Chipkarte,
mit der ihr in der Mensa bargeldlos bezahlen k\"onnt), sonst
m\"usst ihr den G\"astepreis zahlen. Ihr bekommt die Karte beim
AStA neben der Mensa (Studierendenausweis und Lichtbildausweis
nicht vergessen).

\subsubsection{Uni-Bibliothek}

Um B\"ucher in der Uni-Bibliothek ausleihen zu k\"onnen, braucht ihr
einen Ausweis. Diesen k\"onnt ihr an einem der Terminals in der
Bibliothek beantragen und danach gegen eine Geb\"uhr von \unit[5]{\EUR}{}
am Schalter abholen.

\subsubsection{Ummelden}

Wer neu nach Braunschweig gezogen ist, muss sich innerhalb einer
Woche beim Einwohnermeldeamt anmelden. Wenn man Braunschweig als
Erstwohnsitz w\"ahlt, bekommt man eine einmalige Zuzugspr\"amie von
\unit[200]{\EUR}{} (Immatrikulationsbescheinigung nicht vergessen). Wer
dennoch seinen Erstwohnsitz inder Heimat behalten m\"ochte, sollte
glaubhaft darlegen k\"onnen, dass er mehr als die H\"alfte des Jahres
nicht in Braunschweig lebt bzw. seinen Lebensschwerpunkt in der
Heimatstadt hat.

