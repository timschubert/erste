\subsection{Sonstige Informationen}
\subsubsection{Sprachenzentrum}
Am Sprachenzentrum der Uni kannst du verschiedene Sprachkurse belegen, die du auch als Schl"usselqualifikationen in deinen Bachelor-Abschluss einbringen kannst.
Auf den Seiten des Sprachenzentrums (\nurl{www.sz.tu-bs.de}) findest du alle angebotenen Kurse.
Um sich f"ur Kurse anzumelden, brauchst du ein Konto, das du pers"onlich in der Mediothek (im Altgeb"aude \nurl{http://www.sz.tu-bs.de/mediothek/}) registrieren musst.

\textbf{Wichtig!} Die Anmeldung f"ur Sprachkurse beginnt bereits vor jedem Semester.
Um Pl"atze zu bekommen, solltest du dich also so fr"uh wie m"oglich anmelden.
Bevor du an einem Englischkurs teilnehmen kannst, musst du zun"achst einen Einstufungstest machen.
Die Termine findest du hier: \nurl{http://www.sz.tu-bs.de/cn/fremdsprachen/englisch/einstufungstest/}\\
Da gerade bei diesen Kursen die Nachfrage sehr hoch ist, solltest du den Test m"oglichst bereits vor dem Anmeldungszeitraum (beginnt etwa 2 Wochen vor Vorlesungsbeginn) ablegen.

\subsubsection{Schl"usselqualifikationen}
Jeder Bachelorstudent muss sogenannte Schl"usselqualifikationen innerhalb seines Studiums erwerben.
In der Informatik m�ssen dies "`handlungsorientierte Anwendungen"' im Umfang von 10 Leistungspunkten sein.
Hierzu z"ahlen Sprachkurse und "uberfachliche Lehrveranstaltungen.
Informationen "uber das aktuelle Angebot (Pool-Modell) und die zu erf"ullenden Bestimmungen der Veranstaltungen findet ihr auf dieser Webseite: \nurl{http://www.tu-braunschweig.de/informatik-bsc/struktur/schluessel}

\newpage

\subsubsection{Auslandsaufenthalt}
"Uber Auslandssemester solltest du dich ebenfalls so fr"uh wie m"oglich mit dem "`International Office"' (\nurl{http://tu-braunschweig.de/international}) unterhalten.
Der n"achste Termin f"ur die Infoveranstaltungen "`Wege ins Ausland"' und "`Studieren in Europa"' ist der 05.11. ab 16 Uhr im International Office (BW 74). 