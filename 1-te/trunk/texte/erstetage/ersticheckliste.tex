\subsection{Checkliste}

Hier wird zusammengefasst, was ihr in den ersten Tagen des Studiums
unbedingt erledigen solltet. Ihr könnt die Punkte in der folgenden
Tabelle abhaken um den Überblick zu behalten.

Im Folgenden findet ihr noch erweiterte Infos zu manchen der Punkte. 
Andere Hinweise sind durch die restliche Zeitung verteilt, dafür 
gibts eine Spalte mit der jeweiligen Seite.

\end{multicols}

\begin{tabular}{|p{3mm}|c|c|c|c|}
\hline & \textbf{Todo}				&\textbf{Zu erledigen bis}	 		& \textbf{Seite}			& \textbf{Muss?} \\ 
\hline & GITZ-Account freischalten	& Vor Studienbeginn 			 	& \pageref{todogitz}		& ja \\ 
\hline & Wohnsitz Ummelden 			& 1 Woche nach Umzug 				& \pageref{todoummelden}	& ja \\ 
\hline & Prüfungsanmeldung 			& Anmeldewoche (Dezember) 			& \pageref{todoanmeldung}	& ja \\ 
\hline & Prüfungsbogen 				& spätestens Dezember 				&  							& ja \\ 
\hline & Auflagen klären 			& 									&  							& Master \\ 
\hline & Bafög beantragen 			& Ende Oktober 						& \pageref{todobafoeg}		& nein \\ 
\hline & Mailinglisten 				& 									& \pageref{todomailing}		& nein \\ 
\hline & Mensa-Card 				& Beim ersten Mensa-Besuch 			& \pageref{todomensa}		& nein \\ 
\hline & Bibliotheksausweis 		& 									& \pageref{todobib}			& nein \\
\hline & WLAN einrichten 			& Bevor man an der Uni online geht 	&  							& nein \\ 
\hline & Kopierkarte 				& 									&  							& nein \\ 
\hline & Blog abonnieren 			& 									&  							& sollte :) \\ 
\hline
\end{tabular} 

\begin{multicols}{2}

\subsubsection{BAf"oG}
\label{todobafoeg}

Wer BAf"oG beantragen m"ochte, sollte sich am besten gr"undlich
informieren. Sehr zu empfehlen ist da: \\
\mbox{\nurl{http://www.bafoeg.bmbf.de/}}
 
F"orderungsantr"age gibt es zum Download oder in Papierform im
EG des BAf"oG-Amtes, Nordstra"se 11. Am besten so fr"uh
wie m"oglich beantragen, denn BAf"oG wird nicht r"uckwirkend
bezahlt.

\subsubsection{Mensa-Card}
\label{todomensa}

Ihr braucht unbedingt eine Mensa-Card (eine Chipkarte,
mit der ihr in der Mensa bargeldlos bezahlen k"onnt), sonst
m"usst ihr den G"astepreis zahlen. Bei der Immatrikulation bekommt ihr 
einen Gutschein, den ihr gegen die Mensacard eintauschen könnt - falls 
nicht, kann man sie auch einfach für 5 Euro erwerben. Ihr solltet auch 
Studierendenausweis und Lichtbildausweis nicht vergessen, auch wenn 
das nicht immer gewissenhaft kontrolliert wird. Sobald ihr die Karte 
habt, schreibt euch die darauf stehende Nummer auf, so könnt ihr eurer 
Restgeld wiederbekommen, falls ihr die Karte einmal verliert - und dass 
passiert einem leider recht oft.

\subsubsection{Uni-Bibliothek}
\label{todobib}

Um B"ucher in der Uni-Bibliothek ausleihen zu k"onnen, braucht ihr
einen Ausweis. Diesen k"onnt ihr an einem der Terminals in der
Bibliothek beantragen und danach gegen eine Geb"uhr von \EUR{5}
am Schalter abholen. Je nachdem, ob ihr zu Beginn schon Bücher braucht, 
könnt ihr die Karte auch einfach ein bisschen später besorgen.

In der Bibliothek stehen außerdem Kopierer bereit, die ihr nutzen könnt. 
Einen davon könnt ihr mit Kleingeld befüllen, kompfortabler geht es aber 
mit einer Kopierkarte. Und auch diese bekommt ihr für ein paar Euro 
direkt in der Bibliothek.

Zu Semesterbeginn gibt es oft noch Einführungskurse in die 
Bibliotheksbenutzung. Ob ihr eure Bibliothekskarte vor oder nach 
diesem Kurz besorgt, ist egal.

\subsubsection{Ummelden}
\label{todoummelden}

Wer neu nach Braunschweig gezogen ist, muss sich innerhalb einer
Woche beim Einwohnermeldeamt anmelden. Wenn ihr die Frist verpasst, 
drohen theoretisch Strafen, aber praktisch sieht es da nicht so 
streng aus. Wenn man Braunschweig als
Erstwohnsitz w"ahlt, bekommt man (ein Jahr später) eine einmalige 
Zuzugspr"amie von
\EUR{200} (Immatrikulationsbescheinigung nicht vergessen). Wer
dennoch seinen Erstwohnsitz in der Heimat behalten m"ochte, sollte
glaubhaft darlegen k"onnen, dass er mehr als die H"alfte des Jahres
nicht in Braunschweig lebt bzw. seinen Lebensschwerpunkt in der
Heimatstadt hat.

\subsubsection{Pr"ufungsanmeldung}
\label{todoanmeldung}


Ihr m"usst euch f"ur alle Pr"ufungen, an denen ihr teilnehmen
wollt, vorher beim Pr"ufungsamt anmelden. \emph{Das ist nur eine
Woche lang m"oglich}, im Wintersemester meistens Mitte Dezember,
\emph{informiert euch also rechtzeitig, wann genau das ist}!

Vor eurer ersten Pr"ufungsanmeldung m"usst ihr au"serdem ein
Datenblatt ausf"ullen. Es empfiehlt sich, das bereits vor der
Anmeldewoche zu machen, weil die Schlangen dann nicht so lang sind.

Dar"uber hinaus gibt es die M"oglichkeit, sich online f"ur
Pr"ufungen anzumelden. Dazu braucht ihr allerdings eine TAN-Liste,
die ihr euch vorher im Pr"ufungsamt organisieren m"usst.

Weil das ganze etwas komplizierter ist, und euch in den allerersten 
Wochen noch nicht so direkt tangiert, machen wir da zu gegebener Zeit 
noch einen extra Infotermin, der euch alles wichtige dazu erklärt.