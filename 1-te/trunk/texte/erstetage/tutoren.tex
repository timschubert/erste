\subsection{Tutorien}

% TODO Der Begriff Tutorium ist hier recht eigenwillig genutzt. An den meisten anderen Unis versteht man unter "Tutorium" eher das, was hier "Praktikum" oder "Übung" ist, und dort ist dann "Praktikum" wieder was ganz anderes. Darauf könnte man kurz hinweisen.

% TODO dieser Abschnitt ist aus diverse Kapiteln zusammengeklebt und braucht dringend noch Aufräumung!

In diesem Abschnitt geht es um eure Tutorengruppen, die aus 2 Tutorengruppenleitern (Tutoren) sowie 10--15 Erstis bestehen und euch den Einstieg ins Studium erleichtern sollen.\\
Erfahrungsgem"a"s treten in der Anfangszeit einige Fragen auf. Oft wei"s man noch nicht genau, an wen man sich wenden sollte oder kann den etwas bescheidenen Internetseiten der TU nicht die richtigen Informationen entlocken. In diesen F"allen sind die Tutoren die richtigen Ansprechpartner.\\

% TODO hier zerbricht die schöne neue Ordnung - gehört der Folgende Absatz nach "Fachgruppe" oder nach "Erste Tage"?
Vom besonderen Interesse dürften dabei die allgemeine Einführung sein,
sowie das Ersti-Begrüßung mit anschließenden Rundgang mit der
Tutorengruppe.
Dazu werden wir euch nach dem Frühstück studentischen 
Tutoren zuteilen, mit denen ihr das Unigel"ande und Anderes erkunden 
k"onnt. Bei ihnen k"onnt ihr auch die ersten Fragen loswerden, wenn 
ihr nicht bei einem der beiden Beratungstermine davor wart.
Die Einteilung findet nach dem Frühstück statt und erfolgt nach
Bachelor und Master getrennt.
Jedem Ersti soll in der Einf"uhrungswoche (s. Termine) eine Tutorengruppe zugeteilt werden. Damit auch ihr wisst, an wen ihr euch wenden k"onnt, solltet ihr diese Einteilung und das anschlie"sende erste Treffen nicht verpassen!\\
Dort habt ihr dann die M"oglichkeit in "uberschaubarer Runde andere Informatikstudenten des 1. Semesters kennen zu lernen, Fragen zu stellen, weitere Informationen zu euren Veranstaltungen und Dozenten zu erhalten und das Campusgel"ande kennen zu lernen.\\
Scheut euch nicht einen der Tutoren zu kontaktieren. Ihr k"onnt euch auch nachtr"aglich noch in eine Tutorengruppe einteilen lassen, da weitere Treffen geplant sind. Dazu schreibt bitte eine Mail an die Fachgruppe \nurl{fginfo@tu-bs.de} oder direkt an einen der Tutoren.\\
Damit ihr auch ein paar Gesichter zuordnen k"onnt - gerade wenn ihr eventuell selbst noch keinen Tutor habt - sind hier Fotos einiger Tutoren abgebildet:

% TODO hier evtl. den Unterschied zwischen Tutor und Mentor erwähnen?

\end{multicols}

% TODO Bilder aktualisieren, angeben, ob der jenige Bachelors oder Masters betreut (ist ja nicht immer identisch damit, was man selbst gerade ist)

\paragraph{Bachelor} \ \\

\npicture[0.3\linewidth]
{bilder/tutoren/kris.jpg}
{Christoph\\ 3. Semester Master\\ admin@keeg.de}
\hfill
\npicture[0.3\linewidth]
{bilder/tutoren/dominik.png}
{Dominik\\1. Semester Master\\ d.schuermann@tu-bs.de}
\hfill
\npicture[0.3\linewidth]
{bilder/tutoren/franziska.jpg}
{Franziska\\5. Semester Bachelor\\ f.werk@tu-bs.de}
\par \ \par
\npicture[0.3\linewidth]
{bilder/tutoren/hella.png}
{Hella\\ 1. Semester Master\\ h-f.hoffmann@tu-bs.de}
\hfill
\npicture[0.3\linewidth]
{bilder/tutoren/johannes.png}
{Johannes\\ 5. Semester Bachelor\\ J.Starosta@tu-bs.de}
%\hfill
%\npicture[0.3\linewidth]
%{bilder/tutoren/}
%{Judith\\ 5. Semester Bachelor\\ judith.hilpert@web.de}
\hfill
\npicture[0.3\linewidth]
{bilder/tutoren/marekd.jpg}
{Marek\\ 1. Semester Master\\ m.drogon@tu-bs.de}
%\hfill
%\npicture[0.3\linewidth]%{}
%{bilder/tutoren/}
%{Christina\\ 5. Semester Bachelor\\c.eberth@tu-bs.de }
%\hfill
%\npicture[0.3\linewidth]%{}
%{bilder/tutoren/}
%{Christoph\\ 5. Semester Bachelor\\ christoph.harburg@web.de }
\par \ \par

%\npicture[0.3\linewidth]%{}
%{bilder/tutoren/}
%{Serj\\ 1. Semester Master\\s.dechand@tu-bs.de }
%\hfill
%\npicture[0.3\linewidth]%{}
%{bilder/tutoren/}
%{Jonathan\\ 5. Semester Bachelor\\ j.koscielny@tu-bs.de}%.eberth@tu-bs.de }
%\hfill
%\npicture[0.3\linewidth]
%{bilder/tutoren/olav.jpg}
%{Olav\\ 3. Semester Master\\ olav\_bk@web.de}
%\hfill
\npicture[0.3\linewidth]
{bilder/tutoren/sebastian.jpeg}
{Sebastian\\ 1. Semester Master\\ se.busse@tu-bs.de}
\par \ \par

\paragraph{Master} \ \\
\npicture[0.3\linewidth]
{bilder/tutoren/martinw.jpg}
{Martin\\ 3. Semester Master\\ m.wegner@tu-bs.de}
\hfill
\npicture[0.3\linewidth]
{bilder/tutoren/henning.png}
{Henning\\ 4. Semester Master\\ h.guenther@tu-bs.de}
\hfill
\npicture[0.3\linewidth]
{bilder/tutoren/jan.jpg}
{Jan\\ 3. Semester Master\\ jhkluth@gmx.de}
\par \ \par
\npicture[0.3\linewidth]
{bilder/tutoren/brian.jpg}
{Brian\\ 3. Semester Master\\ b.schimmel@tu-bs.de}
\hfill
%\npicture[0.3\linewidth]
%{bilder/tutoren/stephan.jpg}%Christopher Lössl < c.loessl@tu-bs.de>
%{n\\ Christopher 2. Semester Master\\ c.loessl@tu-bs.de}
%\npicture[0.3\linewidth]
%{bilder/tutoren/stephan.jpg}
%{Stephan\\ 5. Semester Master\\ stephan.friedrichs@tu-bs.de}

\begin{multicols}{2}
%%% Local Variables: 
%%% mode: latex
%%% TeX-master: "../../1-te"
%%% End: 
