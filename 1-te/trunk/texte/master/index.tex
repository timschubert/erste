%WTF macht das hier?
\end{multicols}
\newpage
%\label{master}
\section{Spezielles im Master}
\begin{multicols}{2}
\label{master}

%TODO Eigentlich sollte in der index.tex kein Text sein, sondern nur includes für die anderen Texte.

Noch vor kurzem speiste sich der Informatik-Master der TU-Braunschweig fast ausschließlich durch die ansässigen Bachelor-Studenten, doch erfreulicherweise hat sich der Master nun auch anderswo herum gesprochen. Wer seinen Bachelor woanders erworben hat, steht im ersten Mastersemester vielen kleinen und mittelgroßen Schwierigkeiten gegenüber, denn die meisten Einfühungsveranstaltungen und -texte richten sich an Bachelor-Erstis, und längst nicht alles davon trifft auch auf neue Masterstudenten zu. Ich habe vor zwei Semestern selbst diesen Umstieg bzw. Umzug gewagt und versuche mal zu rekapitulieren, was mir leider damals keiner sagte\ldots




\subsection{Unterschiede zwischen den Bachelor-Abschlüssen}
Eventuell hat dein bisheriger Abschluss dir mehr als 180 Credit Points eingebracht - genau so viele hättest du nämlich in einem Bachelor an dieser TU erreicht. Theoretisch könnte es möglich sein, solche überschüssugen CPs auf den Master anzurechenen. Es ist uns nicht bekannt, ob das jemals jemand versucht hat, geschweige denn, ob es geklappt hätte, aber Fragen kostet ja zum Glück (auch hier) nichts.

Selbst bei gleicher Anzahl an CP ist der Bachelor an jeder Hochschule ein wenig anders, wobei Hochschule jetzt mal als Oberbegriff für Universität, Fachhochschule, Berufsakademie und viele andere Formen stehen soll. Zwischen den Universitäten in Deutschland herrscht eine formale Übereinkunft in den Inhalten, die in einem Bachelor-Studium namens "`Informatik"' vorkommen, daher wird in dem Fall von dir inhaltlich vermutlich nichts bedeutendes verlangt, was nicht auch an deiner Universität gelehrt wurde.

Falls du von einer Nicht-Universität (also z.B. Fachhochschule) und/oder aus einem Studiengang der sich nicht exakt "`Informatik"' nennt kommst und/oder dein Abschluss kein Bachelor of Science ist, dann kann es durchaus sein, dass du mit einer anderen Vorbildung hier her kommst als sie hier erwartet wird. Manche dieser Unterschiede sind schlichtweg egal, andere musst du selbst erkennen und ausgleichen, und bei gewissen Unterschieden "`wirst du geholfen"' diese zu beheben\ldots

\label{auflagen}
\subsection{Zulassungsauflagen}
Ob man für das Masterstudium zugelassen ist, lässt sich leider nicht komplett durch true und false ausdrücken, denn das Immatrikulationsamt belegt manche von euch mit so genannten Zulassungsauflagen. Ob du eine solche Auflagen bekommst, steht in einem der ersten Briefe, die du von der TU erhältst - und in der Regel nur dort, also heb diesen Brief gut auf! Wenn du keine solchen Zulassungsauflagen hast, kannst du den restlichen Abschnitt gerne überspringen.

Es handelt sich dabei um Fächer aus dem Informatik-Bachelor, die du zusätzlich zu den Master-Fächern noch belegen musst - für die Note und die zu erreichenden Credit Points im Master zählen sie nicht. Wenn diese innerhalb des ersten Jahres nicht erbracht werden, dann war es das (theoretisch) mit dem Master, und selbst wenn du die Auflagenfächer bestehst, aber dann nicht selbst dafür sorgst, diese Information vom Prüfungsamt zum Immatrikulationsamt zu tragen, droht nach dem zweiten Semester die Exmatrikulation. Lass dir diese Worte eine Warnung sein, aber sei beruhigt: wer durch ein Auflagenfach durchfällt oder den Nachweis vergisst, kann immernoch gewisse Schritte ergreifen, um weiter zu studieren - besser ist es aber, es nicht darauf ankommen zu lassen.

Der (eigentliche) Sinn hinter den Auflagen ist es, Defizite (im Vergleich zum TU-BS-Bachelor) auszugleichen, die du aus deinem vorherigen Studium mitbringst, d.h. Inhalte nachzuholen, die in deiner bisherigen Ausbildung zu kurz kamen oder ganz fehlten, und die für den erfolgreichen Abschluss des Masterstudiums nötig sind. Da nicht jeder die Diziplin hat, diese freiwillig aufzuarbeiten, versucht man dir hier durch den Zwang zu "`helfen"'.

Was vielleicht noch ganz nett klingt, ist im letzten Jahr ziemlich entartet, so dass jeder, der von der TU-Informatik-Bachelor-Norm abwich pauschal "`Theoretische Informatik 1"' und "`Einführung in die Logik"' nachholen musste - da war es dann auch egal, ob man die Fächer schon im Bachelor hatte und mit 1,0 abgeschlossen hatte. Ein Student hatte dagegen geklagt, und in Folge hatte er, und wenig später auch viele andere, die Auflage erlassen bekommen - einige wenige mussten sie trotzdem ableisten. Das einen solche Pflichten davon abhalten, die Inhalte nachzuholen, die einem wirklich fehlen, steht natürlich auf einem anderen Blatt.

Wir kennen das (hoffentlich verbessete) Auflagen-Vergabeverfahren für deinen Jahrgang nicht, aber hoffen natürlich, dass nun generell weniger eine Zulassungsauflage erhalten und wenn doch, dass sie dann wirklich die Inhalte betrifft, bei denen Defizite bestehen und die auch für den Master relevant sind. Bisher hieß solch eine Auflage, dass man die Vorlesung besuchen muss, die Hausaufgaben lösen und einreichen muss, und an der Klausur teilnehmen muss, die man auch unbedingt im ersten Versuch bestehen sollten. 

Für deinen Jahrgang ist nun einiges angenehmer geworden: Es soll nun erstmals möglich sein, zu Semesterbeginn freiwillig an einer mündlichen Prüfung teilzunehmen. Wird diese bestanden, dann ist die Auflage erfüllt, falls nicht, muss wie gehabt die Klausur belegt werden. Auch wird in den meisten Fächern die Hausaufgabe nicht mehr verpflichtend sein um an der Klausur teilzunehmen.

Falls du also eine Auflage erhalten hast, die dir fragwürdig erscheint, oder wenn man vergessen hat, dich über die möglichkeit der freiwilligen mündlichen Prüfung zu informieren, oder du sonst irgendwelche Fragen dazu hast, wende dich am besten an die Fachgruppe. Ratsam ist es auch, mit den anderen Erstis in deinem Jahrgang zu sprechen und zu vergleichen, wie deren Auflagen aussehen.

\subsection{Niveau ist keine Hautcreme\ldots}
\ldots aber auch nichts, was man direkt messen oder vergleichen kann. Die TU-Braunschweig hat eine recht hohe Meinung von ihrem Niveau - aber mal ehrlich, welche Hochschule würde auch etwas anderes von sich behaupten? In der Tat, es gibt hier hochqualitätive Lehre, engagierte Professoren und gewisse Mindestanforderungen an die Studierenden. Aber letzlich kochen auch hier alle mit Wasser, und man braucht als zugezogener Masterstudent keine übermäßige Angst vor dem Niveau-Unterschied zu haben. Der Sprung von der Schule zum Bachelorstudium war sicher größer - und den hast du ja offensichtlich geschafft, wenn du nun hier zum Masterstudium antrittst. Selbst wenn man "`nur"' von einer FH kommt - und die Vorbehalte bezüglich Fachhochschulen sind leider bei manchen Professoren groß - muss man nicht automatisch einen Einbruch im Notenschnitt befürchten.

Bemerkenswert ist, dass in vielen Master-Vorlesungen das Niveau mit fortschreitender Semesterzeit gegen unendlich strebt. Wenn nach 80\% des Semesters nur noch 5\% der Studenten verstehen, was gerade erklärt wird, ist das kein Grund zur Sorge - auch wenn du nicht zu diesen 5\% gehörst. Oft ist es so, dass man mit den ersten zwei Dritteln des Vorlesungsstoffes eine Note im 1er-Bereich bekommen kann - diese zwei Drittel sollte man dann natürlich möglichst perfekt beherrschen, und das ist auch nicht gerade einfach, aber machbar. Über kleinere Aussetzer und Fehler helfen praktisch alle Prüfer freundlich und beruhigend hinweg, schließlich soll Wissen geprüft werden, nicht Stressresistenz. Am besten schaut man dazu in eines der Prüfungsprotokolle, oft beruhigt das schon stark.

Die neuen Regeln bezüglich Abmeldung, Abwahl und nachträglichen Notenaufbesserung von Fächern machen es außerdem recht risikofrei, eine schwer wirkende Prüfung einfach mal auf sich zu kommen zu lassen. Auf keinen Fall sollte man sich vom scheinbar unerreichbaren Niveau einschüchtern lassen und den Großteil der Prüfungen last-minute abmelden.

\subsection{Selbstständiges Nachlernen von Bachelor-Fächern}
Unabhängig von Niveau und Anspruch hat dein Bachelor vielleicht eine andere Ausrichtung gehabt als man es hier gewohnt ist und somit in manchen Bereichen klare Wissenslücken hinterlassen. Wenn du das Gefühl hast, dass dir Wissen fehlt, das im Braunschweiger Bachelor vermittelt wird, kannst du dich natürlich auch freiwillig in jede Bachelor-Vorlesung oder Übung hineinsetzen - Punkte gibts dafür allerdings keine. Aber egal was dir aus dem Bachelor fehlt, es finden sich eigentlich stets genug Master-Fächer die auch ohne bestimmte Vorkenntnisse gut schaffbar sind. Einige wenige Master-Vorlesungen beginnen auch mit einer mehrwöchigen Widerholung der Bachelor-Grundlagen. Im Zweifelsfall frage Studenden aus den höheren Semestern oder den Prof selbt, welche Vorkenntnisse man wirklich braucht.

\label{masterstundenplan}
\subsection{Der eigene Stundenplan}
Es gibt irgendwo in den überabzählbar-unendlichen Weiten der TU-BS-Webseiten auch ein Tool names QIS oder QIP oder HIS oder sonstwie, mit dem du dir die eben erwähnte Untermenge zusammenstellen, speichern und ausdrucken kannst. Dort sind (bzw. waren vor einem Jahr) aber viele Fächer nicht eingetragen, was das Tool dann eher nutzlos macht. Parallel dazu gibt es noch das Stud.IP-Portal, welches ähnliche Funktionen anbietet, aber vermutlich noch unvollständiger und somit nutzloser ist. Wahrscheinlich hilft also nichts außer ein selbst erstellter Stundenplan. Wie kommt man also dahin?

Es gibt durchaus Studenten, die damit kein Problem haben: Sie schauen einige Minuten auf den Gesamstundenplan, es macht Klick, und sie wissen, welche Fächer sie belegen werden. Es gibt andere, nicht weniger schlaue, die bis zu 12 Stunden damit verbringen, bis sie ihren finalen Stundenplan beieinander haben. Falls du nicht zum unteren Extrem gehörst, soll dir dieser Text helfen, auch nicht zum oberen zu gehören:

Wenn du Zulassungsauflagen hast, haben diese oberste Priorität. Die entsprechenden Vorlesungen und Übungen kannst du ohne großes Nachdenken in deinen Stundenplan eintragen - außer wenn du die freiwillige mündliche Prüfung in Anspruch genommen und bestanden hast, bzw. du guter Hoffnung bist, sie zu bestehen.

Danach kannst du probieren, im allgemeien Stundenplan pro Block durchzugehen, und für jeden Block zu entscheiden, welches der dort stattfindenen Fächer für dich interessant klingt, und dieses herausschreiben oder markieren. Wenn du so vorgehst, hast du vermutlich am Ende einen Plan mit viel zu vielen Fächern, also deutlich mehr als 30 Credit Points. Und was zu Beginn noch Überscheidungsfrei aussieht, endet am Ende vielleicht in folgender Situation:

Vorlesung A überschneidet sich mit Vorlesung B und einem Übungstermin von Fach C. Der andere Übungstermin von Fach C kollidiert mit Vorlesung D, deren einzige Übung mit Übung A am Dienstag zusammenstößt, die man alternativ auch am Mittwoch haben könnte, was sich dann aber so halb mit der Vorlesung F überschneidet\ldots

Vielleicht springt dir nun sofort eine Lösung ins Auge. Falls nicht, hier noch ein paar Fälle, in denen eine vermeintliche Kollision gar keine ist, oder zumindest kein wirkliches Problem darstellt:

Manche Übungen finden nur alle zwei Wochen statt. Wenn also in einem Block die Übung zu Fach A und zu Fach B liegen, dann könntest du Glück haben, dass sich diese genau abwechseln. Dann ist aber wieder Vorsicht geboten, da die Lehrenden oft (z.B. wegen Urlaub, Krankheit, Konferenzen, Feiertag\ldots irgendein Grund findet sich immer) die Regelmäßigkeit mitten im Semester brechen und die zuvor abwechselnden Übungen dann wieder aufeinander liegen.

Man muss nicht immer beide Veranstaltungen besuchen: bei manchen Fächern kann man die Übung getrost weglassen, oder den Stoff auch ohne Vorlesung aus Skript und Büchern lernen und nur zur Übung kommen. Oder wenn sich Vorlesung X und die 14-täglich stattfindende Übung Y überschneiden, so kommt man halt nur alle zwei Wochen zur Vorlesung X. Nicht toll, nicht angenehm, aber oft machbar. Manche Institute filmen ihre Vorlesungen auch und machen sie somit auch zeitversetzt studierbar. Frage am besten höhere Semester nach ihren Erfahrungen mit dem betreffenden Fach.

Wenn es für ein Fach mehrere Übungstermine gibt, so sind diese meist für mehrere Übungsgruppen vorgesehen - du bist dann nur in einer dieser Gruppen und besuchst nur einen dieser Termine. Die Gruppe kann man meist frei wählen.

Außerdem passiert es recht oft, dass in den ersten Wochen noch Übungstermine bedarfsgerecht verschoben werden. Dadurch könenn sich Überschneidungen auflösen - aber natürlich auch neue dazu kommen.

\subsubsection{Immernoch keine Lösung?}
Nun ist Kreativität gefragt: Wende an, was du im Bachelor gelernt hast. Stelle die Kollisionen als Graph oder Matrix oder Tupelmenge dar, und lasse ein paar Algorithmen darauf los, die du dir ausdenkst und dann auf dem Papier simulierst. Bastel eine Excel-Tabelle mit Formeln und Makros oder schreibe ein kleines Programm, dass den optimalen Stundenplan berechnet. 

Spaß beiseite, auch die Auflistung in ihrer Schreibweise und Anhäufing leicht ironisch wirkt, sind das durchaus ernst gemeinte Vorschläge. In Extremfällen kann die Sache so vertrackt werden, dass ein paar hundert oder gar tausend Alternativen zu vergleichen sind. Von Hand kann und will das keiner, aber wenn zwei bis drei Stunden Informatiker-typisches gefrickel am Rechner dazu führen, den perfekten Stundenplan fürs nächste Semester zu finden, dann ist es die Sache doch wert.

\subsubsection{Hilfe beim Stundenplanbau}
Wie bieten dieses Jahr auch erstmals Hilfe zum Stundenplanbau an. Hätten wir das schon zu Beginn des Textes erwähnt, dann hätte ja niemand weitergelesen ;) Im Ernst: Würden zu Semesterbeginn 50 Stundenten unvorbereitet mit leeren Pänen zu uns kommen, so könnten wir jedem nur wenige Minuten helfen und letzlich wäre keinem so richtig geholfen. Wir hoffen, dass die obigen Ausführungen den meisten von euch reichen, um auch allein zum fast perfekten Stundenplan zu kommen - wenn es denn sowas gibt. Wer trotzdem noch total auf dem Schlauch steht, kann nun gerne zu uns kommen und sich intensiv und persönlich beraten lassen. Und auch wer glaubt, jetzt alles verstanden zu haben kann mit seinem fertigen Stundenplan gerne zu uns kommen, und wir schauen kurz drauf ob uns offensichtliche Probleme oder Verbesserungsvorschläge auffallen. Wir bitten dich also, vor dem Planbau-Termin ernsthaft zu versuchen, selbst zu einem Plan zu kommen. Und nicht nach 5 Minuten aufgeben: zwei bis drei Stunden kannst du ruhig dafür einplanen. Zu den oben genannten 12 Stunden muss man es ja nicht kommen lassen.

% TODO Dieser Teil kann wahrscheinlich ganz raus, oder?
\subsubsection{Und nun?}
Das war nun eine ganze Menge Text. Und nun weißt du alles, was du für deinen Weg zum Master-Abschluss brauchst? Ganz bestimmt nicht. Dies war die Notfallration für die ersten paar Tage und Wochen. In der ersten Semesterwoche gibt es eine ganze Reihe von Infoveranstaltungen, einige Wochen oder Monate später auch noch vereinzelze Termine mit den Infos, die du dann gerade brauchen wirst, und ab da hoffen wir, dass du dich gut eingefunden hast und dass bis dahin dein Jahrgang soweit zusammengewachsen ist, dass ihr euch gegenseitig auf dem Laufenden haltet bzw. Kontakt zu den höheren Semestern habt. Wenn doch noch Fragen bestehen, so gibts immer noch uns (die Fachgruppe) und diverse andere Ansprechpartner. Aber eine so geballte Packung Infos wirst du von hier an wohl nie wieder im Studium brauchen.


%Nicht nur für Master relevant (vieles davon steht vielleicht schon drin, hab die Liste geschrieben als ich die nte gerade nicht zur Hand hatte):
%Drucken
%GIZT-Konto
%ÖPNV
%Mensa
%Lageplan und Räume
%TUBS-Webseite und Institutsseiten
%Mailinglisten
%Bücher
%Lernräume
%Vollversammlungen
%Studiengebühren
%Wahlen
%Evaluationsbögen
%Prüfungsprototkolle

%%% Local Variables: 
%%% mode: latex
%%% TeX-master: "../../1-te"
%%% End: 
