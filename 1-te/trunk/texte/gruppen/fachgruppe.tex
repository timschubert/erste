\subsection{Fachgruppe bzw. Fachgruppenrat}

Die Fachgruppe Informatik besteht eigentlich aus allen 
Informatikstundenten, also ab jetzt auch aus euch. Der Fachgruppenrar 
ist die studentische Vertretung f"ur Studierende der Informatik, also 
eine Art "`Jahrgangssprecher"', die jedes Jahr von euch gew"ahlt werden 
und als Bindeglied zwischen den Studierenden und dem Fachbereich 
fungieren. Oft wird aber auch einfach "`Fachgruppe"' gesagt, wenn man 
eigentlich vom "`Fachgruppenrat"' spricht.

Unsere Hauptaufgabe ist die Vertretung eurer und unserer Meinung 
gegen"uber der Fakult"at in verschiedenen Kommissionen. Kommissionen 
gibt es an der Uni zuhauf, um die verschiedensten Angelegenheiten zu 
regeln. Ein Beispiel ist etwa die Studienkommission, in der ständig 
an den Studienabschl"ussen Bachelor und Master gefeilt wird. Den Bachelor 
gibt es zwar inzwischen schon seit einigen Jahren, aber dank des 
Bildungsstreiks in den Jahren 2009/2010 gibt es nun eine ganze Reihe 
von neuen (hoffentlich besseren) Regelungen, die eingearbeitet werden 
müssen.

Zus"atzlich versuchen wir, euch bei Fragen und Problemen rund um das 
Studium weiterzuhelfen. Besonders allen Erstsemestern stehen wir 
gerne mit Rat und Tat zur Seite. 
Es gibt zwei Einfühungstermine, an denen wir euch einiges rund um die 
Uni erz"ahlen m"ochten: ??????

Am ??? werden wir euch nach der Einf"uhrungsveranstaltung studentischen 
Tutoren zuteilen, mit denen ihr das Unigel"ande und Anderes erkunden 
k"onnt. Bei ihnen k"onnt ihr auch die ersten Fragen loswerden, wenn 
ihr nicht bei einem der beiden Beratungstermine davor wart.
Die Einteilung findet direkt nach der zweiten Infortmationsveranstaltung 
(IZ 160) statt.

Solltet ihr irgendwann sp"ater noch Fragen haben, k"onnt ihr euch 
praktisch immer an uns wenden. Ihr erreicht uns per Email unter 
fginfo@tu-bs.de. Wenn ihr uns persönlich Fragen stellen wollt, kommt 
Dienstags um 9:45 in unsere regelmäßige Sprechstunde, oder Dientag ab 
16:30 zum Fachgruppentreffen. Wir haben praktisch immer wichtige 
Neuerungen zu diskutieren und suchen permanent Unterstützung und 
Nachwuchs. Ab und zu beim Fachgruppentreffen herein zu schauen ist 
nicht nur der perfekte Einstieg, um sich selbst einmal einzubringen, 
sondern gibt euch euch die Möglichkeit, immer auf dem neuesten Stand 
der Entwicklungen zu bleiben, die euch letztlich selbst betreffen werden.

Falls sich der Termin mal "andern sollte (was zu Beginn eines neuen 
Semesters leider sehr wahrscheinlich ist), findet ihr die neue Zeit 
auf unserer Webseite \mbox{\nurl{http://fginfo.cs.tu-bs.de}}. Hier 
werdet ihr auch "uber aktuelle Veranstaltungen informiert, und k"onnt 
die Erstsemesterzeitung (die ihr gerade in Händen haltet), herunterladen. 
Am besten abonniert ihr unseren RSS-Feed, dann bekommt ihr automatisch 
mit, wenn es etwas neues gibt.