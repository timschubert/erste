\subsection{Fachgruppe}

Die Fachgruppe Informatik ist die studentische Vertretung f"ur Studierende der Informatik.
Wir sind eine Art "`Jahrgangssprecher"', die jedes Jahr von euch gew"ahlt werden und als Bindeglied zwischen den Studierenden und dem Fachbereich fungieren.

Unsere Hauptaufgabe ist die Vertretung eurer und unserer Meinung gegen"uber der
Fakult"at in verschiedenen Kommissionen. Kommissionen gibt es an der Uni zuhauf, um die verschiedensten Angelegenheiten zu regeln. Ein Beispiel ist etwa die Studienkommission, in der zur Zeit unter anderem an den neuen Studienabschl"ussen Bachelor und Master gefeilt wird. Da es den Bachelor hier in Braunschweig inzwischen drei Jahre gibt, kontrollieren wir, was fr"uher gut und was schlecht gelaufen ist und was ge"andert werden muss.

Zus"atzlich versuchen wir, euch bei Fragen und Problemen rund um das Studium weiterzuhelfen. Besonders allen Erstsemestern stehen wir gerne mit Rat und Tat zur Seite. 
Es gibt zwei Termine, an denen wir euch einiges rund um die Uni erz"ahlen m"ochten: Am Donnerstag, den 24.9. um 17 Uhr \& am Dienstag, den 20.10., nach dem Erstsemesterfr"uhst"uck um ca. 11 Uhr. \par
Am Dienstag werden wir euch nach der Einf"uhrungsveranstaltung studentischen Tutoren zuteilen, mit denen ihr das Unigel"ande und Anderes erkunden k"onnt. Bei ihnen k"onnt ihr auch die ersten Fragen loswerden, wenn ihr nicht bei einem der beiden Beratungstermine davor wart.
Die Einteilung findet direkt nach der zweiten Infortmationsveranstaltung (IZ 160) statt.

Solltet ihr irgendwann sp"ater noch Fragen haben, k"onnt ihr gern bei unserem Treffen im Fachgruppenraum, IZ Raum 150, vorbeikommen. Das liegt zur Zeit mittwochs um 16.30 Uhr. 
Falls sich der Termin "andern sollte, findet ihr die neue Zeit auf unserer Webseite \mbox{\nurl{http://fginfo.cs.tu-bs.de}}. \\
Hier werdet ihr auch "uber aktuelle Veranstaltungen informiert, k"onnt diese Erstsemesterzeitung, die Erste, herunterladen, oder das Forum nutzen.
Ihr k"onnt uns auch per Email unter fginfo@tu-bs.de erreichen.
