\subsection{Studentische Selbstverwaltung f"ur Dummies}
Seit die '68er durch die deutschen Universit"aten gefegt sind, ist Demokratie eingekehrt.
Doch was bedeutet das konkret f"ur euch?
\subsubsection*{Studierendenparlament}
Eines der wichtigsten Elemente der studentischen Mitbestimmung ist das Studierendenparlament (Uni-Slang: StuPa).
Es wird jedes Semester gew"ahlt und entscheidet unter anderem "uber den studentischen Haushalt, den ihr als Teil des Semesterbeitrags zahlt.
Au"serdem werden hier Aussch"usse gew"ahlt (Als wichtigster der "`Allgemeine Studierenden Ausschuss"', kurz AStA).
\subsubsection*{AStA}
Der Allgemeine Studierenden Ausschuss ist die "`Exekutive"' der Studenten:
Er vertritt euch nach Au"sen, also zum Beispiel bei Verhandlungen um das Semesterticket, versorgt euch mit Informationen zu politischen Themen ("ofter im Semester erscheint der so genannte "`AStA-Issue"') und ist einer der ersten Ansprechpartner f"ur eure Anliegen.
\subsubsection*{Fachgruppe}
Auch die Fachgruppe wird von euch gew"ahlt.
Allerdings w"ahlt hier jede Fachrichtung ihre eigene, ihr also Fachgruppe Informatik.
Die Fachgruppe unterst"utzt euch bei Infomatik-spezifischen Fragen, organisiert Veranstaltungen, wie zum Beispiel eure Erstsemester-Einf"uhrung und vertritt euch in verschiedenen Gremien.
\subsubsection*{Gremien}
In der Uni gibt es unz"ahlige Gremien, hier seien die wichtigsten genannt.
Jedes Gremium hat eine bestimmte Besetzung, also eine definierte Anzahl von jeweils Studenten, Mitarbeitern und Professoren.
Am relevantesten für euch ist die Studienkommission (\emph{StuKo}):
Hier werden Details des Studiengangs besprochen, Probleme der Studenten gekl"art und die Vergabe der Studiengebühren entschieden.
In diesem Gremium herrscht ein Stimmengleichgewicht zwischen Studenten und Professoren.
Das bedeutet, dass wir hier wirklich die M"oglichkeit haben, aktiv in die Unipolitik einzugreifen.

In der \emph{Informatik-Kommission} und im \emph{Fakult"atsrat} (der außerdem noch Mitglieder aus der Mathematik hat) sieht es da schon schlechter aus, die Studenten stellen in beiden nur eine Minderheit der Stimmen.

Die \emph{Berufungskommission} hat nur selten zu tun:
Wann immer eine Professur besetzt werden muss, tagt sie um Kandidaten f"ur das Amt zu finden.

Wann immer ihr Antr"age im Pr"ufungsamt stellt, landen diese im \emph{Pr"ufungsausschuss}, der entscheidet, ob diese rechtm"a"sig sind.
In diesem sind drei Professoren, ein Student und ein Mitarbeiter vertreten.

\subsection*{Vollversammlung}
Mindestens einmal im Semester findet eine Vollversammlung aller Studenten statt.
Nehmen genug Studenten teil, so k"onnen hier wichtige Themen abgestimmt werden, die alle Studenten betreffen, zum Beispiel wurde die Einf"uhrung des Semestertickets hier beschlossenx.