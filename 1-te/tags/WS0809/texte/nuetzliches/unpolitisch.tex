\subsection{Ich bin unpolitisch!}

Immer wieder h"ort man diese Aussage in Vorlesungen, in der Mensa und im 
Gespr"ach mit Studierenden beim Fachgruppenrat. F"ur die meisten Studierenden 
bedeutet diese Aussage, dass man "`kein Interesse an Politik"' hat oder 
zumindest keine Meinung zu aktuellen Vorg"angen.

\subsubsection*{Gem"a"sigtes Braunschweig}

Das Braunschweiger Umfeld macht es einem relativ leicht, sich politisch passiv 
zu verhalten. Hier herrscht nur ein recht kleines Spannungsfeld zwischen den 
traditionell eher rechten studentischen Verbindungen und den traditionell eher 
linken Fachschafts- und Fachgruppenr"aten sowie dem AStA. Diese 
gegen"uberstehenden Parteien werdet ihr in den allermeisten deutschen 
Universit"atsst"adten wiederfinden. W"ahrend es aber anderenorts so richtig 
kracht (Burschenschaftsh"auser werden mit Farbbeuteln beworfen und mit Parolen 
beschmiert, jeder "offentliche Auftritt von Burschenschaften f"uhrt zu 
Demonstrationen), ist Braunschweig ein gem"utliches Pflaster. AStA und die 
Fachschaften finden nur wenige Unterst"utzer und auch die Burschenschaften 
dominieren in Braunschweig nicht unbedingt das Stadtbild.

\subsubsection*{"`Schnell durchziehen!"'}

Einen erheblichen Beitrag zur "`Ist-mir-doch-egal"'-Haltung leistet meiner 
Ansicht nach die heute "ubliche, st"andig "uber Medien, Politiker oder auch die 
eigenen Eltern verbreitete Doktrin, dass man sein Studium "`schnell 
durchziehen"', zielstrebig, leistungs- und ich-orientiert seinen Abschluss 
ansteuern soll. Solche Leute will die Wirtschaft, daf"ur gibt es Preise und 
Stipendien. Langzeitstudenten werden bel"achelt und als Sozialfall angesehen. 
Unbequeme Themen wie ethische und religi"ose Fragen oder Umweltproblematik 
bleiben bei dieser Sichtweise als erstes auf der Strecke (z.B. gibt es in der 
Informatik in Braunschweig --~anders als zum Beispiel an der Uni Hamburg~-- 
keine Pflichtveranstaltung, die sich mit den gesellschaftlichen Einfl"ussen der 
Informatik auseinander setzt). Man hat das Gef"uhl, dass unm"undige, 
manipulierbare Arbeitnehmer heranzuz"uchtet werden sollen - den fr"uher 
propagierten "`breiten Horizont"' einer Hochschulausbildung konnte ich an 
unserer TU bisher nicht entdecken.

\subsubsection*{Verbindungen zur Politik}

Nun zur"uck zum \textbf{weit verbreiteten Ger"ucht, das eigene Studium habe 
doch nichts mit Politik zu tun}: Die Uni als Institution l"asst sich nicht von 
der Politik l"osen! Wir sind alle direkt betroffen von der Landespolitik (vor 
allem nat"urlich Bildungspolitik) und Lokalpolitik (z.B. Radwege, 
Attraktivit"at der Stadt). Au"serdem gibt es auch eine Uni-interne Politik, wie 
euch die "`Kleine Gremienkunde"' in diesem Heft schon ausf"uhrlich dargelegt 
hat. Wer sich z.B. in einem Institut umh"ort, wird dort nirgends 
Gleichg"ultigkeit gegen"uber der Bildungspolitik zu sp"uren bekommen. Ob 
Professorenstellen neu besetzt werden, ob gen"ugend HiWis f"ur kleine "Ubungen 
bezahlt werden, ob neue Ger"ate angeschafft werden, ob gar ganze Studieng"ange 
geschlossen werden, ob Studierende bei der Gestaltung ihrer Studieng"ange 
mitwirken d"urfen, ob "Offnungszeiten f"ur bestimmte Dienste verl"angert werden 
- all dies h"angt von der so viel geschimpften "`Politik"' der einen oder 
anderen Form ab. \textbf{Politik betrifft euch} und euer Studium. Direkt und 
ohne Wenn und Aber.

Nun will ich nat"urlich von niemandem verlangen, dass er einer Partei beitritt, 
Stra"senaktionen startet oder B"ucher schreibt. Aber zumindest ein kleines 
Interesse an eurem direkten Umfeld sollte doch selbstverst"andlich sein, oder? 
Es hat ja einen Grund, dass euer momentanes Studium so ist, wie es ist. Es gibt 
Studierende, die sich engagieren, die selbst etwas beitragen wollen, z.B. eine 
neue BPO (Bachelorpr"ufungsordnung) mit erarbeiten, f"ur mehr Computer oder 
l"angere "Offnungszeiten streiten etc., um unseren Studiengang und unser 
Hochschulleben attraktiver zu gestalten.

\subsubsection*{Informieren und Engagieren}

Wie kann man nun einen Einblick in das, was die Studierenden bewegen und was 
die Studierenden bewegt, gewinnen? Als erstes w"aren dort die hauptamtlichen 
Mitarbeiter des \textbf{AStA} zu nennen. Hinter der umst"andlichen Abk"urzung 
verbergen sich eine Handvoll Studierende, die entgegen weitl"aufiger Meinung 
weder Steineschmei"ser noch Nazis sondern Studierende wie ihr sind. Dann gibt 
es jeden Monat die \textbf{hochschul"offentliche Sitzung des 
Studierendenparlaments}. Dort tauscht man f"acher"ubergreifend Neuigkeiten aus 
und stimmt "uber entscheidende Dinge ab, z.B. "uber die Verwendung der 
studentischen Gelder, den studentischen Haushalt. Mindestens einmal im Semester 
gibt es die sogenannte VV, das ist die \textbf{studentische Vollversammlung} - 
wenn sie beschlussf"ahig ist, dann ist die Vollversammlung das h"ochste Gremium 
der Studierenden.
Schlie"slich finden einmal im Semester die \textbf{studentischen Wahlen} statt 
- da k"onnt ihr direkt oder indirekt (siehe Gremienkunde) bestimmen, welche 
Studierenden euch in den jeweiligen "Amtern vertreten sollen. Aus 
unerfindlichen Gr"unden ist die Wahlbeteiligung bei den studentischen Wahlen 
stets niedrig. Nehmt das als Aufmunterung -- bei geringer Beteiligung z"ahlt 
eure Stimme um so mehr!
