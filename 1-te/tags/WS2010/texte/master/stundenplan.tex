\subsection{Stunden- und Semesterplan}
 
Im Vergleich zum Bachelorstudium an der TU-BS bietet der Master
verblüffende Freiheiten in der Fächerwahl. Auch wer die Redewendung
bisher dumm fand, lernt spätestens hier was "`die Qual der Wahl"'
bedeutet. Daher sind dieses und die folgenden Unterkapitel auch in erster
Linie für Masterstudenten gedacht. Allerdings werden mit einigen
Themen auch Bachelorstudenten in Berührung kommen, aber erst ab den
3. Semester. Bei Unterschieden zwischen den Studiengängen gehen wir
kurz darauf ein.\\\\
Zu Beginn jedes Semesters muss man sich selbständig entscheiden, welche Fächer man belegen möchte (und kann) und sich danach den Stundenplan zusammenstellen. Im Gegensatz zu den alteingessenen Bacheloranden aus Braunschweig steht man recht unvorbereitet vor einer ganz und gar nicht trivialen Aufgabe, die man am besten noch vor dem ersten Studientag erledigen sollte. Vielleicht hatte dein Bachelor kaum Wahlmöglichkeiten und das Zusammenstellen eines eigenen Stundenplans ist für dich ungewohnt. Und selbst wenn du diese Freiheiten schon im Bachelor hattest, so sind die Rahmenbedinungen und Feinheiten hier sicherlich ganz andere.

Und anders als im Bachelor, bei dem die meisten Vorlesungen erst in der zweiten Woche beginnen, starten viele Mastervorlesungen schon in der ersten Woche, schlimmstenfall schon Montag früh. Die erste Vorlesung zu verpassen ist nun auch nicht so schlimm, aber wenn mans vermeiden kann\ldots Es ist übrigens auch nicht ganz leicht, den Vorlesungsstart eines Faches herauszufinden, somit wirst du in der ersten (und evtl. auch zweiten) Woche oft vor verschlossenen Türen oder leeren Räumen stehen.







\subsection{Studienplan und Reihenfolge}
Mindestens so frei wie die Wahl der Fächer ist auch die Wahl der Reihenfolge, in der man diese belegt. Es kursieren Studienpläne für den Master, die vorschlagen, bestimmte Module im ersten Semester zu machen (z.B. das Seminar), und andere im zweiten und dritten, etc. All dies sind höchstens gut gemeinte Tipps, letzlich muss ja nichtmals die Masterarbeit unbedingt am Ende stehen, und zur Idee, das Seminar im ersten Semester zu belegen, siehe auch den nächsten Absatz\ldots
          



