
% !TEX root = ../../1-te.tex

\subsection{Linux}
	\label{linux}
	Als Informatiker befasst man sich oft mit abstrakten und allgemeinen Konzepten, die unabhängig von konkreten Betriebssystemen gültig sind. Aber sobald man sich an einen Rechner setzt, hat man es dann doch mit einem konkreten System zu tun, und innerhalb der Rechnerpools an der Uni ist dies meist die eine oder andere Linux-Version. Du wirst also im Studium nicht drumherum kommen, etwas Erfahrung damit zu sammeln.

	Auf deinem eigenen Rechner kannst du natürlich machen, was immer du möchtest, aber viele von uns bevorzugen auch dort Linux oder ein anderes Unix-artiges System. Der Umstieg ist gar nicht so schwer wie man denkt bzw. wie er vor 10 Jahren mal war, und dank Live-CDs, Dual Boot und Virtualisierung kannst du sogar Linux und dein bisheriges System parallel laufen lassen und somit ganz unverbindlich reinschnuppern.

	Die Fachgruppe bietet im Rahmen der O-Woche eine Linux-Install-Party an, auf der du unter Anleitung und mit Unterstützung von erfahrenen Linux-Nutzern dein eigenes Linux installieren kannst. Sie findet dieses Semester am 10. Oktober um 16:00 Uhr im IZ 160 statt. \tocheck{5}{Wann findet die Linux-Install-Party dieses Semester statt?}


	\todo{5}{Unix-Basics hinzufügen, Kapitel fällt hinzwischen nur noch sehr mager aus.}

	\subsubsection{SSH -- Zugriff aus der Ferne}
        \todo{5}{So ganz richtig/hilfreich ist das hier ja nicht, oder?}
		Um vom heimischen PC aus Zugriff auf deinen Uniaccount zu haben, kannst du von Linux aus ssh benutzen. Für Windowsbenutzer gibt es zwei nette kleine Tools, Putty und Xming. %Deinen Uniaccount erreichst du über den Server \verUrl{0}{rzstudio.rz.tu-bs.de}.

		\begin{description}
			\item[Putty] stellt dir eine Shell auf dem UNIX-Rechner bereit. Damit kannst du so auf deinem Rechner arbeiten, als würdest du direkt auf dem Server arbeiten (tust du ja auch).  Download: \verUrl{6}{http://www.putty.org/}
			\item[Xming] Um auch grafische Programme starten zu können, musst du noch einen X-Server für Windows
			  installieren, z.B. Xming. Download: \verUrl{6}{http://sourceforge.net/projects/xming/}
		\end{description}

		Zu allen in diesem Text angesprochenen und noch zu vielen anderen Computerproblemen gibt es mehr Informationen im Heft \emph{Don't Panic}, das kostenlos im Rechenzentrum erhältlich ist und dir sehr wahrscheinlich auch per Post zugeschickt wurde.

	\subsubsection{Linux-Bezug an der TU}
		Fast alle Linux-Distributionen und Softwarepakete für Linux sind freie Software und somit kostenlos erhältlich.

		Für Studierende mit Breitband-Internetzugang sind vermutlich die diversen Mirror-Server an der Uni interessant. Hier stehen die größeren Distributionen bereit:

		\begin{description}
			\item[\verUrl{6}{http://www.knopper.net/knoppix-mirrors/}]~\\Offizielle Download-Seite für die Linux-Distribution Knoppix, welche einfach von einer CD oder einem USB-Stick gestartet werden kann
			\item[\verUrl{6}{http://debian.tu-bs.de/}]~\\Debian-, Kanotix- und Knoppixmirror, schön schnell aus dem Uni-Netz
			\item[\verUrl{6}{https://www.ibr.cs.tu-bs.de/kb/services.html}]~\\Verschiedene Online-Dienste des IBR
		\end{description}

%		Für Studierende ohne breitbandigen Netzzugang sind sicherlich die CDs nützlich, die sich jede/r im IT Service-Desk\footnote{\verUrl{4}{http://www.tu-braunschweig.de/it/service-desk}} im Gauß-IT-Zentrum, \textbf{Raum 017}, ausleihen kann. Dort stehen eigentlich immer die neusten Versionen von SuSE, Mandrake, Fedora, Gentoo, Debian und Knoppix sowie diverse ältere Distributionen zur Verfügung. Dank eines DVD-Brenners können inzwischen auch --~soweit vorhanden (SuSE, Knoppix)~-- die DVD-Versionen verliehen werden. Auf der sicheren Seite ist, wer vorher einen Abholtermin vereinbart, damit die gewünschte Distribution garantiert greifbar ist: 0531/391-5555.
