% !TEX root = ../../1-te.tex

\subsection{Linux}
	\label{linux}
	Als Informatiker befasst man sich oft mit abstrakten und allgemeinen Konzepten, die unabhängig von konkreten Betriebssystemen gültig sind. Aber sobald man sich an einen Rechner setzt, hat man es dann doch mit einem konkreten System zu tun, und innerhalb der Rechnerpools an der Uni ist dies meist die eine oder andere Linux-Version. Du wirst also im Studium nicht drumherum kommen, etwas Erfahrung damit zu sammeln.

	Auf deinem eigenen Rechner kannst du natürlich machen, was immer du möchtest, aber viele von uns bevorzugen auch dort Linux oder ein anderes Unix-artiges System. Der Umstieg ist gar nicht so schwer wie man denkt bzw. wie er vor 10 Jahren mal war, und dank Live-CDs, Dual Boot und Virtualisierung kannst du sogar Linux und dein bisheriges System parallel laufen lassen und somit ganz unverbindlich reinschnuppern.

	Die Fachgruppe bietet im Rahmen der O-Woche eine Linux-Install-Party an, auf der du unter Anleitung und mit Unterstützung von erfahrenen Linux-Nutzern dein eigenes Linux installieren kannst. Sie findet dieses Semester am 10. Oktober um 16:00 Uhr im IZ 160 statt. \tocheck{6}{Wann findet die Linux-Install-Party dieses Semester statt?}

        \tocheck{5}{Link auf Linuxparty-Vorbereitungsfoo im FG-Wiki einfügen}

%		Für Studierende ohne breitbandigen Netzzugang sind sicherlich die CDs nützlich, die sich jede/r im IT Service-Desk\footnote{\verUrl{4}{http://www.tu-braunschweig.de/it/service-desk}} im Gauß-IT-Zentrum, \textbf{Raum 017}, ausleihen kann. Dort stehen eigentlich immer die neusten Versionen von SuSE, Mandrake, Fedora, Gentoo, Debian und Knoppix sowie diverse ältere Distributionen zur Verfügung. Dank eines DVD-Brenners können inzwischen auch --~soweit vorhanden (SuSE, Knoppix)~-- die DVD-Versionen verliehen werden. Auf der sicheren Seite ist, wer vorher einen Abholtermin vereinbart, damit die gewünschte Distribution garantiert greifbar ist: 0531/391-5555.
