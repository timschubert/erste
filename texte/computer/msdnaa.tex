\begin{multicols}{2}
\subsection{Microsoft Academic Alliance}
	\label{msdnaa}
	Die TU hat seit 2003 eine Campuslizenz\footnote{http://msdn.microsoft.com/en-us/default.aspx} von Microsoft erworben, in deren Rahmen du Microsoftprodukte kostenlos beziehen kannst.\\ 
	Zur Auswahl stehen die meisten Betriebssysteme, Entwicklungswerkzeuge und diverse Serversoftware\footnote{\sloppy Eine komplette Liste der Software findet sich unter \url{http://msdn.microsoft.com/en-us/subscriptions/downloads/default.aspx}}. Die Office-Suite ist explizit \textbf{nicht} enthalten.

	Die Software darf zu nicht-kommerziellen Zwecken in Forschung und Lehre eingesetzt werden, jedoch keine Infrastrukturaufgaben erf"ullen\footnote{Die Nutzungsbedingungen sind nachzulesen unter \url{http://msdn.microsoft.com/en-us/academic/bb250609.aspx}}. Infos gibt es unter \url{https://www.tu-braunschweig.de/it/service-interaktiv/software/doku/msdn-aa}.

	Etwas paradox ist dabei, dass du ein laufendes Windows brauchst, um Software (also auch Windows selbst) herunterzuladen. Du kannst Microsoft Windows XP aber auch bei den Operateuren im Rechenzentrum in \textbf{Raum 015} f"ur eine Schutzgeb"uhr von 5 Euro erwerben, die "ubrige Software kannst du dort ausleihen oder unter \url{https://www.tu-braunschweig.de/it/service-interaktiv/software/doku/msdn-aa} downloaden.
\end{multicols}