% !TEX root = ../../1-te.tex

\subsection{Elektronisch informiert}
	\label{elekinf}
	Die wichtigsten Aufgaben der Studierenden sind der Besuch von Lehrveranstaltungen, Zeitmanagement für Studium und Freizeit und Informationsbeschaffung. In diesem Artikel geht es um den letzten Punkt. Da wir nun mal Informatik studieren, soll die Informationsbeschaffung über das Internet erfolgen.

	\subsubsection*{Mailinglisten}
	\label{mailinglisten}
		Die wichtigste Mailingliste für Informatikstudierende ist die Liste \textbf{cs-studs}. Sie ist \emph{die} Informationsquelle. Hier werden Ankündigungen zu Lehrveranstaltungen gemacht, die Fachgruppe kündigt hier Spiele- und Grillabende an und es gibt oft Angebote zu Hiwistellen oder offenen Teamprojekten, Bachelorarbeiten etc. und selbstverständlich ist dies auch ein guter Ort, um Fragen zum Studium loszuwerden.

		\tocheck{4}{Re-evaluate whether cs-studs-discuss is relevant}
		% Wer längere Diskussionen sucht, kann diese auf der Liste \textbf{cs-studs-discuss} finden bzw. führen. Diese Liste ist noch relativ neu und damit liegt es auch an dir, ihr Leben einzuhauchen.

		Da bei den Wirtschaftsinformatikern oftmals auch informatikrelevante Themen diskutiert werden, lohnt sich möglicherweise auch ein Blick in \textbf{winfo-studs}. 
		Wer an Stellenangeboten und Werbung aus der freien
		Wirtschaft interessiert ist, sollte Mailingliste
		\textbf{firmenkontakt} abonieren. Die
		Informatik-Kolloquien, das sind Vorträge von
		üblicherweise externen Referent/innen zu Informatik-Themen,
		werden auf der Mailingliste \textbf{kolloq} angekündigt.
		Unter
		\verUrl{4}{https://mail.ibr.cs.tu-bs.de/mailman/listinfo/}
		findest du eine umfassende Liste der angebotenen Mailinglisten in der Informatik.

	\subsubsection*{Stud.IP-Nachrichten weiterleiten}
	\label{studipfwd}
	    Einige Dozenten nutzen als Kommunikationsmedium außerhalb der Vorlesung neben den Mailinglisten oder Ankündigungen die Nachrichten-Funktion in Stud.IP. Diese birgt in der Standardeinstellung jedoch die Gefahr, dass lediglich Nutzer, die aktiv auf \verUrl{4}{https://studip.tu-braunschweig.de} vorbeischauen.
    Es gibt aber die Möglichkeit, sich die internen Nachrichten auch an die im System hinterlegte E-Mail-Adresse weiterzuleiten. Dies funktioniert wie folgt: Melde dich auf Stud.IP an, klicke oben rechts unter dem Logo auf \enquote{Einstellungen} und wähle den Reiter \enquote{Nachrichten} an. Dort musst du die Einstellung \enquote{Kopie empfangener Nachrichten an eigene E-Mail-Adresse schicken} auf \enquote{immer} abändern. 

	\subsubsection*{Chat}
		\tocheck{4}{Seit neustem ersetzen :)}
		Die Fachgruppe betreibt seit neustem einen öffentlichen Chatroom mit dem dezentralisierten Chatprotokoll Matrix\footnote{\verUrl{4}{https://matrix.org}}. Im Raum \texttt{\#fginfo:stratum0.org} sind immer ein paar BraunschweigerInnen und große Teile der Fachgruppe online. Der Raum kann mit geeigneten Clients oder per Web-Chat\footnote{\verUrl{4}{https://chat.stratum0.org/\#/room/\#fginfo:stratum0.org}} besucht werden.

		% Schon deutlich länger gibt gibt es den Channel \texttt{\#\#cs-studs} im Freenode IRC Chat \footnote{\verUrl{4}{http://freenode.net}}. Hier sind einige aktuelle und inzwischen hauptsächlich ehemalige Informatik-Studierende aus Braunschweig online. 

	% \subsubsection*{Clevershit}

	% 	Auf jeden Fall einen Besuch wert und eine gute Hilfe bei allem, was das Studium betrifft, ist das von Studierenden ins Leben gerufene Portal \mbox{\verUrl{2}{https://clevershit.de/}}.

	% 	Dieses von Studierenden für Studierende erstellte und geführte Plattform bietet eine gute Anlaufstelle für Fragen jeglicher Art. Es gibt eine Materialsammlung zu allen Veranstaltungen der ersten Semester.

\subsubsection*{Sonstige Informationen}
	\begin{description}
		\item[Allgemeines Vorlesungsverzeichnis:] ~\\
			{\footnotesize\verUrl{4}{https://vorlesungen.tu-bs.de}}
		\item[Uni-Bibliothek:] 
			{\footnotesize\verUrl{4}{https://ub.tu-braunschweig.de}}
		\item[Druckkosten:] ~\\
			{\footnotesize\verUrl{4}{https://www.tu-braunschweig.de/it/service-interaktiv/druckkosten}}
		\item[Speiseplan Mensa Hauptcampus:] ~\\
			{\footnotesize\verUrl{5}{http://www.stw-on.de/braunschweig/essen/menus/mensa-1\#heute}}
		\item[Don't Panic online] ~\\
			{\footnotesize\verUrl{4}{http://www.tu-braunschweig.de/Medien-DB/it/dontpanic.pdf}}
	\end{description}
