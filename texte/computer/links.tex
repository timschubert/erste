\begin{multicols}{2}
\subsection{Elektronisch informiert}
	\label{elekinf}
	Die wichtigsten Aufgaben der Studierenden sind der Besuch von Lehrveranstaltungen, Zeitmanagement für Studium und Freizeit und Informationsbeschaffung. In diesem Artikel geht es um den letzten Punkt, und da wir nun mal Informatik studieren, soll die Informationsbeschaffung über das Internet erfolgen.

	\subsubsection*{Mailinglisten}
	\label{mailinglisten}
		Die wichtigste Mailingliste für Informatikstudierende ist die Liste \textbf{cs-studs}. Sie ist \emph{die} Informationsquelle. Hier werden Ankündigungen zu Lehrveranstaltungen gemacht, eure Fachgruppe kündigt hier Spiele- und Grillabende an und es gibt oft Angebote zu Hiwistellen oder offenen Teamprojekten, Bachelorarbeiten etc. und selbstverständlich ist dies auch ein guter Ort, um Fragen zum Studium loszuwerden.

		Wer längere Diskussionen sucht, kann diese auf der Liste \textbf{cs-studs-discuss} finden bzw. führen. Diese Liste ist noch relativ neu und damit liegt es auch an euch, ihr Leben einzuhauchen.

		Da bei den Wirtschaftsinformatikern oftmals auch informatikrelevante Themen diskutiert werden, lohnt sich möglicherweise auch ein Blick in \textbf{winfo-studs}. Wenn ihr an Stellenangeboten und Werbung aus der freien Wirtschaft interessiert seid, steht die Mailingliste \textbf{firmenkontakt} zu eurer Verfügung. Die Informatik-Kolloquien, das sind Vorträge von üblicherweise externen Referenten zu Informatik-Themen, werden auf der Mailingliste \textbf{kolloq} angekündigt. Alle bisher genannten Mailinglisten sind über \url{http://www.cs.tu-bs.de/mailinglisten.html} erreichbar. Außerdem findet ihr unter \url{https://mail.ibr.cs.tu-bs.de/mailman/listinfo/} eine umfassendere Liste der angebotenen Mailinglisten in der Informatik.

	\subsubsection*{IRC}
		Im Freenode IRC Chat (\url{http://freenode.net}) gibt es den Channel \url{##cs-studs}. Hier sind immer ein paar BraunschweigerInnen und große Teile der Fachgruppe online. Die Gesprächsthemen haben (im weitesten Sinne ;) mit dem Studium zu tun.

	\subsubsection*{Clevershit}
		Auf jeden Fall einen Besuch wert und eine gute Hilfe bei allem, was das Studium betrifft, ist das von Studenten im letzten Jahr ins Leben gerufene Portal \mbox{\url{http://www.clevershit.de}}.\\
		Diese von Studenten für Studenten erstellte und geführte Plattform bietet eine gute Anlaufstelle für Fragen jeglicher Art. In der Wiki der Seite gibt es eine Materialsammlung zu allen Veranstaltungen der ersten Semester. Im gut besuchten Forum werden stets aktuelle Informationen und Änderungen zu den Vorlesungen weiter gegeben, Hausaufgaben und Klausuren diskutiert oder einfach etwas Smalltalk gehalten.

	\subsubsection*{Sag's uns}
		\emph{Sag's uns} ist ein Blog, der im Auftrag des Präsidiums und in Kooperation mit dem Institut für Wirtschaftsinformatik, insbesondere Informationsmanagement, von Studierenden für Studierende entwickelt wurde und Anfang des Jahres 2009 an den Start ging.

		Eure Ideen, Kritiken, Anregungen und Belobigungen werden hier transparent von zentraler Stelle aus schnellstmöglich bearbeitet und moderiert, die zuständigen Einrichtungen der TU werden zur Absprache und Beantwortung einbezogen. Wir freuen uns über unsere Studierenden und Mitarbeiter, die dazu beitragen, dass der Blog sehr effizient und zielführend eingesetzt und auch angenommen wird.

		Gute Ideen und Lösungen werden bei \emph{Sag's uns} veröffentlicht, so dass auch andere davon profitieren können. Darüber hinaus kannst du die Ideen anderer bewerten – und umgekehrt. Natürlich kannst du auch Anfragen stellen, ohne dass sie veröffentlicht werden, ebenso wie du anonym bleiben kannst, wenn dir das lieber ist. 

		Unter \url{https://sagsuns.tu-braunschweig.de/} kannst du loswerden, was dir an der Universität wichtig ist.

		Noch Fragen? Die Referentin für Studienqualität, Frau Dipl.-Päd. Anja Reisch, steht Dir als Ansprechpartnerin in der Geschäftsstelle des Präsidiums zur Verfügung (sagsuns@tu-braunschweig.de; 0531 391 4109).

	\begin{description}
		\item[Allgemeines Vorlesungsverzeichnis:] ~\\
			{\footnotesize\url{http://vorlesungen.tu-bs.de}}
		\item[Uni-Bibliothek:] ~\\
			{\footnotesize\url{http://www.biblio.tu-bs.de}}
		\item[Druckkosten:] ~\\
			{\footnotesize\url{https://www.tu-braunschweig.de/it/service-interaktiv/druckkosten}}
		\item[Don't Panic online] ~\\
			{\footnotesize\url{http://www.tu-braunschweig.de/Medien-DB/it/dontpanic.pdf}}
		\item[Putty Homepage] ~\\
			{\footnotesize\url{http://www.putty.org}}
		\item[WinSCP Homepage] ~\\
			{\footnotesize\url{http://winscp.net}}
	\end{description}
\end{multicols}
