%\begin{multicols}{2}
	\subsection{Wozu Computer?}
		\subsubsection{Vorlesungen Online}
			Zu den meisten Vorlesungen kann man die Skripte im Internet finden. Für einige Vorlesungen gibt es sogar Ton- oder Videomitschnitte.

			Es gibt auch immer engagierte Studierende, die ihre Vorlesungsmitschriften online stellen. Da diese sehr wahrscheinlich in deinem Semester sind, hilft es, wenn du dich in den Vorlesungen umhörst. Ansonsten ist \url{https://www.clevershit.de} die richtige Anlaufstelle für den Informationsaustausch zwischen Studenten.

		\subsubsection{Organisatorisches ohne Papier}
			Ansonsten gibt es eine Reihe von Informationen, die man nur über das Web bekommt, und mehr und mehr Formalitäten (z.B. die Prüfungsanmeldung) werden auch in die virtuelle Relatität verlagert.

			Desweiteren kannst du dir im Netz einen individuellen Stundenplan zusammenstellen, in Erfahrung bringen, wann die nächsten Klausuren stattfinden, lesen, was es in der Mensa zu essen gibt, endlich herausfinden, wann das Prüfungsamt geöffnet hat,  offene HiWi"~Stellen bei den Instituten finden und vieles mehr.

			Die Webseiten der TU sind ein großer Dschungel, durch den man sich am besten  mit Machete und Googlesuche kämpft. Um an der TU etwas zu finden, solltest du deinem eigentlichen Suchbegriff wahlweise \enquote{tu braunschweig} oder \enquote{site:tu-braunschweig.de} anhängen, und schon hast du gute Chancen zum Ziel zu kommen.

		\subsubsection{Mitschreiben am PC}
			Auf den ersten Blick mag es naheliegen, sich
			während der Vorlesungen Notizen  am Laptop
			anzufertigen. In der Praxis gibt es da aber eine
			Reihe von Problemen, vor denen wir  warnen
			möchten. Es hat schließlich seinen Grund, das
			nur 10\% der Studenten in der Vorlesung am
			Laptop sitzen und davon 90\% diesen nur nutzen,
			um zu zocken: Die meisten Tafelanschriften
			bestehen  aus verschachtelten Formeln,
			fremdartigen Buchstaben und verworrenen
			Zeichnungen. Diese in Echtzeit in den Laptop
			einzuhacken ist eine besondere Kunst, die du mit
			Notepad und Word gar nicht erst probieren
			brauchst. Eine Chance hast du vielleicht mit
			einem Tablet PC, oder wenn du
			\LaTeX\ bereits im Schlaf beherrscht -
			aber wer tut das schon zu Beginn des Studiums?

			In den Vorlesungen, in denen du nicht Tafelweise abschreiben musst, sondern nur hier und da mal etwas notieren, zeigt sich der PC schon als nützlicher. Wenn du ab und zu den Vortrag des Profs damit vergleichen möchtest, was er in sein Skript geschrieben hat, kann dir der mitgebrachte Laptop unter Umständen das Ausdrucken von ein paar hundert Seiten ersparen. Du wirst aber schnell merken, dass es in praktisch keinem der Hörsääle und Seminarräume Steckdosen gibt, und in manchen nichtmals ausreichende WLAN-Signalstärke.

		\subsubsection{Hausaufgaben am PC}
			In vielen Fächern musst du regelmäßig
			Hausaufgaben erledigen und einreichen. Keiner
			erwartet von dir, dass diese mit dem PC gemacht
			werden, es ist also völlig ok sie von Hand zu
			schreiben. Es hat aber auch gewisse Vorteile,
			sie am PC zu schreiben (z.B. mittels \LaTeX) und
			dann auszudrucken. Bei \LateX\ handelt es sich
			um ein Satzsystem für wissenschaftliche
			Texte wie Haus- oder Abschlussarbeiten.
			Erwähnenstwert ist die hervorragende
			Unterstützung für den Satz mathematischer
			Formeln und dass dabei mit Befehlen, ähnlich wie
			in HTML gearbeitet wird. Es gibt \LaTeX-Kurse die du im Schlüsselqualifiktationsbereich anrechnen lassen kannst, aber mit den Infos im WWW kann man sich das auch selbst beibringen. Je eher du damit anfängst, desto weniger Probleme hast du später, wenn du damit z.B. deine Abschlussarbeit aufsetzt.

		\subsection{Computer-Pools an der Uni}
			Es ist immer nützlich zu wissen, wo man mal schnell an einen Computer kann. Zumindest ab und zu wirst du die Computer in der Uni benutzen, besonders die Linuxarbeitsplätze in \textbf{PK4.5} oder \textbf{PK4.8}, an denen du die Hausaufgaben für Programmieren abgeben musst.

			\begin{itemize}
				\item[*] Im Erdgeschoss des Altbaus gibt es auf der rechten Seite zwei Computerräume, einen weiter vorne (\textbf{PK4.6}) und einen genau in der Ecke des Gebäudes (\textbf{PK4.5}). Zwei weitere Räume (\textbf{PK4.8} und die \textbf{Datenstation}) findest du im ersten Stock des Altbaus, auch wieder in der rechten Ecke. Die Rechner in \textbf{PK4.5} und \textbf{PK4.8} sind mit Linux ausgestattet. Im ersten Stock gibt es nun auch einen Windowsrechnerraum. Da kann man mal eine Word- oder Powerpoint-Datei ausdrucken, wenn man denn muss.

				\item[*] Reichlich Computer findet man schließlich im Gauß-IT-Zentrum~(GITZ) an der Hans-Sommer-Straße. Das ist der gedrungene, fast würfelförmige, dunkle Klotz hinter dem Elektrotechnik-Hochhaus (\emph{E-Tower}). Hier gibt es mehrere frei zugängliche Räume mit  Linux- und Windowsrechnern. Es gibt hier auch Räume für Medienbearbeitung, wo du etwa Video-Digitalisierer, ein Tonstudio und Rechner mit der Adobe Creative Suite Production Premium nutzen kannst.

				\item[*] Seit 2010 stellt das IBR (Institut für Betriebssysteme und Rechnerverbund) im Raum G40 des Informatikzentrums einen Rechnerraum mit vielen, schnellen Linux-Rechnern  zur Verfügung. Zu diesem CIP-Pool (Computer-Investitions-Programm) bekommt man mit seiner y-Nummer Zutritt. Wenn man Glück hat, funktioniert sogar einer der beiden Drucker in diesem Raum, so dass man zum Drucken nicht das IZ verlassen muss.
			\end{itemize}

		\subsection{Der eigene Rechner}
			Wenn du trotz aller Widrigkeiten planst, dir
			extra für's Studium einen (tragbaren) Rechner
			anzuschaffen, dann hast du hier gleich ein wenig
			Kaufberatung: Viel (Rechen- bzw.
			Grafik-)Leistung brauchst du im Studium  nur für
			sehr wenige spezielle Fachgebiete - das
			einfachste Netbook wird also vermutlich schon
			reichen. Wichtiger ist vielmehr die Akkulaufzeit
			und die WLAN-Empfangsstärke. %a die Här	

		\subsubsection{Welches System?}
			Dir wird auffallen, dass zwar alle Systeme geduldet sind, aber die Linux hier deutlich öfter über den Weg laufen wird als in der freien Wildbahn. Auch wir sind große Linux-Fans und haben deshalb ab Seite \pageref{linux} ein paar Infos dazu zusammengetragen.

			Aber trotz dieser nicht ganz unauffälligen Beeinflussung gilt: Beim Betriebssystem hast du freie Wahl. Sämtliche Software, die du für's Studium brauchen  könntest, gibt es für alle großen Systeme, meist sogar gratis. Für Linux ist eh  praktisch alles frei erhältlich, für Windows spendiert Microsoft den Studenten auch alles außer Office (siehe Seite \pageref{msdnaa}), und auch Apple bringt dich dank satter Studentenrabatte durch Bachelor und Master. 

		\subsubsection{Wege ins Uni-Netz}
			Um den eigenen Rechner ins Netz zu bekommen, stehen die in der Uni WLAN und LAN offen. Zur Konfiguration siehe Seite \pageref{wlan}.

			Für manche Aktivitäten (z.B. den Zugriff auf Prüfungsprotokolle) musst du dich direkt im Uni-Netz befinden. Wenn du und dein Rechner aber gerade zuhause oder sonstwo sind, heißt dass nicht, dass du dich nun physisch auf den Weg machen musst. Mittels VPN kannst du dich virtuell ins Uni-Netz einklinken. Schau einfach mal auf den Seiten des GITZ nach, um mehr zu erfahren.
%\end{multicols}
