\subsection{Microsoft DreamSpark}
	\label{msdnaa}
	Die TU hat  2003 eine Campuslizenz von Microsoft erworben, in deren Rahmen du Microsoftprodukte kostenlos beziehen kannst.\\ 
	Zur Auswahl stehen die meisten Betriebssysteme, Entwicklungswerkzeuge und diverse Serversoftware\footnote{\sloppy Eine komplette Liste der Software findet sich unter \verUrl{1}{https://www.dreamspark.com/Student/Software-Catalog.aspx}}. Die Office-Suite ist explizit \textbf{nicht} enthalten.

	Die Software darf zu nicht-kommerziellen Zwecken in Forschung und Lehre eingesetzt werden, jedoch keine Infrastrukturaufgaben erfüllen\footnote{Die Nutzungsbedingungen sind nachzulesen unter \verUrl{1}{https://www.dreamspark.com/}}. Infos gibt es unter \verUrl{1}{https://www.tu-braunschweig.de/it/service-interaktiv/software/doku/msdn-aa}.

	Du brauchst ein laufendes Windows, um Software (also auch
	Windows selbst) herunterzuladen. Du kannst  bei den Operateuren
	im Rechenzentrum in \textbf{Raum 015} eine Windows-CD gegen eine Schutzgebühr von ca. 10 Euro erwerben, die übrige Software kannst du dort ausleihen oder unter \verUrl{1}{https://www.tu-braunschweig.de/it/service-interaktiv/software/doku/msdn-aa} downloaden.
