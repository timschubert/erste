% !TEX root = ../../1-te.tex

\subsection{Gauß-IT-Zentrum}

	Das Rechenzentrum der TU-Braunschweig heißt Gauß-IT-Zentrum (oder kurz GITZ). Es bietet dir eine Vielzahl an Diensten an. Manche davon kannst du nur vor Ort, also in der Hans-Sommer-Str. 65, direkt hinter dem E-Tower nutzen, andere sind auch in den Außenstellen, wie z.B. im Altgebäude zu finden. Das allermeiste lässt sich über das Netz an der gesamten Uni oder sogar weltweit in Anspruch nehmen.

\subsubsection{GITZ-Account}
\label{todogitz}
	Das GITZ stellt diverse Dienste zur Verfügung, wovon manche quasi lebenswichtig sind, andere eher nebensächlich. Aber für all diese Dienste brauchst du eine GITZ-Account-Nummer und ein Passwort. Diese sogenannte y-Nummer ist nicht das gleiche wie deine Immatrikulationsnummer. In der Regel bekommst du schon vor Semesterbeginn eine Nummer und ein vorläufiges Passwort per Post zugesendet. Dieses Passwort brauchst du dir nicht merken, denn du kannst es nur verwenden, um dir ein richtiges Passwort für die spätere Verwendung auszusuchen. Das solltest du schnellstmöglichst erledigen, da du sonst die Dienste des GITZ (z.B. WLAN, die Pool-Rechner etc.) nicht nutzen kannst.
	
	Es kann auch passieren, dass du den besagten Brief vom GITZ  gar nicht bekommst, dann gehst du einfach selbst zum GITZ in die Hans-Sommer-Straße und besorgst dir dort einen. Keine Sorge, das passiert halt ab und an, ist aber nicht weiter schlimm.

	\subsubsection{E-Mailadresse}
		Zusammen mit deinem GITZ-Account bekommst du auch ein neues E-Mail-Postfach mit zwei Adressen (y0000000@tu-bs.de, v.nachname@tu-bs.de oder vorname.nachname@tu-bs.de). Deine E-Mails kannst du bequem mit dem Webmail Dienst (\verUrl{6}{https://groupware.tu-braunschweig.de/}) im Browser oder in einem E-Mail-Client deiner Wahl abrufen. Anleitungen dazu gibt es ebenfalls online auf den Seiten des GITZ.

	\subsubsection{WLAN}
		\label{wlan}
		WLAN wird vom Rechenzentrum praktisch auf dem ganzen Campus angeboten. Alle notwendigen Informationen, um das \emph{eduroam} nutzen zu können, findest du auf folgender Website: \verUrl{6}{http://www.tu-braunschweig.de/it/dienste/11/1106}

		Das \emph{eduroam} ist ein international standardisierter Zugang, der an vielen europäischen Hochschulen funktioniert. Einmal eingerichtet kannst du also mit deinen TU-BS-Zugangsdaten problemlos an anderen Unis surfen.

		Die Anleitungen der TU-Braunschweig werden dir nahelegen, eine spezielle Software nachzuinstallieren. Es geht aber für alle aktuellen Betriebssysteme auch ohne, also nur mit Boardmitteln -- um herauszufinden wie, schau einfach im Netz nach, was andere Unis zu \emph{eduroam} zu sagen haben. Du solltest, um sicher zu surfen das Zertifikat der Uni installieren.

		An einigen Orten gibt es auch Ethernet-Dosen, z.B. in der Uni-Bibliothek (z.T. versteckt unter runden Klappen im Boden, z.T. an der Fensterseite freiliegend), dem Informatik-Zentrum, sowie einigen Rechnerräumen im Altgebäude und Rechenzentrum. Ein Kabel dazu musst du dir selbst mitbringen.

    \subsubsection{Wege ins Uni-Netz}
        Für manche Aktivitäten musst du dich direkt im Uni-Netz befinden. Wenn du und dein Rechner aber gerade zuhause oder sonstwo seid, kannst du dich mittels VPN virtuell ins Uni-Netz einklinken. Schau einfach mal auf den Seiten des GITZ\footnote{\verUrl{6}{https://www.tu-braunschweig.de/it/dienste/11/1105}} nach, um mehr zu erfahren.
