\subsection{Gau"s-IT-Zentrum}
Das Rechenzentrum der TU-Braunschweig heißt Gau"s-IT-Zentrum oder kurz GITZ. Es bietet euch eine Vielzahl an Diensten an. Manche davon könnt ihr nur vor Ort nutzen, also in der Hans-Sommer-Str. 65, direkt hinter dem ,E-Tower'. Den wundersch"onen ,braunen W"urfel' findet Ihr z.B. im Standplan \nurl{http://stadtplan.braunschweig.de}.

Andere Dienste sind auch in den Außenstellen, wie z.B. im Altgebäude zu finden, und das allermeiste könnt ihr über das Netz an der gesamten Uni oder sogar weltweit in Anspruch nehmen.

% TODO Die folgenden drei subsections wurden aus dem Kapitel "Ersticheckliste" hierher verschoben, aber noch nicht geprüft, ob jetzt hier was doppelt oder komisch ist.
\subsection{GITZ-Account}
\label{todogitz}
Unser Rechenzentrum, das Gauß-IT-Zentrum, stellt euch diverse 
Dienste zur Vefügung, wovon manche quasi lebenswichtig sind, 
andere eher nebensächlich. Aber für all diese Dienste braucht 
ihr eine GITZ-Account-Nummer und ein Passwort. Diese so genannte 
y-Nummer ist nicht das gleiche wie eure Immatrikulationsnummer. 
In der Regel bekommt ihr schon vor Semesterbeginn eine Nummer 
und ein vorläufiges Passwort per Post zugesendet. Dieses 
Passwort müsst ihr euch nicht mehrken, denn ihr braucht es nur 
einmal, nämlich um sich ein richtiges Passwort für die spätere 
Verwendung auszusuchen. Das solltet ihr auf jeden Fall möglichst 
früh von einem eigenen PC von zuhause aus machen (denn ohne das 
gemacht zu haben, stehen euch die Uni-PCs nicht zur Verfügung, 
und ihr kommt in der Uni auch noch nicht ins WLAN). Dann solltet 
ihr euch alle drei wichtigen Daten - Matrikelnummer, Y-Nummer 
und das neue Passwort gut einprägen (ihr braucht sie dann 
ständig zu den unmöglichsten Zeiten), gegebenenfalls auch 
aufschreiben und sicher verwahren.

Es kann auch passieren, dass ihr den besagten Brief vom GITZ 
gar nicht bekommt, dann müsst ihr euch selbst um all das kümmern. 
Keine Sorge, das passiert halt ab und zu, ist aber nicht weiter 
schlimm.

\subsubsection{Emailadresse}
\label{todomailing}
Zusammen mit eurem GITZ-Account bekommt ihr auch ein neues 
Email-Postfach mit bis zu drei Adressen (y00000000@tu-bs.de, 
vorname.nachname@tu-bs.de, v.nachname@tu-bs.de). Für die oben 
genannte Mailingliste, und diverse andere Zwecke, könnt ihr euch 
meist aussuchen, ob ihr eure vorherige private Emailadresse nutzt, 
oder die neue von der TU-Braunschweig. Aber egal wie ihr euch 
entscheidet, ab und zu erreichen euch auch Emails auf eurem 
TU-Braunschweig-Postfach, also schaut dort regelmäßig rein! Wer mit der TU-Mailadresse nichts zu tun haben möchte\footnote{Es gibt immer wieder mal technische Probleme damit, weshalb viele es bevorzugen, selbst für studienspezifische Dinge nicht die TU-Adresse zu verwenden.}, sollte sich zumindest eine Weiterleitung auf seine Hauptadresse einrichten.

\subsubsection{IRC-Channel und Forum/Wiki}

Viele Studierenden der Informatik, Nebenfachh"orer und
Fachgruppenmitglieder sind im IRC-Channel \texttt{\#\#cs-studs}
(ja, der zweite ,,\#'' ist korrekt) auf \texttt{irc.freenode.net}
unterwegs. Auch hier ist ein guter Ort, Fragen zu stellen.

Unter \url{http://clevershit.de} findet ihr außerdem ein Forum und ein 
Wiki extra für Informatiker an der TU-Braunschweig, auf dem ihr Fragen 
stellen könnt und extrem viele nützliche Infos für's Studium findet. Um 
euch dort anzumelden, braucht ihr übrigens die TU-Braunschweig-Emailadresse, 
die ihr vom GITZ bekommt.

\label{kopieren}
\subsubsection{Drucken und Kopieren}
Es gibt viele Gründe, etwas zu Drucken, von 1000-Seitigen Skripten über am Rechner angefertigte Hausaufgaben bis hin zu Formularen die ihr online erhaltet aber nur offline einreichen dürft. Zur Wahl stehen euch der heimische Drucker (falls vorhanden), diverse Copyshops im Uni-Viertel und die Drucker des GITZ.

Dabei ist das GITZ mit Abstand konstengünstigste Alternative, da es euch sämtliche Aufträge zum Selbstkostenpreis erfüllt. Praktisch gesehen kannst du dort sogar kostenlos drucken, denn alle Druckaufträge werden von einem persönlichen Druckkostenkonto abgebucht, dass sich zu Beginn jedes Semesters auf magische Weise auf 15,00 Euro regeneriert. (Naja, da diese 15 Euro aus deinen 500 Euro Studienbeiträgen kommen, ist es da mit der Magie und der Kostenlosigkeit so eine Sache\ldots)

Bislang war das Drucken im GITZ sehr nervenaufreibend, es sei denn man wartet gerne mehr als eine Stunde auf ein einzelnes Blatt Papier. Pünktlich zum Semesterwechsel wird nun das System umgestellt, und in Zukunft soll alles besser werden. Also gebt dem GITZ eine Chance, und probiert es mal aus. Es kostet euch ja schließlich nichts - außer Zeit. Die Drucker des GITZ findet ihr im GITZ-Gebäude, im Altgebäude und im Raum G40 im IZ.

Kopieren\footnote{Das Kopieren hat weder mit Computern noch mit dem GITZ etwas zu tun, aber passt trotzdem so schön hierher} könnt ihr auch sehr kostengünstig an der Uni. In der Bibliothek stehen euch verschiedene Kopierer zur Verfügung, von denen manche Kleingeld schlucken und andere eine Kopierkarte erfordern, die ihr für 5 Euro am Schalter erwerben könnt. Ansonsten bleibt euch auch hier der Copyshop als Alternative.

\subsubsection{WLAN}
\label{wlan}
WLAN wird vom Rechenzentrum in vielen H"ors"alen (wie dem \nroom{Audimax} und
\nroom{SN19.1}), im IZ, in der Universit"atsbibliothek (UB) und im GITZ angeboten - 
also fast überall außer der Mensa.
Notebookbesitzer finden auf folgender Webseite alle notwendigen Informationen, um 
das "`eduroam'' nutzen zu k"onnen.
\nurl{http://www.tu-braunschweig.de/it/dienste/11/1106}

Das "`eduroam'' ist ein international standartisierter Zugang, der an vielen 
europäischen Hochschulen funktioniert. Einmal eingerichtet kannst du also mit 
deinen TU-BS-Zugangsdaten problemlos an anderen Unis surfen.

Die Anleitungen der TU-Braunschweig werden dir nahelegen, eine spezielle Software 
nachzuinstallieren. Es geht aber für alle aktuellen Betriebssysteme auch ohne, also nur mit 
Boardmitteln - um herauszufinden wie, schau einfach im Netz nach, was andere 
Unis zu "`eduroam'' zu sagen haben. Für Windows XP (und eng verwandte Versionen) 
bietet z.B. die Uni Graz eine schöne Anleitung.

% TODO ist das nicht totaler Quatsch? eduroam ist doch über WPA 2 Enterprise gesichert und somit wahrscheinlich deutlich sicherer als Ethernet.
Aber Vorsicht beim kabellosen Vergn"ugen. Unverschl"usselt "ubertragene
Passw"orter (z.B. bei ftp, http, pop3 und imap) k"onnen alle WLAN
Benutzer in deinem Umkreis mith"oren. Also verwende immer "uber SSL
gesicherte Protokolle, wenn du sensible Daten "ubertr"agst.

Wer etwas schneller unterwegs sein will (oder wessen Empfang überhaupt 
nicht ausreicht), dem sei das normale Ethernet ans
Herz gelegt. Ein Kabel dazu musst du dir selbst mitbringen. Dosen zum
Anschlie"sen gibt es in der Uni"~Bibliothek (z.T. versteckt unter runden
Klappen im Boden, z.T. an der Fensterseite frei liegend) und im
Rechenzentrum (im Laptopraum \nroom{R003} und in \nroom{R001} zwischen den
Mappits).

\subsubsection{Noch ein bisschen Text}
Wem der klassische Kommunikationsweg per Email \nurl{it-zentrum@tu-braunschweig.de} oder Internet \mbox{\nurl{http://www.tu-braunschweig.de/it}} zu schwierig erscheint, kann auch per Telefon (0531-391 5555) oder pers"ohnlich das Rechenzentrum besuchen, "ahm das hei"st ja jetzt \textbf{G}au"s-\textbf{IT}-\textbf{Z}entrum, Wer den weiten Weg nicht scheut, der findet au"ser den Linux-Distributionen noch viele weitere n"utzlige Features und Gadjets, die hin und wieder das Leben und Studium vereinfachen. Angefangen mit \textbf{A} wie Antworten zu Problemen rund um Euren Account (y-Nummer, etc.)
"uber \textbf{B} wie B"ucher "uber g"angige IT-Themen wie Betriebssysteme, Netze oder Programmiersprachen. Eine "ubersicht dieser sehr g"unstigen und oft guten Zusammenstellungen findet Ihr auf \nurl{http://www.tu-braunschweig.de/it/service-desk/rrzn-handbuecher}. Weiter geht es mit \textbf{K} wie Kurse \nurl{https://www.tu-braunschweig.de/it/service-interaktiv/kurse} zu g"angigen Programmen wie zum Beispiel Maya, Photoshop oder auch AutoCAD sowie PHP oder auch C-Programmierung und nat"urlich Java. Diese werden f"ur Studierende zumeist kostenlos vom GITZ angeboten. Am besten Ihr schaut einfach selber unter \textbf{S} wie Dienstleistungen\footnote{bis vor kurzem hieß dieser Punkt noch \textit{Services}} \nurl{http://www.tu-braunschweig.de/it/dienste} und bekommt eine Übersicht der angebotenen Ger"aten, Scannern, Software und Kursen.\newline
Der aufmerksame Leser der GITZ-Seiten ist bestimmt "uber den Abschnitt mit seinem Workspace gestolpert. Jeder Studi hat ungef"ahr 250~MB zur freien Ferf"ugung, er kann sich auch einen Ordner anlegen, der im Netz erreichbar ist, also f"ur statische HTML-Seiten, oder per FTP Dateien von Zuhause auf den Uni-Account schieben, damit diese dann in der Uni abrufbar sind.\newline
Eine mehr oder wenig "Ubersichtliche Linksammlung findet Ihr unter \nurl{http://www.tu-braunschweig.de/it/hotlinks}, so auch zum Beispiel \textbf{Z} wie Zusammenstellung der wichtigsten Befehle f"ur Linux, das ,Don't Panic' \nurl{http://www.tu-braunschweig.de/Medien-DB/it/dontpanic.pdf}
und wem all diese Informationen doch nicht weiter geholfen haben, der sollte mal \textit{man man} ausprobieren...