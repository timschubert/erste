\subsection{Gauß-IT-Zentrum}

	Das Rechenzentrum der TU-Braunschweig heißt Gauß-IT-Zentrum oder kurz GITZ. Es bietet dir eine Vielzahl an Diensten an. Manche davon kannst du nur vor Ort nutzen, also in der Hans-Sommer-Str. 65, direkt hinter dem ,E-Tower'. 
	
	Andere Dienste sind auch in den Außenstellen, wie z.B. im
	Altgebäude zu finden und das allermeiste lässt sich über das Netz an der gesamten Uni oder sogar weltweit in Anspruch nehmen.

\subsubsection{GITZ-Account}
\label{todogitz}
	Unser Rechenzentrum, das Gauß-IT-Zentrum, stellt  diverse Dienste zur Vefügung, wovon manche quasi lebenswichtig sind, andere eher nebensächlich. Aber für all diese Dienste brauchst du eine GITZ-Account-Nummer und ein Passwort. Diese so genannte y-Nummer ist nicht das gleiche wie eure Immatrikulationsnummer. In der Regel bekommt man schon vor Semesterbeginn eine Nummer und ein vorläufiges Passwort per Post zugesendet. Dieses Passwort brauchst du dir nicht merken, denn man kann es nur verwenden, um  sich ein richtiges Passwort für die spätere Verwendung auszusuchen. Das sollte man schnellstmöglichst erledigen, da man sonst die Dienste des GITZ (z.B: WLAN, die Pool-Rechner etc) nicht nutzen kann. 
	
	Es kann auch passieren, dass du den besagten Brief vom GITZ  gar nicht bekommst, dann gehst du einfach selbst zum GITZ in die Hans-Sommer-Straße und besorgst dir dort einen. Keine Sorge, das passiert halt ab und zu, ist aber nicht weiter schlimm.

	\subsubsection{Emailadresse}
		Zusammen mit eurem GITZ-Account bekommt ihr auch ein neues Email-Postfach mit bis zu drei Adressen (y00000000@tu-bs.de, vorname.nachname@tu-bs.de, v.nachname@tu-bs.de). Leider kommt es dabei manchmal zu Problemen, also nicht wundern, wenn euch Emails mal mit kleiner Verzögerung erreichen. 

	\subsubsection{WLAN}
		\label{wlan}
		WLAN wird vom Rechenzentrum in vielen Hörsälen (wie dem \textbf{Audimax} und \textbf{SN19.1}), im IZ, in der Universitätsbibliothek (UB), der Mensa und im GITZ angeboten. Notebookbesitzer finden auf folgender Webseite alle notwendigen Informationen, um das \emph{eduroam} nutzen zu können. \url{http://www.tu-braunschweig.de/it/dienste/11/1106}

		Das \emph{eduroam} ist ein international standardisierter Zugang, der an vielen europäischen Hochschulen funktioniert. Einmal eingerichtet kannst du also mit deinen TU-BS-Zugangsdaten problemlos an anderen Unis surfen.

		Die Anleitungen der TU-Braunschweig werden dir nahelegen, eine spezielle Software nachzuinstallieren. Es geht aber für alle aktuellen Betriebssysteme auch ohne, also nur mit Boardmitteln - um herauszufinden wie, schau einfach im Netz nach, was andere Unis zu \emph{eduroam} zu sagen haben. Für Windows XP (und eng verwandte Versionen) bietet z.B. die Uni Graz eine schöne Anleitung.

		Wer etwas schneller unterwegs sein will (oder wessen Empfang überhaupt nicht ausreicht), dem sei das normale Ethernet ans Herz gelegt. Ein Kabel dazu musst du dir selbst mitbringen. Dosen zum Anschließen gibt es in der Uni-Bibliothek (z.T. versteckt unter runden Klappen im Boden, z.T. an der Fensterseite freiliegend), dem Informatik-Zentrum, sowie einigen Rechnerräumen im Altgebäude und Rechenzentrum.
