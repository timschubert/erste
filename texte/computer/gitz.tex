%\begin{multicols}{2}
\subsection{Gauß-IT-Zentrum}

	Das Rechenzentrum der TU-Braunschweig heißt Gauß-IT-Zentrum oder kurz GITZ. Es bietet dir eine Vielzahl an Diensten an. Manche davon kannst du nur vor Ort nutzen, also in der Hans-Sommer-Str. 65, direkt hinter dem ,E-Tower'. 
	%Den wunderschönen ,braunen Würfel' findet Ihr z.B. im Standplan \url{http://stadtplan.braunschweig.de}.

	Andere Dienste sind auch in den Außenstellen, wie z.B. im
	Altgebäude zu finden, und das allermeiste lässt sich über das Netz an der gesamten Uni oder sogar weltweit in Anspruch nehmen.

\subsubsection{GITZ-Account}
\label{todogitz}
	Unser Rechenzentrum, das Gauß-IT-Zentrum, stellt  diverse
	Dienste zur Vefügung, wovon manche quasi lebenswichtig sind,
	andere eher nebensächlich. Aber für all diese Dienste brauchst du eine GITZ-Account-Nummer und ein Passwort. Diese so genannte y-Nummer ist nicht das gleiche wie eure Immatrikulationsnummer. In der Regel bekommt man schon vor 
	Semesterbeginn eine Nummer und ein vorläufiges Passwort per Post
	zugesendet. Dieses Passwort brauchst du dir nicht merken, denn man kann es nur verwenden, um  sich ein richtiges Passwort für die spätere Verwendung auszusuchen.
	Das sollte man schnellstmöglichst erledigen, da man sonst die
	Dienste des GITZ (z.B: WLAN, die Pool-Rechner etc) nicht nutzen
	kann. 
	%Das solltet ihr auf jeden Fall möglichst früh von einem eigenen PC von zuhause aus machen (denn ohne das gemacht zu haben, stehen euch die Uni-PCs nicht zur Verfügung, und ihr kommt in der Uni auch noch nicht ins WLAN). Dann solltet ihr euch alle drei wichtigen Daten - Matrikelnummer, Y-Nummer und das neue Passwort gut einprägen (ihr braucht sie dann ständig zu den unmöglichsten Zeiten).

	Es kann auch passieren, dass du den besagten Brief vom GITZ  gar
	nicht bekommst, dann gehst du einfach selbst zum GITZ in die
	Hans-Sommer-Straße und besorgst dir dort einen. eine Sorge, das passiert halt ab und zu, ist aber nicht weiter schlimm.

	\subsubsection{Emailadresse}
		Zusammen mit eurem GITZ-Account bekommt ihr auch ein
		neues Email-Postfach mit bis zu drei Adressen
		(y00000000@tu-bs.de, vorname.nachname@tu-bs.de,
		v.nachname@tu-bs.de). Leider kommt es dabei manchmal zu
		Problemen, also nicht wundern, wenn euch Emails mal mit
		kleiner Verzögerung erreichen. 
		%erse andere Zwecke, könnt ihr euch meist aussuchen, ob ihr eure vorherige private Emailadresse nutzt, oder die neue von der TU-Braunschweig. Aber egal wie ihr euch entscheidet, ab und zu erreichen euch auch Emails auf eurem TU-Braunschweig-Postfach, also schaut dort regelmäßig rein! Wer mit der TU-Mailadresse nichts zu tun haben möchte\footnote{Es gibt immer wieder mal technische Probleme damit, weshalb viele es bevorzugen, selbst für studienspezifische Dinge nicht die TU-Adresse zu verwenden.}, sollte sich zumindest eine Weiterleitung auf seine Hauptadresse einrichten.


%	\subsubsection{Drucken und Kopieren}
%		\label{kopieren}
%		Es gibt viele Gründe, etwas zu Drucken, von 1000-Seitigen Skripten über am Rechner angefertigte Hausaufgaben bis hin zu Formularen die ihr online erhaltet aber nur offline einreichen dürft. Zur Wahl stehen euch der heimische Drucker (falls vorhanden), diverse Copyshops im Uni-Viertel und die Drucker des GITZ.
%
%		Dabei ist das GITZ mit Abstand konstengünstigste Alternative, da es euch sämtliche Aufträge zum Selbstkostenpreis erfüllt. Praktisch gesehen kannst du dort sogar kostenlos drucken, denn alle Druckaufträge werden von einem persönlichen Druckkostenkonto abgebucht, dass sich zu Beginn jedes Semesters auf magische Weise auf 15,00 Euro regeneriert. (Naja, da diese 15 Euro aus deinen 500 Euro Studienbeiträgen kommen, ist es da mit der Magie und der Kostenlosigkeit so eine Sache\ldots)

%		Bislang war das Drucken im GITZ sehr nervenaufreibend, es sei denn man wartet gerne mehr als eine Stunde auf ein einzelnes Blatt Papier. Pünktlich zum Semesterwechsel wird nun das System umgestellt, und in Zukunft soll alles besser werden. Also gebt dem GITZ eine Chance, und probiert es mal aus. Es kostet euch ja schließlich nichts - außer Zeit. Die Drucker des GITZ findet ihr im GITZ-Gebäude, im Altgebäude und im Raum G40 im IZ.

%		Kopieren\footnote{Das Kopieren hat weder mit Computern noch mit dem GITZ etwas zu tun, aber passt trotzdem so schön hierher} könnt ihr auch sehr kostengünstig an der Uni. In der Bibliothek stehen euch verschiedene Kopierer zur Verfügung, von denen manche Kleingeld schlucken und andere eine Kopierkarte erfordern, die ihr für 5 Euro am Schalter erwerben könnt. Ansonsten bleibt euch auch hier der Copyshop als Alternative.

	\subsubsection{WLAN}
		\label{wlan}
		WLAN wird vom Rechenzentrum in vielen Hörsälen (wie dem \textbf{Audimax} und \textbf{SN19.1}), im IZ, in der Universitätsbibliothek (UB) und im GITZ angeboten - also fast überall außer der Mensa. Notebookbesitzer finden auf folgender Webseite alle notwendigen Informationen, um das \emph{eduroam} nutzen zu können. \url{http://www.tu-braunschweig.de/it/dienste/11/1106}

		Das \emph{eduroam} ist ein international standardisierter Zugang, der an vielen europäischen Hochschulen funktioniert. Einmal eingerichtet kannst du also mit deinen TU-BS-Zugangsdaten problemlos an anderen Unis surfen.

		Die Anleitungen der TU-Braunschweig werden dir nahelegen, eine spezielle Software nachzuinstallieren. Es geht aber für alle aktuellen Betriebssysteme auch ohne, also nur mit Boardmitteln - um herauszufinden wie, schau einfach im Netz nach, was andere Unis zu \emph{eduroam} zu sagen haben. Für Windows XP (und eng verwandte Versionen) bietet z.B. die Uni Graz eine schöne Anleitung.

%		Aber Vorsicht beim kabellosen Vergnügen. Unverschlüsselt übertragene Passwörter (z.B. bei ftp, http, pop3 und imap) können alle WLAN Benutzer in deinem Umkreis mithören. Also verwende immer über SSL gesicherte Protokolle, wenn du sensible Daten überträgst.

		Wer etwas schneller unterwegs sein will (oder wessen
		Empfang überhaupt nicht ausreicht), dem sei das normale
		Ethernet ans Herz gelegt. Ein Kabel dazu musst du dir
		selbst mitbringen. Dosen zum Anschließen gibt es in der
		Uni"~Bibliothek (z.T. versteckt unter runden Klappen im
		Boden, z.T. an der Fensterseite frei liegend), dem
		Informatik-Zentrum, sowie einigen Rechnerräumen im
		Altgebäude und Rechenzentrum.
		%und im Rechenzentrum (im Laptopraum \textbf{R003} und in \textbf{R001} zwischen den Mappits).

	% Viel zu viele Fehler nonsens und geflame...
	% \subsubsection{Noch ein bisschen Text}
	% 	Wem der klassische Kommunikationsweg per Email \url{it-zentrum@tu-braunschweig.de} oder Internet \url{http://www.tu-braunschweig.de/it} zu schwierig erscheint, kann auch per Telefon (0531-391 5555) oder persönlich das \textbf{G}auß-\textbf{IT}-\textbf{Z}entrum aka Rechenzentrum besuchen. Wer den weiten Weg nicht scheut, der findet außer den Linux-Distributionen noch viele weitere nützliche Features und Gadgets, die hin und wieder das Leben und Studium vereinfachen. Angefangen mit \textbf{A} wie Antworten zu Problemen rund um Euren Account (y-Nummer, etc.) ¨Uber \textbf{B} wie Bücher über gängige IT-Themen wie Betriebssysteme, Netze oder Programmiersprachen. Eine übersicht dieser sehr günstigen und oft guten Zusammenstellungen findet Ihr auf \url{http://www.tu-braunschweig.de/it/service-desk/rrzn-handbuecher}. Weiter geht es mit \textbf{K} wie Kurse \url{https://www.tu-braunschweig.de/it/service-interaktiv/kurse} zu gängigen Programmen wie zum Beispiel Maya, Photoshop oder auch AutoCAD sowie PHP oder auch C-Programmierung und natürlich Java. Diese werden für Studierende zumeist kostenlos vom GITZ angeboten. Am besten Ihr schaut einfach selber unter \textbf{S} wie Dienstleistungen\footnote{bis vor kurzem hieß dieser Punkt noch \textit{Services}} \url{http://www.tu-braunschweig.de/it/dienste} und bekommt eine Übersicht der angebotenen Geräten, Scannern, Software und Kursen.

		% Der aufmerksame Leser der GITZ-Seiten ist bestimmt über den Abschnitt mit seinem Workspace gestolpert. Jeder Studi hat ungefähr 250~MB zur freien Ferfügung, er kann sich auch einen Ordner anlegen, der im Netz erreichbar ist, also für statische HTML-Seiten, oder per FTP Dateien von Zuhause auf den Uni-Account schieben, damit diese dann in der Uni abrufbar sind. 

		% Eine mehr oder wenig übersichtliche Linksammlung findet Ihr unter \url{http://www.tu-braunschweig.de/it/hotlinks}, so auch zum Beispiel \textbf{Z} wie Zusammenstellung der wichtigsten Befehle für Linux, das ,Don't Panic' \url{http://www.tu-braunschweig.de/Medien-DB/it/dontpanic.pdf} und wem all diese Informationen doch nicht weiter geholfen haben, der sollte mal \textit{man man} ausprobieren...
%\end{multicols}
