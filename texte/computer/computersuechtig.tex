\subsection*{\Large{Du bist computersüchtig, wenn\ldots}}

\begin{enumerate}
\item \ldots du eine Viertelstunde brauchst, um durch deine Bookmarks zu scrollen.
\item \ldots du deinen Lautsprecher aufdrehst, bevor du das Zimmer verläßt, damit du das akustische Signal hörst, wenn eine neue E-Mail eintrifft.
\item \ldots dein Hund eine eigene Homepage hat.
\item \ldots du deine Mutter nicht anrufen kannst, weil sie kein VoIP-Telefon hat.
\item \ldots du deine Mail abrufst, die Meldung kommt: \emph{No new messages} - und du sie gleich nochmal abrufst.
\item \ldots du das Geschlecht von dreien deiner besten Freunde nicht kennst, weil sie neutrale Nicknames haben und du sie nie danach gefragt hast.
\item \ldots du morgens um 3 Uhr aufwachst, zum Klo gehst und auf dem Rückweg am Computer halt machst, um deine Mailbox abzurufen.
\item \ldots du dich tätowieren lässt: \emph{Diesen Körper betrachtet man am besten mit Mozilla 5.0 oder höher}.
\item \ldots dein Partner sagt, dass das Gespräch für eine Beziehung wichtig ist, also kaufst du einen zweiten Rechner und richtest ihm/ihr einen IRC-Client ein.
\item \ldots dir jemand einen Witz erzählt und du mit *lol* antwortest.
\item \ldots du deinen Freunden von einer heißen Verabredung erzählst und ihnen verschweigst, dass sie in einem Chatroom stattfindet.
\item \ldots du dir einen Laptop kaufst, um auch auf dem Klo surfen zu können.
\item \ldots du auf eine Webseite schaust, die voll mit Links von jemand anderem ist, und alle Links bereits in Lila erscheinen.
\item \ldots dich dein Provider bei technischen Schwierigkeiten um deine Hilfe bittet.
\item \ldots du bei \url{http://www.wetter.de/} nachschaust, anstatt aus dem Fenster.
\item \ldots Google bei dir anfragt, was noch in ihrer Suchmaschine fehlt.
\item \ldots du deinen Kopf zur Seite beugst, um zu lächeln.
\item \ldots deine Kaffeemaschine eine eigene IP hat.
\item \ldots du versuchst Texte aus deinem handgeschriebenen Script per copy and paste in ein \LaTeX-Dokument einzufügen.
\item \ldots du keine Kiste mit alten Computerteilen hast, weil z.B. der alte 386er noch als Anrufbeantorter genutzt wird.
\item \ldots du deine HiFi-Anlage über einen eigens dafür aufgesetzten Webserver steuerst.
\item \ldots du bei vier Webbrowserspielen unangefochten auf Platz 1 stehst.
\item \ldots du weißt, was man unter\\ \url{http://www.google.de/search?&amp;q=5\%5E2\%2B23\%2D3\%21&amp;btnG=Suche&amp;meta=} findet.
\end{enumerate}
