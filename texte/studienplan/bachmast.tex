\begin{multicols}{2}
\subsection{Zwei Studiengänge unter einem Hut}
	Seit der Bologna-Reform gibt es an der TU Braunschweig zwei Studiengänge - \textit{Bachelor und Master}. Viele Informationen über das Studium betreffen beide, deshalb ist diese Zeitung für alle Erstsemester. Nachdem der allgemeinen Einleitung folgen die speziellen Abschnitte für Bachelor (ab S. \pageref{bachelor}) und Master-Ersties (ab Seite \pageref{master}).

\subsubsection{Herden, Rudel und Einzelgänger}
	Bevor es in die Untiefen der Prüfungsordnungen und formalen Anforderungen geht, ein paar Worte zu einem sozialen Phänomen. Der recht feste Stundenplan im Bachelor-Studium sorgt dafür, dass man dort in der Regel mit vielen Kommilitonen zusammensitzt, die in der gleichen Situation sind wie man selbst: Neu hier und mit den gleichen Fragen und Sorgen. Und ist ein Block zu Ende, so zieht man gemeinsam zum nächsten Raum, wo man mit praktisch der gleichen Gruppe das nächste Fach abgrast. Eine typische Herde also.

	Im Master ist das grundlegend anders. Jeder hört andere Vorlesungen, und in den \emph{Mastervorlesungen} tummeln sich nicht nur Masterstudenten, sondern auch Bachelor- und Diplom- oder gar fachverwandte Studenten, wie z.B. Wirtschaftsinformatik. Da kann es eine ganze Weile dauern, bis man weiß, wer auch im Masterstudium ist und gegebenfalls auch noch im gleichen Jahrgang. Selbst dann haben diese Leute ihren Bachelor hier oder dort, in diesem oder jenem Fach an einer Uni oder FH gemacht. Vielleicht haben die neben dir zuvor ganz andere Dinge gelernt, vielleicht sind sie hier um sich auf etwas komplett anderes zu spezialisieren als du.

	Keine Frage: Diese Mischung macht es spannender, bunter und vielseitiger, aber auf jeden Fall auch schwieriger. Wir können hier kaum Tipps geben, wie man als Neuling und eventuell unfreiwilliger Einzelgänger ein kleines Rudel findet oder bildet. Weder wir noch dieses Heft könnten all das ersetzen, was eine Gruppe von Gleichgesinnten mit gleichen Problemen und Interessen könnte. Aber wir wissen, dass man in den ersten Tagen und Wochen viele Fragen hat, und gerade als Master oft nur wenige an der Seite, die die gleichen Fragen und/oder passende Antworten haben. Deshalb dieses Heft.

	Um deine Kommilitonen schneller kennen zu lernen gibt es unter anderem die vielfältigen Angebote der Fachgruppe (Spieleabende, Kneipentouren, Grillen, etc.) - siehe \url{http://fginfo.cs.tu-bs.de/}
\end{multicols}{2}