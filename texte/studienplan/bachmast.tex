\begin{multicols}{2}
\subsection{Zwei Studiengänge unter einem Hut}
	Es ist schon nicht ganz einfach: Dank Bologna-Reform haben wir nun zwei Studiengänge - \textit{Bachelor und Master} - die aufeinander aufbauen und doch parallel zueinander laufen. Diese Zeitung wendet sich an Erstsemester beiderlei Sorte, denn viele Infos sind einfach allgemeingültig. Im folgenden Kapitel gibt es aber vereinzelte Ausnahmen und viele Abschnitte mit kleinen Einschüben extra für Bachelor und Master. Nachdem ein grundlegendes Fundament gelegt wurde, folgen dann auch nochmal ein ganzes Kapitel nur für Bachelor-Erstis (ab S. \pageref{bachelor}) und eines für angehende Master (ab Seite \pageref{master}).

	Als Bachelor-Ersti ist man meist noch nicht sicher, ob man sich später auch noch den Master antun\ldots ertragen\ldots genießen möchte, und daher könnte es auch jetzt schon interessant sein, sich anzuschauen, was später auf einen zukommen könnte. Umgekehrt ist es für Master-Studenten, die von woanders zu uns stoßen, ganz praktisch zu wissen wie der TU-BS-Bachelor aufgebaut ist. Wir ermutigen euch also, bei Zeiten gerne auch in das Kapitel der anderen zu schnuppern. Nun aber zu dem, was für euch alle relevant ist.

\subsubsection{Herden, Rudel und Einzelgänger}
	Bevor es in die Untiefen der Prüfungsordnungen und formalen Anforderungen geht, ein paar Worte zu einem sozialen Phänomen. Der recht feste Stundenplan im Bachelor-Studium sorgt dafür, dass man dort in der Regel mit vielen Kommilitonen zusammensitzt, die in der gleichen Situation sind wie man selbst: Neu hier und mit den gleichen Fragen und Sorgen. Und ist ein Block zu Ende, so zieht man gemeinsam zum nächsten Raum, wo man mit praktisch der gleichen Gruppe das nächste Fach abgrast. Eine typische Herde also.

	Im Master ist das Grundlegend anders. Jeder hört andere Vorlesungen, und in den oft so genannten \emph{Mastervorlesungen} tummeln sich Bachelor-, Master- und Diplomstudenten aus diversen Jahrgängen, oft auch aus anderen artverwandten Studiengängen wie z.B. Wirtschaftsinformatik. Da kann es eine ganze Weile dauern, bis man gecheckt hat, wer nun auch im Masterstudium ist und gegebenfalls auch noch im gleichen Jahrgang. Selbst dann haben diese Leute ihren Bachelor hier oder dort, in diesem oder jenem Fach an einer Uni oder FH gemacht. Vielleicht haben die neben dir zuvor ganz andere Dinge gelernt, vielleicht sind sie hier um sich auf etwas komplett anderes zu spezialisieren als du.

	Keine Frage: Diese Mischung macht es spannender, bunter und vielseitiger. Aber auf jeden Fall auch schwieriger. Wir können hier kaum Tipps geben, wie man als Neuling und eventuell unfreiwilliger Einzelgänger ein kleines Rudel findet oder bildet (denn wenn man dem Informatiker-Clichee glaubt, wissen wir das nämlich selbst nicht\ldots). Weder wir noch dieses Heft könnten all das ersetzen, was eine Gruppe von Gleichgesinnten mit gleichen Problemen und Interessen könnte. Aber wir wissen, dass man in den ersten Tagen und Wochen viele Fragen hat, und gerade als Master oft nur wenige an der Seite, der die gleichen Fragen und/oder passende Antworten haben. Deshalb wollen wir euch alle wichtigen Infos mitgeben, die man nicht unbedingt rechtzeitg per Hörensagen mitbekommen würde.

	Und nicht dass du jetzt denkst, du würdest als Einzelgänger bis ans Ende deiner Studenlaufbaun verlassen und allein dahin vegitieren. Das wird schon noch, die Sache mit dem Rudel. Erwarte nur nicht zu viel von den ersten paar Tagen. Als Bachelor und gerade auch als Master-Student solltest du die vielfältigen Angebote der Fachgruppe (Spieleabende, Kneipentouren, Grillen, etc.) nutzen um die anderen kennen zu lernen - siehe \url{http://fginfo.cs.tu-bs.de/}
\end{multicols}{2}