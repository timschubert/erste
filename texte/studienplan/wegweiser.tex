\subsection{Quo vadis? - Wo geht die Reise hin?}
        Das Leben und Lernen an der Uni ist sehr spannend. Es bieten
	sich viele Möglichkeiten, das Studium individuell zu gestalten,
	nach Interessen zu wählen und schließlich den erwünschten
	Abschluss zu erhalten. Ein Studierender genießt große
	Freiheiten. Aus diesen großen Freiheiten ergibt sich aber auch
	eine große Verantwortung. Wie das zusammenhängt und welche
	Gefahren daraus resultieren, soll hier einmal kurz aufgearbeitet
	werden. \\ \\
        Grundsätzlich gilt an der Uni zunächst, dich zwingt niemand irgend etwas zu tun. Vorlesungen können besucht werden, müssen aber nicht
Hausaufgaben sind in einigen Modulen Studienleistung, müssen aber nicht vor der Klausur erbracht werden. In anderen Modulen sind sie optional und können, müssen aber nicht gemacht werden. 
Prüfungen können zum vorgesehenen Zeitpunkt abgelegt werden, müssen aber
nicht. \\\\
        
        Dieses Konzept spiegelt eine gewisse Scheinfreiwilligkeit
	wieder, die es aber gar nicht ist. Der spannende Unterschied ist
	der folgende: ,,Dich zwingt niemand etwas zu tun.'' heißt noch
	lange nicht ,,Du mußt nichts tun.''! \\\\
Studieren heißt, sich selbstständig, mit wissenschaftlichen Methoden, in
die Thematik des Faches einzuarbeiten und einen umfassenden Überblick
sowie, möglicherweise, Schwerpunktspezialisierungen zu erwerben.
Vorlesungen und Übungen dienen dabei zur Grundlagenvermittlung und als
Hilfestellung. Ohne etwas zu tun, wirst du zwar studieren, aber nichts
davon haben. Die zentrale Frage für dich ist also: ,,Wie gehst ich  mit
dieser neuen Situation um?''\\\\
 Schauen wir uns einmal die typischen Lehrveranstaltungen an. 
 In den Vorlesungen werden die wichtigen, theoretischen Inhalte vermittelt. 
 In den Übungen werden Aufgaben und Herangehensweisen zu dem Stoff der Vorlesung vermittelt. 
 Beides ist wichtiges Wissen, dass Fachkompetenz aufbaut und für die
 Prüfung am Ende des Semesters benötigt wird.\\\\
Ziel muss es im Semester also sein, den Stoff zu verstehen, zu lernen und in der Prüfung auf Aufgaben 
anwenden zu können, egal ob du Veranstaltungen besucht werden oder
nicht. Klar, manche Vorlesungen sind gähnend langweilig, manche Inhalte
erscheinen einen viel zu theoretisch und manchen Lehrenden kann einfach nicht zugehört werden. 
Das sind alles Gründe, irgendwann nicht mehr in die Vorlesung zu gehen, aber dann fehlt eben ein wichtiger Teil des Lernens. 
,,Ich kann doch ein oder zwei Bücher lesen und mir das Wissen selber aneignen.''
 Ja, das ist richtig, das kannst du machen. Für einige mag dies tatsächlich der bessere Weg sein, aber im großen und ganzen ist dies viel mühsamer als die Vorlesung zu besuchen.       
        Was heißt das jetzt genau? \\
Das heißt eigentlich nur eines: Lass dich von deinen neu gewonnen Freiheiten nicht daran hindern, erfolgreich zu studieren. Du hast dir deinen Studiengang ausgesucht und hast das Interesse, dich wissenschaftlich ausbilden zu lassen. Die Uni bietet dir diese Chance, also nutze sie!
        Gehe lieber einmal zu oft zur Vorlesung und Übung als das eine Mal zu
	wenig. Gerade in den ersten Semestern ist dies wärmstens von uns
	empfohlen, da du deinen eigenen Lernstil noch finden
	musst.\\\\
        Trotzdem: Genieße deine neuen Freiheit, aber nutze sie weise, bevor sie
        zum Fluch wird. :)
