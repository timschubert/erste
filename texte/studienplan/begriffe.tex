\subsection{Hinweise zu Lehrveranstaltungen}

Du wirst sehr bald feststellen, dass es verschiedene Lerntypen gibt. Manche
deiner Kommilitonen werden kaum eine Vorlesung besuchen, sondern stattdessen die großen
und kleinen Übungen verschlingen. Wieder andere lassen sich sowieso kaum im
Hörsaal blicken, sondern können am besten zu Hause oder in der Uni-Bibliothek
autodidaktisch lernen.

Wenn trotz Vorlesungen, großer und kleiner Übungen noch Fragen
auftreten, hilft dir das Gespräch mit HiWis oder Kommilitonen und der Blick in
entsprechende Literatur.
Wichtig: Kaufe nicht gleich jedes empfohlene Buch neu,
das ist Geldverschwendung. Frage höhere Semester nach wirklich sinnvoller
Literatur, leihe dir die Bücher aus der UB, gebrauchte Bücher gibt es
günstig z.B. in der Newsgroup \url{http://groups.google.de/group/braunschweig.kaufrausch/} (siehe den Artikel
ab Seite \pageref{elekinf}). 

An der Uni wirst du nicht wie in der
Schule oder der betrieblichen Ausbildung umsorgt, du trägst ein wesentlich höheres
Maß an Eigenverantwortung. Zur Orientierung in der ersten Zeit ist ein
Ansprechpartner unentbehrlich. Wenn die Kommilitonen aus
deinem eigenen Semester nicht weiterhelfen können, dann vielleicht dein/e TutorIn oder andere
Studierende im höheren Semester (zum Beispiel Mitbewohner, Fachgruppe).
