\begin{multicols}{2}
\subsection{Grobplanung zuerst}
\label{grob}
	Später, in den Bachelor- bzw. Master-Spezifischen Kapiteln, wird es darum gehen, wie du dir den Stundenplan für das jetzt beginnende Semester zusammenstellst. Wie zu Beginn des Abschnitts \textit{Verantwortung} schon angedeutet, gibt es aber zunächst noch weitreichendere Entscheidungen für dein Studium zu treffen, bevor es an die Feinplanung geht. Keine Sorge, deine \textit{Studiengrobplanung} ist ein abstraktes Konzept, du wirst sie nirgends aufschreiben und einreichen müssen, du kannst also große Teile davon so oft ändern wie du möchtest. Aber Vorsicht: Zum einen studiert es sich besser, wenn man von Anfang an weiß, wo es hin geht, zum anderen gibt es gewisse Enscheidungen, die man später nicht mehr ändern kann, wie z.B. das Nebenfach. Aber dazu später mehr\ldots

\subsubsection{Wie viele Credit Points?}
	Standardmäßig ist vorgesehen, pro Semester 30 Credit Points zu erlagen - so hat man nach 6 Semestern den Bachelor und nach weiteren 4 den Master  in der Tasche. Man ist dann aber auch zeitlich sehr ausgelastet, und für Urlaub, Familie und Nebenjob bleibt nicht unbedingt Zeit. Wenn man außerdem mit Zulassungsauflagen gesegnet ist, sind dies bis zu 15 weitere Credit Points, die man irgendwie auf die ersten beiden Semester aufteilen muss, und so lohnt es sich durchaus frühzeitig darüber nachzudenken, wie viele Semester man wirklich studieren möchte und wie viele Credit Points man pro Semester ableisten möchte und kann. Du wirst hier und da noch Gerüchte hören, dass man mindestens 15 CP pro Semester schaffen muss. Das war früher mal so, wurde aber glücklicherweise nun abgeschafft, also lass dich von solchen Aussagen nicht allzusehr beeinflussen.

	Dann steht ja dem entspannten Studium (fast) nur noch die Finanzierung im Wege. BAFöG-Höchstförderungsdauer, Langzeitstudiengebühren, sowie das Ende von Kindergeld, Kindesunterhalt und Famlienversicherung bei der Krankenkasse sind hier die relevanten Stichwörter, die viele Masterstudenten irgendwann ereilen. Hiwi-Jobs, Studienkredite und Stipendien verschaffen vielleicht Linderung.

	Was auch immer du nun denkst, wie viele CP du im kommenden Semester belegen möchte, plane vielleicht ein paar Reserve-Punkte ein, also zusätzliche Fächer, die du belegst. Du kannst dann immernoch im laufenden Semester mache Vorlesungen \emph{kicken} wenn es doch nicht so spannend ist wie zuerst gedacht bzw. du kommst noch auf die angepeilte Punktezahl, auch wenn du durch ein oder zwei Prüfungen fällst. Durchfallen ist weder eine Schande noch ein großes Problem, da die Prüfungsordnung dir erlaubt, bis zu drei Fächer, bei denen du im 1. Versuch durchgefallen bist, so abzuwählen als hättest du sie nie belegt. Dennoch sollte man es vielleicht mit den Reservefächern nicht übertreiben. Versuche einfach, den folgenden plumpen Witz auf diese Situation zu übertragen, und frag dich, ob das zielführend ist:

	Kunde: Ich hätte gerne 20 Brötchen. 
	Bäckerin: So viele? Davon wird ihnen doch die Hälfte trocken bevor sie die gegessen haben!
	Kunde: Oh, das hab ich nicht bedacht. Dann nehm ich doch lieber 40 Stück.

\subsubsection{Nebenfach und Studienrichtung}
\label{nebenfach}
	Im Bachelor musst du, im Master kannst du ein Nebenfach wählen. Die Nebenfach-Enscheidung (ob und welches) will gut überlegt sein, denn wenn man erstmal \emph{drin} ist (d.h. zwei Prüfungen im Nebenfach bestanden hat) kommt man nicht mehr raus. Dies gilt für beide Studiengänge, man kann das Nebenfach dann auch nicht mehr wechseln. 
 
	Die Studienrichtung ist  optional, aber im Gegensatz zum Nebenfach geht man damit keinerlei Verpflichtung ein. Am Ende des Studiums wird einfach geschaut, ob man 50 (Bachelor) oder 70 (Master) Credit Points aus einem artverwanden Bereich gemacht hat und bekommt dann auf Wunsch ein Sonderprädikat aufs Zeugnis. Aber Vorsicht: manche Studienrichtungen erfordern außerdem noch, das man eine gewisse Untermenge von Seminar, Projektarbeit und Abschlussarbeit, sowie eine Mindestanzahl von Praktika im entsprechenden Bereich absolviert hat. Informiere dich also rechtzeitig! Im schlimmsten Fall kann einem somit aber nur passieren, dass man sich zwar in einer Richtung spezialisiert hat, darüber aber keinen expliziten Nachweis auf dem Zeugnis erhält.

	Beide Entscheidungen (Nebenfach, Studienrichtung) musst du nicht im ersten Semester treffen, sondern kannst dich auch später (aber am besten nicht zu spät) spezialisieren. Um dir dabei zu helfen, sammelt die Fachgruppe Berichte zu den Nebenfächern unter \url{http://fginfo.cs.tu-bs.de/index.php/studium/erfahrungsberichte-zu-den-nebenfachern/}. Leider haben wir da noch viele Lücken, die wir aber hoffen im Laufe der nächsten Semester (gerne auch mit deiner Hilfe!) schließen zu können.

\subsubsection{Welche Fächer gibt es?}
	Die Liste der Fächer ist groß und ständig im Wandel. Offiziell festgelegt sind diese im Modulhandbuch (MHB), und anders als der Name vermuten lässt, präsentiert sich dieses nicht etwa als handliches Buch für die linke Jackeninnentasche, sondern als recht unübersichtliche Webanwendung. \\ Unter \url{https://mhb.tu-bs.de/mhb1112ws/studiengangAbstract.do?id=210&call=studiengangListeNavigation} (Bachelor) bzw. \url{https://mhb.tu-bs.de/mhb1112ws/studiengangAbstract.do?id=222&call=studiengangListeNavigation} (Master) findest du eine Liste sämtlicher Module, die du  einbringen kannst. Und da so ein kryptischer Link weder angenehm zu tippen ist, noch garantiert ist, dass er am Tage nach dem Druck dieses Heftes noch erreichbar ist, kannst du auch \url{https://mhb.tu-bs.de/mhb1112ws/} aufrufen und dann über \emph{Studiengänge ansehen} navigieren, um dann nach \emph{Informatik} Ausschau zu halten.

	All diese Fächer kannst du als Informatikstudent belegen - aber längst nicht jedes davon wird in diesem Semester angeboten. Nach einem Klick auf ein Fach siehst du die Details. Dort steht dann auch alles weitere zum Modul, und manches davon ist verbindlich und stets aktuell und korrekt. Die Information, ob ein Fach im Winter oder Sommer angeboten wird, gehört definitiv nicht dazu, was uns zur nächsten Informationsquelle bringt\ldots

\subsubsection{Der generelle Stundenplan}
	Unter \url{http://theo.iti.cs.tu-bs.de/STP/stundenplan.php} solltest du den aktuellen Plan  finden. Dort sind - theoretisch - alle Veranstaltungen der Informatikmodule eingetragen, allerdings ohne die Nebenfächer und den Schlüsselqualifikations-Pool (siehe entsprechender Abschnitt weiter oben). Der Stundenplan enthält sowohl Bachelor- als auch Masterfächer. Der Plan ist nicht getrennt, da nämlich die Bachelorstudenten auch ein paar Master-Fächer einbringen dürfen - andersrum gilt das aber nicht. Also musst du für jedes Fach, was du hier findest, erstmal verifizieren ob du dessen Punkte überhaupt  einbringen kannst. Wie du dir vielleicht schon denken kannst, wird dein persönlicher Stundenplan eine Untermenge dieses Mammut-Plans, erweitert um ein paar Veranstaltungen die selbst hier nicht stehen.

	Wenn etwas darauf hindeutet, dass eine bestimmte Vorlesung im Semester angeboten wird, diese aber im Stundenplan nicht auftaucht, dann hilft eine Suche auf den Institusseiten, und wenn selbst das nicht hilft, eventuell eine Mail an den verantwortlichen Professor. Das gleiche gilt, wenn irgendwas komisch wirkt, z.B. wenn im Stundenplan zu einem Fach 5 Übungstermine und kein Vorlesungstermin stehen, was nun durchaus nicht das erste Mal wäre.

\subsubsection{Auslandsaufenthalt}
	Über Auslandssemester solltest du dich ebenfalls so früh wie möglich mit dem \emph{International Office} (\url{http://tu-braunschweig.de/international}) unterhalten.

	Die Infoveranstaltungen \emph{Wege ins Ausland} und \emph{Studieren in Europa} finden an jedem ersten Mittwoch eines Monats ab 16 Uhr im International Office (BW 74) statt. Der nächste Termin ist damit also bereits der 08. Oktober.

\subsubsection{Mentoren und Beratungsgespräche}
	Laut Studienordnung bekommst du auch einen Mentor zugewiesen - das ist Professor aus der Informatik, der dich bei Entscheidungen zum Studium im persönlichen Gespräch beraten soll. Gerade wenn du weißt, dass du dich spezialisieren möchtest, oder wenn du zumindest mit dem Gedanken spielst, solltest du einen Mentor haben, der aus der jeweiligen Fachrichtung kommt. Wird dir zu Beginn ein völlig fachfremder Mentor zugewiesen, dann kannst du recht formlos darum bitten, diesen zu wechseln. Gespräche mit dem Mentor sind weder verpflichtend noch planmäßig vorgesehen, es liegt also an dir, dich um einen Termin zu kümmen, wenn du beraten werden möchtest. Manche Mentoren veranstalten auch im einige Wochen nach Semesterstart ein großes Treffen mit all ihren Schützlingen.Leider sind diese Treffen, wenn es sie denn gibt, oft schlecht besucht. Nutze diese Chance! Die Aussicht auf diese Versammlung sollte dich außerdem nicht davon abhalten, schon vorher das Gespräch zu suchen.

	Außer dem dir zugewiesenen Mentor gibt es noch weitere Ansprechpartner für verschiedenste Anlässe. Die wichtigsten haben wir für dich unter \url{http://fginfo.cs.tu-bs.de/index.php/kontakt/ansprechpartner/} zusammengefasst.
\end{multicols}