% !TEX root = ../../1-te.tex

\subsection{Die Prüfungsordnung}
\label{po}
	An einer Universität gibt es tausende Regeln und Ordnungen. Die wichtigste ist die Prüfungsordnung: Sie enthält Antworten auf 95\% aller Fragen, die im Studium auftreten - nicht nur, wenn es um die eigentlichen Prüfungen geht. Die genaue Bezeichnung lautet \emph{Besonderer Teil der Prüfungsordnung für den Bachelor-/Masterstudiengang Informatik der Technischen Universität Braunschweig}. Und da sie weder besonders lang, noch kompliziert geschrieben ist, sollten sie alle Studierenden mindestens einmal lesen.

	Dann gibt es noch die APO, die Allgemeine Prüfungsordnung. Sie gilt uniweit für alle Studiengänge, doch die beiden BPOs überschreiben die meisten APO-Regelungen.

	Wenn du es noch nicht getan hast, lade dir deine aktuelle Prüfungsordnung am besten von \verUrl{4}{http://www.tu-braunschweig.de/fk1/service/informatik/dokumente} herunter.


