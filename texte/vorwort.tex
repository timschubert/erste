\end{multicols}
\section{Vorwort}
\begin{multicols}{2}

%\renewcommand{\baselinestretch}{0.75}\normalsize
\subsection*{Willkommen in der Informatik in Braunschweig!}

In den ersten Wochen werdet ihr mit Informatitionen überschüttet, 
unzählige Hochschulvertreter buhlen um eure Aufmerksamkeit. Die 
Erstsemesterzeitung, die ihr hier in Händen haltet, wird von der 
Fachgruppe Informatik herausgegeben, und das sind Studenten wie 
ihr, nur dass wir die aufregenden ersten Monate bzw. Jahre schon 
hinter uns haben und euch aus Erfahrung sagen können, was 
wirklich wichtig ist. Mehr darüber, wer wir sind und was wir machen
findest du ab Seite \pageref{fachgruppe}.

\subsection*{Aufbau dieses Heftes}

In der ersten Hälfte dieses Heftes findet ihr wichtige 
Erklärungen, eine Todo-Liste, Tipps, viele Ansprachpartner, 
etc. die ihr unbedingt so früh wie möglich lesen solltet, denn 
gerade zu Beginn gibt es irre viel Organisatorisches zu beachten, 
und manches davon sollte man schon ein paar Wochen vor 
Semesterbeginn angehen. Aber danach (noch) nicht ins Altpapier damit!

In der zweiten Hälfte stehen dann Infos, die nicht so dringend
sind - aber dennoch genauso wichtig! Zum Beispiel wie die 
verschiedenen studentischen Gruppen und Hochschulgremien auf 
die Regelungen einwirken, nach denen ihr schließlich studieren 
müsst, und wie ihr aktiv darauf Einfluss nehmen könnt. 

%Das ist sicher 
%nichts, was euch in den ersten Wochen und Monaten beschäftigt, aber 
%wenn ihr euch einmal fragen solltet "`Warum ist diese Regelung denn 
%so bl"od und kann ich wirklich nichts tut, damit es besser wird?"' 
%dann nehmt euch dieses Heftchen zur Hand und schaut nochmal nach.

%\subsection*{Sommer-Anfänger }
%Während man den Master seit Einführung zum Sommer starten konnte, ist
%es dieses Jahr erstmals möglich zum Sommer mit den Bachelor zu
%starten. Dadurch ergeben sich einige Besonderheiten, beachtet dazu
%bitte das eigene Kapitel ,,Bachelorbeginn im Sommer''. 

\subsection*{Bachelor, Master}
Früher richtete sich 1-te an neue Diplom-Studenten, seit einigen 
Jahren nun stattdessen an Bachelor-Anfänger und Master-Anfänger. Damit einher geht, dass nicht 
alle Infos in diesem Heft für jeden von euch gleichermaßen relevant 
sind. Daher teilt sich unser Text in der Mitte des Heftes kurz auf,
danach kommen aber wieder Informationen, die euch alle angehen.\\\\
Allen Bachelor-Erstis ein besonderes ,,Herzlich wilkommen!''. Ihr seid
die Ersten, die den Bachelor zum Sommer starten. Für euch ist einiges
anders, als für uns. Wir haben uns daher Mühe gegeben, möglichst viele
Informationen zusammenzutragen. Trotzdem können sich Fehler
einschleichen oder noch Dinge ändern. Gerade dann interessieren dich sicher 
die\ldots

\subsection*{Nachträge und Korrekturen online}

Auch wenn wir es jedes Jahr versuchen, alles wichtige hier 
unterzubringen, fehlt naturgemäß immer irgendwas. Außerdem finden
mal wieder einige Änderungen der Ordnungen statt nachdem wir dieses 
Heft drucken und bevor ihr es in Händen haltet - perfektes Timing 
halt - so dass ein Teil dessen, was wir gerade schreiben, schon 
wieder out-of-date sein wird. Falls uns also noch etwas wichtiges 
einfällt, findet ihr es unter \url{http://fginfo.cs.tu-bs.de}. Dort % TODO evtl. hier einen Deeplink, etwa http://fginfo.cs.tu-bs.de/erste oder sowas
gibt es auch die 1-ste als HTML-Fassung und als PDF-Download, was
beides deshalb praktisch ist, da du dann die vielen URLs aus diesem
Text nicht abtippen musst. Wenn du Vorschläge zum Verbessern dieser
Zeitung hast, kannst (und sollst) du diese auch auch gerne dort
loswerden.

\vspace*{0.5cm}

Viel Spa"s und Erfolg in den ersten Tagen w"unscht euch die\\
\hspace*{2cm}Fachgruppe Informatik

\vfill

\small{\textit{(Ausgabe vom \today)}}

%\renewcommand{\baselinestretch}{1}\normalsize