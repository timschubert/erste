% !TEX root = ../../1-te.tex

\subsection{Wichtige Termine am Anfang des Studiums}

\tocheck{5}{Termine für aktuelle O-Woche einfügen}

Wir möchten den Start an der TU Braunschweig so gut wie möglich begleiten. Daher wird es zu Beginn des Semesters wieder Begrüßungs- und Einführungsveranstaltungen geben. Bis zum Semesterstart können sich einzelne Termine noch ändern. Den ganz aktuellen Stand gibt es online unter \verUrl{5}{https://fginfo.cs.tu-bs.de/ersti}.

\renewcommand{\labelitemi}{$\bullet$}
\renewcommand{\labelitemii}{$\bullet$}
\renewcommand{\labelitemiii}{$\bullet$}
\renewcommand{\labelitemiv}{$\bullet$}

\begin{itemize}
	\item Vorkurs: 19. - 30. März
	\item Dienstag, 3. April
	\begin{itemize}
		\item 09:45 Uhr: 1. Vorlesung Algebra für Informatiker (PK 2.2)
		\item 11:30 Uhr: 1. Vorlesung Analysis (PK 2.2)
		\item 13:15 Uhr: Begrüßung durch die Professoren der Informatik \& Informationen zum Studium (IZ 160)
		\item 11:00 - 17:00 Uhr: Infobörse für Erstsemester (Wiese vor der Mensa)
	\end{itemize}
	\item Mittwoch, 4. April
	\begin{itemize}
		\item 09:45 Uhr: Einführung in die Logik (PK 2.2)
		\item 11:30 Uhr: Gemeinsames Brunch (Plaza, IZ 1.OG). \textbf{Bitte eigenes Besteck und Teller mitbringen}
		\item 12:30 Uhr: Anmeldeschluss Erstiwochenende
		\item 12:30 Uhr: Campustour
		\item 13:15 Uhr: Vorstellung der Fachgruppe Informatik (IZ 160)
		\item Anschließend Stundenplanbau
	\end{itemize}
	\newpage
	\item Donnerstag, 5. April
	\begin{itemize}
		\item 09:45 Uhr: 1. Vorlesung Computernetze (SN 19.1)
		\item 11:30 Uhr: 1. Übung Computernetze (SN 19.1)
		\item 19:00 Uhr: Kneipentour (Haupteingang der Mensa 1)
	\end{itemize}
	\item Freitag, 6.April
	\begin{itemize}
		\item 14:30 Uhr: Treffen Abfahrt Erstifahrt (Foyer Mühlenpfordstraße 23)
	\end{itemize}
	\item Montag, 9. April
	\begin{itemize}
		\item 16:45 Uhr: Linux-Install-Party (IZ 160)
	\end{itemize}
	\item Dienstag, 10. April
	\begin{itemize}
		\item 18:30 Uhr: Analoger Spieleabend (Flur vor IZ 150)
	\end{itemize}
	\item Ersti-Wochenende
	\begin{itemize}
		\item Wann? 6. – 8. April
		\item Wo? Naturfreundehaus Eichsfelder Hütte (St. Andreasberg)
		\item Was? Lerne deine Mitstudierenden kennen, habe Spaß :)
		\item Finanzierung? Größtenteils aus Studienqualitätsmitteln, dazu 20 Euro Selbstkostenbeitrag
		\item Fristen: Anmeldung und Bezahlung des Selbstkostenbeitrags bis 4. April
		\item Die Anmeldung ist online möglich\footnote{\verUrl{5}{https://pretix.coldney.de/fsinfo/erstifahrtss2018/}}
	\end{itemize}
\end{itemize}