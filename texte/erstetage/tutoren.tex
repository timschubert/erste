\subsection{Tutorien}
Da der Einstieg ins Studium alleine relativ schwierig ist und sich viele Fragen in einer Gruppe am einfachsten beantworten lassen, gibt es Tutorengruppen, denen du am Montag um ca. 15 Uhr zugeteilt wirst. 
In diesen findet dann ein Rundgang über den Campus und die wichtigsten Einrichtungen in der Nähe statt. 
Dort hast du außerdem die Möglichkeit in überschaubarer Runde andere Informatikstudenten und eben eure Tutoren kennen zu lernen, sowie weitere Informationen zu deinen Veranstaltungen und Dozenten zu erhalten.
Scheu dich nicht, einen der Tutoren zu kontaktieren. Da weitere Treffen geplant sind kannst du dich auch nachträglich noch in eine Tutorengruppe einteilen lassen. Dazu schreibt bitte eine Mail an die Fachgruppe \url{fginfo@tu-bs.de} oder direkt an einen der Tutoren.


\ifpdf
Damit du schonmal das ein oder andere Gesicht kennst, sind hier die Fotos der Tutoren abgebildet.
\end{multicols}



% TODO Bilder aktualisieren, angeben, ob der jenige Bachelors oder Masters betreut (ist ja nicht immer identisch damit, was man selbst gerade ist)
% NICHT als \paragraph, da das leider das Layout verhaut. Ist natürlich
% unschön ):
%Leider ist das ganze Layout etwas screwed
%Also ein alter Trick aus Web 1.0 Tagen\ldot.
%Layout Tabellen! Nicht schön, aber wenns sonst nicht geht\ldot.
  \begin{tabular}{c}
\begin{tabular}{lll}
  { \textbf{Bachelor Tutoren}} \ \\ 
% \picture[0.3\linewidth]
% {bilder/tutoren/kris.jpg}
% {Christoph\\ 3. Semester Master\\ \randomize{admin@keeg.de}}
% \ {
{
\picture[0.3\linewidth]
{bilder/tutoren/dominik_pass}
{Dominik\\%3. Semester Master\\ 
\randomize{d.schuermann@tu-bs.de}}
}%\hfill
& \ 
%{\picture[2.3\linewidth]
%{bilder/tutoren/franziska.jpg}
%{Franziska\\6. Semester Bachelor\\ \randomize{f.werk@tu-bs.de}}
%}%\\ \ \\
{\picture[0.3\linewidth]
{bilder/tutoren/jan_germann.jpg}
{Jan\\%2. Semester Bachelor\\ 
\randomize{j.germann@tu-bs.de}}
} 
&\
{
\picture[0.3\linewidth]
{bilder/tutoren/hf}
{Hella\\% 2. Semester Master\\ 
\randomize{h-f.hoffmann@tu-bs.de}}}
\\ \  \\
%\hfill
{\picture[0.3\linewidth]
{bilder/tutoren/johannes.jpg}
{Johannes\\% 7. Semester Bachelor\\ 
\randomize{J.Starosta@tu-bs.de}}
}& \ 
%\hfill
% {\picture[0.3\linewidth]
% {bilder/tutoren/judith.jpg}
% {Judith\\ 5. Semester Bachelor\\ \randomize{judith.hilpert@web.de}}}\\
% \ \\
% \picture[0.3\linewidth]
% {bilder/tutoren/marekd.jpg}
% {Marek\\ 1. Semester Master\\ \randomize{m.drogon@tu-bs.de}}
% &  %\hfill
 {
 \picture[0.3\linewidth]
 {bilder/tutoren/sebastian.jpeg}
 {Sebastian\\ %9. Semester Bachelor\\ 
\randomize{se.busse@tu-bs.de}}} 
%&{
%\picture[0.3\linewidth]%{}
%{bilder/tutoren/serj}
%{Serj\\ 1. Semester Master\\\randomize{s.dechand@tu-bs.de }}}
%\par \ \par  
%\end{tabular}
%
%\begin{center}
%  \begin{tabular}{ccc}
 %    { \picture[0.3\linewidth]%{}
 % {bilder/tutoren/christina}
 % {Christina\\ 5. Semester Bachelor\\\randomize{c.eberth@tu-bs.de}}}
%{
%\picture[0.3\linewidth]%{}
%{bilder/tutoren/viktor}
%{Viktor Richert\\ 4. Semester Bachelor\\%}%\\ 5. Semester Bachelor\\
%\randomize{InformatikWiki@gmx.de}}%.eberth@tu-bs.de }
% }&{
%  \picture[0.3\linewidth]%{}
%  {bilder/tutoren/dummy}
%  {Christoph\\ 5. Semester Bachelor\\ \randomize{christoph.harburg@web.de}}
% }\\
&\ {\picture[0.3\linewidth]%{}
{bilder/tutoren/joko_n}
{Jonathan\\% 7. Semester Bachelor \\
\randomize{j.koscielny@tu-bs.de}}}
\\ \  \\
{\picture[0.3\linewidth]%{}
{bilder/tutoren/viktor}
{Viktor\\% 5. Semester Bachelor\\%%\\ 5. Semester Bachelor\\
\randomize{InformatikWiki@gmx.de}}}
&\ {\picture[0.3\linewidth]%{}
{bilder/tutoren/rcfinster_sm}
{Rebecca \\ % Bachelor\\%}%\\ 5. Semester Bachelor\\
\randomize{r.finster@tu-bs.de}}}
& \ 
{\picture[0.3\linewidth]
{bilder/tutoren/keno_sw.jpg}
{Keno\\%2. Semester Bachelor\\ 
\randomize{k.garlichs@tu-bs.de}}
}
\end{tabular}%\tabularnewline \\
\end{tabular}\tabularnewline \\
%\begin{left}
\newpage
%\begin{center}
\begin{tabular}{ccc}
{ \textbf{Master Tutoren}}\\
% NICHT als \paragraph, da das leider das Layout verhaut. Ist natürlich
% unschön ):
\picture[0.3\linewidth]
{bilder/tutoren/martinw_sw.jpg}
{Martin\\% 4. Semester Master\\ 
\randomize{m.wegner@tu-bs.de}}
&
%{ 
%\picture[0.3\linewidth]
%{bilder/tutoren/jan.jpg}
%{Till\\  Master\\ \randomize{t.lorentzen@tu-bs.de}
%}&
%\\
%\hfill %\par \ \par
%{
%\picture[0.3\linewidth]
%{bilder/tutoren/henning.jpg}
%{Henning\\ 5. Semester Master\\ \randomize{h.guenther@tu-bs.de}}
%}&
{ 
\picture[0.3\linewidth]
{bilder/tutoren/sophia.jpg}
{Sophia\\% 3. Semester Master\\
\randomize{s.scholtka@tu-braunschweig.de}}
}
&
{
\picture[0.3\linewidth]%{}
{bilder/tutoren/serj}
{Serj\\% 2. Semester Master\\
\randomize{s.dechand@tu-bs.de }}}
\\ \ \\
{\picture[0.3\linewidth]
{bilder/tutoren/lena_sm.jpg}
{Lena\\ %5. Semester Master\\ 
\randomize{dielenamaria@gmail.com}}
}&
%\hfill
%{\picture[0.3\linewidth]
%{bilder/tutoren/hashier.jpg}%Christopher Lössl < c.loessl@tu-bs.de>
%{Chris \\3. Semester Master\\ \randomize{c.loessl@tu-bs.de}}
%}%&
{\picture[0.3\linewidth]
{bilder/tutoren/till_sw}
{Till\\ %\\  4. Semester Master\\
\randomize{t.lorentzen@tu-bs.de}}
}
%\picture[0.3\linewidth]
%{bilder/tutoren/stephan.jpg}
%{Stephan\\ 5. Semester Master\\ stephan.friedrichs@tu-bs.de}
  \end{tabular}
%  \end{tabular}
  
%\end{center}
\else
In der Druckfassung wären hier die Tutoren mit Emailadresse und Foto aufgelistet. Um die Privatspäre der Tutoren zu schützen, stehen diese jedoch nicht direkt hier in der Onlinefassung.
\fi

\begin{multicols}{2}
%%% Local Variables: 
%%% mode: latex
%%% TeX-master: "../../1-te"
%%% End: 
