% !TEX root = ../../1-te.tex

\subsection{Checkliste}
\label{checkliste}
	Hier wird zusammengefasst, was du in den ersten Tagen des Studiums unbedingt erledigen solltest. Wenn du die ToDos auf der Checkliste nach Erledigung abhakst, verlierst du nicht den Überblick und vergisst nichts.
	
%\vspace*{0.5cm}
% !TEX root = ../../1-te.tex

\begin{tabular}{|p{3mm}|l|l|c|c|}
\hline \checkmark 
       & \textbf{Todo}             & \textbf{Zu erledigen bis}                                  & \textbf{Seite}               & \textbf{Muss?} \\ 
\hline & BAföG beantragen          & Spätestens Ende \iftoggle{winter}{Oktober}{April}          & \pageref{todobafoeg}         & optional \\ 
\hline & Wohnsitz Ummelden         & 1 Woche nach Umzug                                         & \pageref{todoummelden}       & ja \\ 
\hline & Mailinglisten             & So früh wie möglich                                        & \pageref{mailinglisten}      & ja \\ 
\hline & Studiengrobplanung        & Vor dem Stundenplanbauen                                   & \pageref{grob}               & ja \\ 
\hline & Auflagen klären           & So früh wie möglich, final: Ende 2. Semester               & \pageref{auflagen}           & nur Master \\ 
\hline & Persönlicher Stundenplan  & Siehe Terminzettel der Fachgruppe                          & \pageref{masterstundenplan}  & ja \\ 
\hline & Prüfungsbogen             & Apätestens \iftoggle{winter}{Dezember}{Mai}                & \pageref{todoanmeldung}      & ja \\ 
\hline & Prüfungsanmeldung         & Anmeldewoche (30.05 - 03.06) oder Online (09.05 - 20.06)   & \pageref{todoanmeldung}      & ja \\ 
\hline & Blog abonnieren           & So früh wie möglich                                        & \pageref{fachgruppe}         & ja \\ 
\hline & Prüfungsordnung lesen     & Zu den ersten Klausuren                                    & \pageref{po}                 & ja \\ 
\hline & TUcard validieren         & Zu Beginn und zu jedem neuen Semester                      & \pageref{tucard}             & ja \\
\hline & Bibliotheksausweis        & Vor der ersten Buchausleihe                                & \pageref{todobib}            & optional \\
\hline & Kopierkarte               & Wenn man was kopieren muss                                 & \pageref{todobib}            & optional \\ 
\hline
\end{tabular} 
\tocheck{2}{Exakte Daten Anmeldewoche einfügen, s.\url{https://www.tu-braunschweig.de/fk1/service/informatik/pa/}}

\begin{multicols}{2}
\raggedcolumns

\subsubsection{BAföG}
	\label{todobafoeg}
	Wer Studierendenförderung nach dem Bundesausbildungsförderungsgesetz (BAföG) beantragen möchte, sollte sich am besten gründlich informieren: \verHref{6}{https://www.xn--bafg-7qa.de}{https://www.bafög.de}
 
	Förderungsanträge gibt es zum Download oder in Papierform im EG des Amtes für Ausbildungsförderung in der Wilhelmstraße 1. Wenn du BAföG beantragen möchtest, stelle den Antrag so früh wie möglich, denn es wird nicht rückwirkend gezahlt.

	Zum Anfang des Semester ist mit längeren Wartezeiten zu rechnen, im Notfall kannst du beim AStA-Sozialreferat ein kurzfristiges, zinsloses Darlehen beantragen, um den ersten Monat zu überbrücken. Das Darlehen ist auf 450 Euro begrenzt und muss spätestens nach drei Monaten zurückgezahlt werden. Mehr Informationen findest du auf der Seite des Sozialreferats: \verUrl{6}{https://www.asta.tu-braunschweig.de/referate/sozialreferat/}


\subsubsection{Ummelden}
	\label{todoummelden}

	Wer neu nach Braunschweig gezogen ist, muss sich innerhalb einer Woche beim Einwohnermeldeamt anmelden. Wenn du die Frist verpasst, drohen theoretisch Strafen, aber praktisch sieht es da nicht so streng aus. Wenn man Braunschweig als Erstwohnsitz wählt, bekommt man (ein Jahr später) eine einmalige Zuzugsprämie von 100 Euro (Immatrikulationsbescheinigung nicht vergessen). Alternativ kann man Braunschweig auch als Zweitwohnsitz wählen.

\subsubsection{Prüfungsanmeldung}
	\label{todoanmeldung}

	Du musst dich für alle Prüfungen, an denen du teilnehmen willst, vorher beim Prüfungsamt anmelden.
Die Fristen sind relativ früh im Semester und werden auf den Seiten des Prüfungsamtes (\verUrl{6}{https://www.tu-braunschweig.de/fk1/service/informatik/pa}) veröffentlicht und über die Mailingliste kommuniziert.
Prüfungen können im Prüfungsanmeldezeitraum schriftlich im Prüfungsamt oder online über das QIS-Portal (\verUrl{6}{https://vorlesungen.tu-braunschweig.de}) angemeldet werden.

	\todo[inline]{Es gibt keine Meldebögen mehr, überarbeiten!}
	Vor deiner ersten Prüfungsanmeldung musst du außerdem einen Meldebogen (mit Foto) ausfüllen.
Es empfiehlt sich, das bereits vor der Anmeldewoche zu machen, weil die Schlangen dann nicht so lang sind. Ohne den Meldebogen ist in diesem Semester keine Prüfungsanmeldung möglich!

	Für die Online-Anmeldung benötigst du eine TAN-Liste, die du dir vorher im Prüfungsamt organisieren musst.

	Unter folgendem Link findest du außerdem alle Prüfungstermine für die Informatik:
	\verUrl{6}{https://www.tu-braunschweig.de/fk1/service/informatik/pa}

\subsubsection{TUcard}
	\label{tucard}
	
	Alle Studierenden der TU erhaten den elektronischen Studierendenausweis TUcard, der auch als Bibliotheksausweis, Mensakarte und Semesterticket genutzt werden kann.

	Damit die Karte gültig ist, muss sie zu Beginn und zu jedem neuen Semester validiert werden. Das bedeutet, dass der Thermostreifen auf der Karte in einem Validierungsdrucker mit den aktuellen Daten beschrieben werden muss.

	Das Börsenguthaben der Karte, beispielsweise zum Bezahlen in der Mensa, kann an Börsenaufwertern aufgeladen werden.

	Zum Drucken kann Guthaben der Karte auf ein Druckkonto umgebucht werden. Dies geschieht an den Druckkontenumbuchern.

	Weitere Informationen zur TUcard findest du unter: \verUrl{6}{https://www.tu-braunschweig.de/studium/imstudium/tucard}

\subsubsection{Uni-Bibliothek}
	\label{todobib}

	Um Bücher in der Universitätsbibliothek (UB) ausleihen zu können, muss deine TUcard für die Nutzung als Bibliotheksausweis freigeschaltet werden. Dafür stellst du an einem der Terminals direkt in der Bibliothek oder online einen Antrag, die Freischaltung erfolgt dann am Schalter in der Bibliothek. Je nachdem, ob du zu Beginn schon Bücher brauchst, kannst du die Funktion auch später aktivieren.

	In der Bibliothek stehen außerdem Kopierer bereit, die du nutzen kannst. Einen davon kannst du mit Kleingeld befüllen, kompfortabler geht es aber mit einer Kopierkarte. Die bekommst du für ein paar Euro in der Bibliothek. Zu Semesterbeginn gibt es oft noch Einführungskurse in die Bibliotheksbenutzung.
\end{multicols}
