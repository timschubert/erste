
\subsection{Checkliste}
\label{checkliste}
\begin{multicols}{2}
  %}ction{Checkliste}
	Hier wird zusammengefasst, was du in den ersten Tagen des Studiums
	unbedingt erledigen solltet. Du kannst die Punkte in der folgenden
	Tabelle abhaken, um den Überblick zu behalten.
	Im Folgenden findest du dann erweiterte Infos zu manchen der Punkte. 
	Andere Hinweise sind durch die restliche Zeitung verteilt, dafür 
	gibt es eine Spalte mit der jeweiligen Seite.
\end{multicols}
% !TEX root = ../../1-te.tex

\begin{tabular}{|p{3mm}|l|l|c|c|}
\hline \checkmark 
       & \textbf{Todo}             & \textbf{Zu erledigen bis}                                  & \textbf{Seite}               & \textbf{Muss?} \\ 
\hline & BAföG beantragen          & Spätestens Ende \iftoggle{winter}{Oktober}{April}          & \pageref{todobafoeg}         & optional \\ 
\hline & Wohnsitz Ummelden         & 1 Woche nach Umzug                                         & \pageref{todoummelden}       & ja \\ 
\hline & Mailinglisten             & So früh wie möglich                                        & \pageref{mailinglisten}      & ja \\ 
\hline & Studiengrobplanung        & Vor dem Stundenplanbauen                                   & \pageref{grob}               & ja \\ 
\hline & Auflagen klären           & So früh wie möglich, final: Ende 2. Semester               & \pageref{auflagen}           & nur Master \\ 
\hline & Persönlicher Stundenplan  & Siehe Terminzettel der Fachgruppe                          & \pageref{masterstundenplan}  & ja \\ 
\hline & Prüfungsbogen             & Apätestens \iftoggle{winter}{Dezember}{Mai}                & \pageref{todoanmeldung}      & ja \\ 
\hline & Prüfungsanmeldung         & Anmeldewoche (30.05 - 03.06) oder Online (09.05 - 20.06)   & \pageref{todoanmeldung}      & ja \\ 
\hline & Blog abonnieren           & So früh wie möglich                                        & \pageref{fachgruppe}         & ja \\ 
\hline & Prüfungsordnung lesen     & Zu den ersten Klausuren                                    & \pageref{po}                 & ja \\ 
\hline & TUcard validieren         & Zu Beginn und zu jedem neuen Semester                      & \pageref{tucard}             & ja \\
\hline & Bibliotheksausweis        & Vor der ersten Buchausleihe                                & \pageref{todobib}            & optional \\
\hline & Kopierkarte               & Wenn man was kopieren muss                                 & \pageref{todobib}            & optional \\ 
\hline
\end{tabular} 
\tocheck{2}{Exakte Daten Anmeldewoche einfügen, s.\url{https://www.tu-braunschweig.de/fk1/service/informatik/pa/}}

\begin{multicols}{2}
\subsubsection{BAföG}
	\label{todobafoeg}

	Wer BAföG beantragen möchte, sollte sich am besten gründlich informieren. Sehr zu empfehlen ist da: \\
	\url{http://www.bafoeg.bmbf.de/}
 
	Förderungsanträge gibt es zum Download oder in Papierform im
	EG des BAföG-Amtes, Nordstraße 11. Am besten so früh
	wie möglich beantragen, denn BAföG wird nicht rückwirkend
	bezahlt.


\subsubsection{Ummelden}
	\label{todoummelden}

	Wer neu nach Braunschweig gezogen ist, muss sich innerhalb einer Woche beim Einwohnermeldeamt anmelden. Wenn ihr die Frist verpasst, drohen theoretisch Strafen, aber praktisch sieht es da nicht so streng aus. Wenn man Braunschweig als Erstwohnsitz wählt, bekommt man (ein Jahr später) eine einmalige Zuzugsprämie von 200 Euro (Immatrikulationsbescheinigung nicht vergessen). Wer dennoch seinen Erstwohnsitz in der Heimat behalten möchte, sollte glaubhaft darlegen können, dass er mehr als die Hälfte des Jahres nicht in Braunschweig lebt bzw. seinen Lebensschwerpunkt in der Heimatstadt hat.

\subsubsection{Prüfungsanmeldung}
	\label{todoanmeldung}

	Ihr müsst euch für alle Prüfungen, an denen ihr teilnehmen wollt, vorher beim Prüfungsamt anmelden. \emph{Das ist nur eine Woche lang möglich}, im Wintersemester meistens Mitte Dezember, \emph{informiert euch also rechtzeitig, wann genau das ist}!

	Vor eurer ersten Prüfungsanmeldung müsst ihr außerdem ein Datenblatt ausfüllen. Es empfiehlt sich, das bereits vor der Anmeldewoche zu machen, weil die Schlangen dann nicht so lang sind.

	Darüber hinaus gibt es die Möglichkeit, sich online für Prüfungen anzumelden. Dazu braucht ihr allerdings eine TAN-Liste, die ihr euch vorher im Prüfungsamt organisieren müsst.

%\subsubsection{Mensa-Card}
%	\label{todomensa}
%
%Du  brauchst unbedingt eine Mensa-Card (eine Chipkarte, mit der man in der Mensa bargeldlos bezahlen kann), sonst müsst ihr den Gästepreis zahlen. Bei der Immatrikulation bekommt ihr einen Gutschein, den ihr gegen die Mensacard eintauschen könnt - falls nicht, kann man sie auch einfach für 5 Euro erwerben. Ihr solltet auch Studierendenausweis und Lichtbildausweis nicht vergessen, auch wenn das nicht immer gewissenhaft kontrolliert wird. Sobald ihr die Karte habt, schreibt euch die darauf stehende Nummer auf, so könnt ihr eurer Restgeld wiederbekommen, falls ihr die Karte einmal verliert - und dass passiert einem leider recht oft.

\subsubsection{Uni-Bibliothek}
	\label{todobib}

	Um Bücher in der Uni-Bibliothek ausleihen zu können, braucht ihr einen Ausweis. Diesen könnt ihr an einem der Terminals in der Bibliothek beantragen und danach gegen eine Gebühr von 5 Euro am Schalter abholen. Je nachdem, ob ihr zu Beginn schon Bücher braucht, könnt ihr die Karte auch einfach ein bisschen später besorgen.

	In der Bibliothek stehen außerdem Kopierer bereit, die ihr nutzen könnt. Einen davon könnt ihr mit Kleingeld befüllen, kompfortabler geht es aber mit einer Kopierkarte. Und auch diese bekommt ihr für ein paar Euro direkt in der Bibliothek.

	Zu Semesterbeginn gibt es oft noch Einführungskurse in die Bibliotheksbenutzung. Ob ihr eure Bibliothekskarte vor oder nach diesem Kurs besorgt, ist egal.
\end{multicols}
