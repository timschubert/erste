\begin{multicols}{2}
\subsection{Ich bin unpolitisch!}
	Immer wieder hört man diese Aussage in Vorlesungen, in der Mensa und im Gespräch mit Studierenden beim Fachgruppenrat. Für die meisten Studierenden bedeutet diese Aussage, dass man \emph{kein Interesse an Politik} hat oder zumindest keine Meinung zu aktuellen Vorgängen.

\subsubsection*{Gemäßigtes Braunschweig}

Das Braunschweiger Umfeld macht es einem relativ leicht, sich politisch passiv 
zu verhalten. Hier herrscht nur ein recht kleines Spannungsfeld zwischen den 
traditionell eher rechten studentischen Verbindungen und den traditionell eher 
linken Fachschafts- und Fachgruppenräten sowie dem AStA. Diese 
gegenüberstehenden Parteien werdet ihr in den allermeisten deutschen 
Universitätsstädten wiederfinden. Während es aber anderenorts so richtig 
kracht (Burschenschaftshäuser werden mit Farbbeuteln beworfen und mit Parolen 
beschmiert, jeder öffentliche Auftritt von Burschenschaften führt zu 
Demonstrationen), ist Braunschweig ein gemütliches Pflaster. AStA und die 
Fachschaften finden nur wenige Unterstützer und auch die Burschenschaften 
dominieren in Braunschweig nicht unbedingt das Stadtbild.

\subsubsection*{\emph{Schnell durchziehen!}}

Einen erheblichen Beitrag zur \emph{Ist-mir-doch-egal}-Haltung leistet meiner 
Ansicht nach die heute übliche, ständig über Medien, Politiker oder auch die 
eigenen Eltern verbreitete Doktrin, dass man sein Studium \emph{schnell 
durchziehen}, zielstrebig, leistungs- und ich-orientiert seinen Abschluss 
ansteuern soll. Solche Leute will die Wirtschaft, dafür gibt es Preise und 
Stipendien. Langzeitstudenten werden belächelt und als Sozialfall angesehen. 
Unbequeme Themen wie ethische und religiöse Fragen oder Umweltproblematik 
bleiben bei dieser Sichtweise als erstes auf der Strecke (z.B. gibt es in der 
Informatik in Braunschweig --~anders als zum Beispiel an der Uni Hamburg~-- 
keine Pflichtveranstaltung, die sich mit den gesellschaftlichen Einflüssen der 
Informatik auseinander setzt). Man hat das Gefühl, dass unmündige, 
manipulierbare Arbeitnehmer heranzuzüchtet werden sollen - den früher 
propagierten \emph{breiten Horizont} einer Hochschulausbildung konnte ich an 
unserer TU bisher nicht entdecken.

\subsubsection*{Verbindungen zur Politik}

Nun zurück zum \textbf{weit verbreiteten Gerücht, das eigene Studium habe 
doch nichts mit Politik zu tun}: Die Uni als Institution lässt sich nicht von 
der Politik lösen! Wir sind alle direkt betroffen von der Landespolitik (vor 
allem natürlich Bildungspolitik) und Lokalpolitik (z.B. Radwege, 
Attraktivität der Stadt). Außerdem gibt es auch eine Uni-interne Politik, wie 
euch die \emph{Kleine Gremienkunde} in diesem Heft schon ausführlich dargelegt 
hat. Wer sich z.B. in einem Institut umhört, wird dort nirgends 
Gleichgültigkeit gegenüber der Bildungspolitik zu spüren bekommen. Ob 
Professorenstellen neu besetzt werden, ob genügend HiWis für kleine übungen 
bezahlt werden, ob neue Geräte angeschafft werden, ob gar ganze Studiengänge 
geschlossen werden, ob Studierende bei der Gestaltung ihrer Studiengänge 
mitwirken dürfen, ob öffnungszeiten für bestimmte Dienste verlängert werden 
- all dies hängt von der so viel geschimpften \emph{Politik} der einen oder 
anderen Form ab. \textbf{Politik betrifft euch} und euer Studium. Direkt und 
ohne Wenn und Aber.

Nun will ich natürlich von niemandem verlangen, dass er einer Partei beitritt, 
Straßenaktionen startet oder Bücher schreibt. Aber zumindest ein kleines 
Interesse an eurem direkten Umfeld sollte doch selbstverständlich sein, oder? 
Es hat ja einen Grund, dass euer momentanes Studium so ist, wie es ist. Es gibt 
Studierende, die sich engagieren, die selbst etwas beitragen wollen, z.B. eine 
neue BPO (Bachelorprüfungsordnung) mit erarbeiten, für mehr Computer oder 
längere öffnungszeiten streiten etc., um unseren Studiengang und unser 
Hochschulleben attraktiver zu gestalten.

Übrigens war die Hochschulpolitik bis zum Sommersemester 2011 überwiegend 
unabhängig von parteipolitischen Interessen. Nun sind aber erstmals auch 
Wahllisten angetreten, die etablierten Parteien aus der Landes- und Bundespolitik
nahestehen und dies durch ihren Namen eindeutig aufzeigen. Ob und wie sich dies auf die 
Hochschulpolitik auswirkt, wird sich in den kommenden Semestern zeigen.

\subsubsection*{Informieren und Engagieren}

Wie kann man nun einen Einblick in das, was die Studierenden bewegen und was 
die Studierenden bewegt, gewinnen? Als erstes wären dort die hauptamtlichen 
Mitarbeiter des \textbf{AStA} zu nennen. Hinter der umständlichen Abkürzung 
verbergen sich eine Handvoll Studierende, die entgegen weitläufiger Meinung 
weder Steineschmeißer noch Nazis sondern Studierende wie ihr sind. Dann gibt 
es jeden Monat die \textbf{hochschulöffentliche Sitzung des 
Studierendenparlaments}. Dort tauscht man fächerübergreifend Neuigkeiten aus 
und stimmt über entscheidende Dinge ab, z.B. über die Verwendung der 
studentischen Gelder, den studentischen Haushalt. Mindestens einmal im Semester 
gibt es die sogenannte VV, das ist die \textbf{studentische Vollversammlung} - 
wenn sie beschlussfähig ist, dann ist die Vollversammlung das höchste Gremium 
der Studierenden.
Schließlich finden einmal im Semester die \textbf{studentischen Wahlen} statt 
- da könnt ihr direkt oder indirekt (siehe Gremienkunde) bestimmen, welche 
Studierenden euch in den jeweiligen ämtern vertreten sollen. Aus 
unerfindlichen Gründen ist die Wahlbeteiligung bei den studentischen Wahlen 
stets niedrig. Nehmt das als Aufmunterung -- bei geringer Beteiligung zählt 
eure Stimme um so mehr!
