\subsection{Studentische Vereinigungen an der Technischen Universität
Braunschweig}
\label{vereinigungen}
%xp
%	\url{http://fginfo.cs.tu-bs.de}
	An der TU Braunschweig gibt es eine Vielzahl studentischer
	Vereinigungen, bei denen ihr euch neben dem Studium einbringen
	könnt. Dabei ist eigentlich für jeden etwas dabei. Die
	vollständige Liste findet ihr unter 
	\url{http://www.tu-braunschweig.de/abt11/studentischevereinigungen/studverein}. 
	Im Folgenden stellen wir einige unserer Ansicht nach für
	Informatik-Studenten besonders Interessante vor.

	\subsubsection{Gesellschaft für Informatik- Studierendengruppe
	Braunschweig}	
	Die Gesellschaft für Informatik e.V. (GI) ist die größte
	Vereinigung von Informatikerinnen und Informatikern im
	deutschsprachigen Raum. Sie versteht sich als Plattform für
	Informatikfachleute aus Wissenschaft und Wirtschaft, Lehre und
	Öffentlicher Verwaltung und versammelt eine geballte
	Konzentration an Wissen, Innovation und Visionen. \\\\
	 Die SG Braunschweig ist gerade noch mitten im Gründungsprozess
	 und somit die jüngste der derzeit 12 SGen in Deutschland. Die
	 SGen repräsentieren an Informatik interessierte junge Menschen,
	 ohne dabei begrenzt zu sein auf einen einzelnen Studiengang
	 oder eine einzelne Hochschule, und sorgen durch ihre Vernetzung
	 untereinander auch für den überregionalen Austausch. Dazu
	 wollen wir Veranstal 

\subsubsection{Wissenschaftliche Arbeitsgemeinschaft für Studio- und Senderfragen an der TU Braunschweig}
Die ags gibt Studierenden die Gelegenheit während ihres Studiums
praktische Erfahrungen in der Fernseh- und Medientechnik, sowie in
allgemeiner Elektronik zu sammeln. 


Neben den zahlreichen externen Produktionen, z.
B. den Vorlesungen der Kinder-Uni und des MacGyver-Ideenwettbewerbs, die
von der ags begleitet werden, betreibt sie in ihren Räumen 
ein professionelles Fernsehstudio. 
\\\\
 Im Elektroniklabor ,,e.lab'' finden sich sich sechs
 vollausgestattete Elektronikarbeitsplätze, die von jedem Studierenden
 genutzt werden können. Darüber hinaus gibt es ein umfangreiches
 Kursangebot an, z.B. die begehrten Mikrocontrollerkurse. 
\\\\
  Organisiert von Studierenden als gemeinnütziger Verein ist die ags
  seit 1953 für Studierende aktiv. 
  \\\\
   Sie trifft sich jeden Dienstags und Donnerstags ab 19:00 Uhr in der
   Zimmerstraße 24c, ,,Keller des Grotrian''. Komm einfach vorbei!

\subsubsection{
