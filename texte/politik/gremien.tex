	\subsection*{ABC der Studentischen Selbstverwaltung}
		Folgende Gremien setzen sich für dich und die Interessen anderer Studierender an der Hochschule ein:

	\subsubsection*{Studierendenparlament}
		Eines der wichtigsten Elemente der studentischen Mitbestimmung ist das Studierendenparlament (Uni-Slang: StuPa). Es wird jedes Semester gewählt und entscheidet unter anderem über den studentischen Haushalt, den ihr als Teil des Semesterbeitrags zahlt. Außerdem werden hier Ausschüsse gewählt (Als wichtigster der \emph{Allgemeine Studierenden Ausschuss}, kurz AStA).

	\subsubsection*{AStA}
		Der Allgemeine Studierenden Ausschuss ist die \emph{Exekutive} der Studenten: Er vertritt euch nach Außen, also zum Beispiel bei Verhandlungen um das Semesterticket, versorgt euch mit Informationen zu politischen Themen (öfter im Semester erscheint der so genannte \emph{AStA-Issue}) und ist einer der ersten Ansprechpartner für eure Anliegen.

	\subsubsection*{Gremien}
		In der Uni gibt es unzählige Gremien, hier seien die wichtigsten genannt. Jedes Gremium hat eine bestimmte Besetzung, also eine definierte Anzahl von jeweils Studenten, Mitarbeitern und Professoren.

		Am relevantesten für euch ist die Studienkommission (\emph{StuKo}): Hier werden Details des Studiengangs besprochen, Probleme der Studenten geklärt und die Vergabe der Studiengebühren entschieden. In diesem Gremium herrscht ein Stimmengleichgewicht zwischen Studenten und Professoren. Das bedeutet, dass wir hier wirklich die Möglichkeit haben, aktiv in die Unipolitik einzugreifen.

		In der \emph{Informatik-Kommission} und im
		\emph{Fakultätsrat} (der außerdem noch Mitglieder aus
		den Sozial- und Wirtschaftswissenschaften, sowie der Mathematik hat) sieht es da schon schlechter aus, die Studenten stellen in beiden nur eine Minderheit der Stimmen.

		Die \emph{Berufungskommission} hat nur selten zu tun: Wann immer eine Professur besetzt werden muss, tagt sie um Kandidaten für das Amt zu finden.

		Wann immer ihr Anträge im Prüfungsamt stellt, landen diese im \emph{Prüfungsausschuss}, der entscheidet, ob diese rechtmäßig sind. In diesem sind drei Professoren, ein Student und ein Mitarbeiter vertreten.

		Der \emph{Zulassungsausschuss} bearbeitet einmal im Semester die Master-Bewerbungen. Dabei haben wir eine beratende Stimme, mit der wir zum Beispiel dafür sorgen, dass es nur wenige und sinnvolle Zulassungsauflagen gibt. 

	\subsection*{Uniweite Vollversammlung (VV)}
		Mindestens einmal im Semester findet eine
		Vollversammlung aller Studenten statt, d.h. theoretisch
		stürmen 13500 Studenten ins Audimax. Obwohl jeder kommen
		soll, reicht schon ein winziger Bruchteil dessen, damit
		die VV beschlussfähig ist. So können hier wichtige
		Themen abgestimmt werden, die alle Studenten betreffen,
		zum Beispiel wurde die Einführung des Semestertickets
		hier beschlossen. Leider kommt meist nichtmal dieser
		Bruchteil zustande, so dass die VV dann nur Empfehlungen
		an das StuPA aussprechen kann. Wenn ihr informiert darüber 
		bleiben wollt, was neben eurem Studiengang so an der Uni vor sich geht, solltet ihr diese Versammlungen nicht verpassen.

		\subsubsection*{Vollversammlung der Informatik}
			Was vor vielen Jahren noch regelmäßig war, ist
			zwischenzeitlich etwas eingeschlafen. Seit dem
			Sommersemester 2010 gibt es aber wieder VV's der
			Informatikstudenten, auf denen die Fachgruppe
			wichtige Informationen verkündet und einen
			breiteren Dialog sucht, als es über die
			Fachgruppentreffen oder die Mailkommunikation
			möglich ist. Hier ist die möglichkeit für jeden,
			sich mit minimalem Aufwand in die Gestaltung des
			Studienganges einzubringen. Auch hier gilt: Wenn
			20\% der Studenten anwesend sind  ist die VV offiziell beschlussfähig.