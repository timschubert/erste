%\begin{multicols}{2}
\subsection{Kleine Gremienkunde}
	Des Deutschen liebstes Kind ist -- nein, nicht sein Auto! Die Bürokratie, denn ohne sie herrschte Chaos im Dunkel und Angst, Furcht und Schrecken allüberall. Damit auch die Studierenden sich gut verwaltet fühlen dürfen, gibt es natürlich ebenfalls an der TU Braunschweig eine Menge Gremien und Organe, die Entscheidungen fällen (oder verschieben ;-)), Kompetenzen zuteilen (oder verschieben ;-)) und aufpassen, dass alles mehr oder weniger seinen demokratischen Gang geht.

	Damit ihr euch im Dschungel ein wenig besser orientieren könnt, wollen wir im Folgenden versuchen, die einzelnen Gremien und deren Aufgaben vorzustellen und euch zeigen, wie und in welchem Umfang ihr unmittelbar (durch Wahl) oder mittelbar (durch die Gewählten) Einfluss auf die Hochschulpolitik nehmen könnt. Für die, die lieber Bildchen begucken, stellen wir eine Grafik zur Verfügung, die ihr auf der Seite vor diesem Artikel findet -- schenkt ihr Beachtung, sie hat es verdient!

	Lange Rede, gar kein Sinn, wir fangen an:

	\subsubsection*{Organe der Studierendenschaft}
		Dieser inzwischen von allen maskulinen und femininen Kennzeichen befreite Begriff vereint nichts anderes als alle StudentInnen der TU-BS unter sich (also auch DICH!). Die StudentInnen, die mehr oder weniger zufällig an der gleichen Fakultät studieren, fasst man als \textbf{Fachschaft (FS)} zusammen, derer gibt es 10 an der guten alten Carolo-Wilhelmina. Eigentlich solltest du es inzwischen mitbekommen haben, aber du, verehrte/r LeserIn, gehörst mit großer Wahrscheinlichkeit zur Fachschaft der Fakultät 1. Gibt es innerhalb einer Fachschaft noch Unterschiede in den Studienrichtungen, so wird in \textbf{Fachgruppen (FG)} aufgeteilt, für dich ist das die Fachgruppe Informatik.

		Die Studierenden einer Fachschaft werden üblicherweise durch den \textbf{Fachschaftsrat (FSR)} vertreten. Da die Studiengänge der Fakultät 1 aber sehr unterschiedlich sind, findet der Hauptteil der Arbeit in den \textbf{Fachgruppenräten (FGR)} statt. Der FGR kümmert sich um die Belange der Fachgruppe, beruft die Fachgruppen-Vollversammlungen ein, streitet sich mit der Fakultät, wenn's mal wieder Meinungsverschiedenheiten wegen irgendwelcher Neuerungen gibt, organisiert die Orientierungseinheit für die Erstsemester am Anfang des Wintersemesters, verwaltet Prüfungsprotokolle, und über das Internet \url{http://fginfo.cs.tu-bs.de/} und trägt das ganze Semester über Informationen aus den verschiedenen Gremien zusammen, und an euch weiter. Für dich ist der FGR der wichtigste Ansprechpartner, denn auch wenn wir deine Probleme mal nicht lösen können, dann können wir dir wenigstens sagen, an wen oder was du dich wenden kannst. Damit auch zwischen den verschiedenen Fachschaften und Fachgruppen kommuniziert wird, gibt es das \textbf{Fachschaftenplenum}, was kein Gremium im eigentlichen Sinne ist, aber ein Forum zum Meinungs- und Interessenaustausch darstellt. Es trifft sich etwa einmal im Monat und ist für jeden offen, der einen Einstieg in die Unipolitik sucht.

		Ganz basisdemokratisch ist auf allen Hierarchie\-ebenen der Studierendenschaft die jeweilige \textbf{Vollversammlung (VV)} das oberste Organ, allerdings nur mit empfehlendem Charakter. Sie findet ein- bis zweimal pro Jahr statt und dort wird über Aktuelles und Wichtiges informiert und/oder abgestimmt. Eine Vollversammlung aller Studierenden wird vom StuPa-Präsidium, eine Fachschafts- oder Fachgruppen-VV vom FSR oder FGR einberufen und geleitet.

		Womit wir bei Abkürzungen wären, die noch nicht erklärt wurden -- aber keine Bange, das kommt: Das \textbf{Studierendenparlament (StuPa, SP)} ist die unmittelbare Vertretung aller StudentInnen und wird von der Studierendenschaft direkt in jedem Semester gewählt, und tagt \textbf{hochschulöffentlich}. Die etwa 40 Mitglieder des StuPa beschließen studentische Angelegenheiten, verabschieden den studentischen Haushalt und wählen den \textbf{Allgemeinen Studierendenausschuss (AStA)}, den \textbf{übergeordneten Wahlausschuss (ügWA)} und verschiedene weitere Ausschüsse. Das StuPa wählt außerdem sein eigenes Präsidium, welches die Sitzungen und (uniweiten) Vollversammlungen leitet und das StuPa nach außen hin vertritt.

		Von allen studentischen Ausschüssen ist der \textbf{AStA} sicherlich der sichtbarste. Er ist das ausführende Organ der Studierendenschaft und vertritt alle Studierenden nach außen, z.B. bei Verhandlungen mit der BVAG wegen des Semestertickets. Seine Aufgaben werden vom StuPa festgelegt und beinhalten z.B. Serviceangebote (Kopieren, Binden, Internationaler Studiausweis) oder Informationsquellen zu den unterschiedlichsten Themen. Um sich zu entlasten, kann er ReferentInnen bestellen, die sich um einzelne Bereiche mehr oder weniger hauptamtlich kümmern. Das zweite vom StuPa gewählte Gremium ist der \textbf{übergeordnete Wahlausschuss (ügWA)}, der die studentischen Wahlen organisiert und überwacht.

	\subsubsection*{Kollegialorgane}
		Neben den bis jetzt vorgestellten Organen der Verfassten Studierendenschaft gibt es natürlich auch noch Schnittstellen zwischen den Studis und den anderen an der Universität vertretenen Personengruppen, den MTVlern (MitarbeiterInnen aus Technik und Verwaltung), den WiMis (Wissenschaftliche MitarbeiterInnen, AssistentInnen) und natürlich den Lehrenden (ProfessorInnen). Hier ist das oberste Organ innerhalb der Fakultäten der \textbf{Fakultätsrat (FKR)}, dem 7 Professoren, 2 Studis, 2 MTVler und 2 WiMis angehören. Hier wird all das entschieden, was andere Gremien oder das Dekanat erarbeitet haben, bspw. Änderungen an der BPO. Wird eine Entscheidung getroffen, so ist diese sozusagen offiziell geworden und kann umgesetzt werden. Da auf Grund der Stimmenverteilung (s.o.) die Professoren immer eine Mehrheit haben, müssen wir in den Gremien, die vorher die inhaltliche Arbeit leisten, versuchen, unsere und eure Vorstellungen einzubringen. Die studentischen Vertreter werden einmal im Jahr, jeweils im Wintersemester, direkt gewählt. Da wie gesagt die Mathematik und die Informatik doch durchaus unterschiedliche Studiengänge sind, gibt es einen nicht formellen \emph{kleinen Fakultätsrat}, die \textbf{Informatik-Kommission}. Die Informatik-Kommission, die im Verhältnis 3 : 1 : 1 : 1 besetzt ist, berät informatikspezifische Dinge und bereitet sie für den Fakultätsrat vor, damit die Entscheidungen im FKR schneller gefällt werden können und sich die Mathematiker nicht so langweilen ;-).

		Das formal oberste Gremium der Uni ist der \textbf{Senat}, der sich mit allgemeinen Sachen befasst, die über der Zuständigkeit der Fakultäten liegen (als wichtiger Punkt ist hier die Verteilung des universitären Haushaltes zu nennen). Wie in den FKR ist hier die Stimmengewichtung 7 : 2 : 2 : 2, auch seine Mitglieder werden jährlich gewählt. Wie der AStA hat auch der Senat die Möglichkeit, seine Arbeit unterstützende Kommissionen einzusetzen.

	\subsubsection*{Kommissionen und Ausschüsse}
		Da wir so oft Kommissionen und Ausschüsse erwähnt haben, seien die drei wichtigsten hier kurz vorgestellt: zunächst ist da die \textbf{Studienkommission (StuKo)} zu erwähnen, die mit dem neuen Niedersächsischen Hochschulgesetz (NHG) im vergangenen Jahr eingeführt wurde. Sie ist das einzige gemischte Gremium, in dem die Studierenden die Mehrheit haben: 1 : 2 : 0 : 1 lautet die Verteilung der stimmberechtigen Mitglieder. Die Studienkommission erarbeitet vor allem Vorschläge für die Verbesserung der Qualität in der Lehre, so werden z.B. Vorschläge zur Änderung der Studienordnung und der BPO diskutiert. Die Studienkommission muss vor allen Entscheidungen des Fakultätsrates, welche die Lehre, das Studium oder Prüfungen betreffen, angehört werden. Eingesetzt wird die StuKo von den Fakultätsräten, die studentischen Vertreter rekrutieren sich meist aus den FSR/FGRn oder deren Umfeld (obwohl theoretisch jede/r Interessierte mitarbeiten kann). Die Sitzungen sind hochschulöffentlich, d.h. auch nicht gewählte Studierende können (und sollten) dort jederzeit ihre Stimme einbringen.

		Auch Professoren ist es einmal vergönnt, sich in den Ruhestand zu begeben oder andere Hochschulluft zu schnuppern. Wenn dies ansteht, dann muss die freigewordene Stelle (logischerweise) in den meisten Fällen neu besetzt werden. Dafür wird eine \textbf{Berufungskommission} vom Senat eingesetzt, um die Nachfolge zu regeln. Hier werden die Kandidaten, nachdem eine Vorauswahl getroffen wurde, sozusagen auf Herz und Nieren überprüft, und zwar im Rahmen eines öffentlichen Vortrags, den sich jede/r Interessierte anhören kann. Die zwei studentischen Vertreter in der Kommission interessiert dabei vor allem, ob der/die KandidatIn fähig ist, eine Vorlesung verständlich und klar strukturiert zu halten oder ob er sich in schweren wissenschaftlichen Formulierungen verliert, denn es gibt immer wieder Personen, die sichhauptsächlich auf die Forschungs- und kaum auf die Lehraufgaben konzentrieren. Die Berufungskommission erstellt nach ausgiebigen Beratungen eine Liste, die, nachdem sie den Senat passiert hat, ans Kultusministerium (MWK) weitergeleitet wird, das dann nach dieser Liste entscheidet, mit wem es, vertreten durch den Uni-Präsidenten, der ja formal auch Angestellter des MWK ist, in Verhandlungen tritt.

		Ein ziemlich wichtiger, von den FKR eingesetzter Ausschuss ist der \textbf{Prüfungsausschuss (PA)}. Der PA besteht aus 5 Mitgliedern (3 : 1 : 0 : 1) und ist für alle Fragen zuständig, die im Zusammenhang mit Prüfungen auftreten können. Bei (fast) allen Problemen, die mit Prüfungen zusammenhängen, kann man sich an den Prüfungsausschuss wenden - so können z.B. weitere Nebenfächer auf Antrag der Studierenden vom Prüfungsausschuss genehmigt werden. 

		Daneben gibt es natürlich noch ungezählte weitere kleine und große Gremien, Ausschüsse, Kommissionen und damit verbunden viele viele Pöstchen, die immer wieder zu vergeben sind. Wenn ihr also Blut geleckt habt und nicht nur durch eure Beteiligung bei den Wahlen Einfluss auf die Hochschulpolitik nehmen wollt, dann meldet euch doch im Fachschaftsrat und arbeitet mit -- ihr seid herzlich willkommen!
%\end{multicols}
