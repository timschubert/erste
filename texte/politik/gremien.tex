	\subsection{ABC der Studentischen Selbstverwaltung}
	Zusammen bilden alle Studierenden unserer Uni die Studierendenschaft. Ihre Vertreter setzen sich für dich und die
	Interessen anderer Studierenden, unter anderen in folgenden
	Gremien, ein:

	\subsubsection*{AStA}
		Der Allgemeine Studierenden Ausschuss ist die
		\emph{Exekutive} der Studierdendenschaftten: Er vertritt
		sie nach Außen, also zum Beispiel bei Verhandlungen um
		das Semesterticket, versorgt sie mit Informationen zu
		politischen Themen  und ist einer der ersten
		Ansprechpartner für  Anliegen, die die gesamte Uni
		betreffen. Ausserdem bietet er einige Service-Angebote
		an (Kopierer, Internationaler Studierendenausweis etc.).

		\subsubsection*{Weitere Gremien}
		Neben der verfassten Studierendenschaft gibt es an der Uni  unzählige Gremien, hier seien aber nur die wichtigsten genannt. Jedes Gremium hat eine bestimmte Besetzung, also eine definierte Anzahl von jeweils Studierenden, wissenschaftlichen Mitarbeiterinnen und  Mitarbeitern (kurz WiMi), sowie Profesorinnen  und Professoren. In einigen Gremien sind außerdem noch Mitarbeiter und Mitarbeiterinnen aus Technik und Verwaltung (Kurz MTV) vertreten.

		Am relevantesten für dich ist die Studienkommission (\emph{StuKo}): Hier werden Details des Studiengangs besprochen, Probleme der Studierenden geklärt und über die Vergabe der Studiengebühren entschieden. In diesem Gremium herrscht ein Stimmengleichgewicht zwischen Studierenden und Lehrenden. Das bedeutet, dass wir hier wirklich die Möglichkeit haben, aktiv in die Unipolitik einzugreifen.

		In der \emph{Informatik-Kommission} und im
		\emph{Fakultätsrat} (der außerdem noch Mitglieder aus
		den Sozial- und Wirtschaftswissenschaften, sowie der Mathematik hat) sieht es da schon schlechter aus, die Studierenden stellen in beiden nur eine Minderheit der Stimmen.

		Die \emph{Berufungskommission} hat nur selten etwas zu tun: Wann immer eine Professur besetzt werden muss, tagt sie, um jemand geeigneten zu finden. Auch wenn auch hier Studierende nur mit einer Stimme vertreten sind, gibt es hier die Möglichkeit darauf Einfluss zu nehmen, wer auf zukünftige Studierendengenerationen losgelassen wird.

		Wann immer du einen Antrag im Prüfungsamt stellt, landet
		dieser im \emph{Prüfungsausschuss}, der darüber
		entscheidet. In diesem sind drei Professor/innen, ein/e Studierende/r und ein WiMi vertreten.

		Der \emph{Zulassungsausschuss} bearbeitet einmal im
		Semester die Master-Bewerbungen. Dabei haben wir eine
		beratende Stimme. Wir versuchen dabei zu erreichen,
		dass es möglichst wenige  und sinnvolle Zulassungsauflagen gibt.

	\subsubsection*{Studierendenparlament}
		Eines der wichtigsten Elemente der studentischen
		Mitbestimmung ist das Studierendenparlament (Uni-Slang:
		StuPa). Es wird jedes Semester gewählt und entscheidet
		unter anderem über den studentischen Haushalt, für den
		du einen Teil des Semesterbeitrags zahlst. Außerdem werden hier Ausschüsse gewählt (Als wichtigster der \emph{Allgemeine Studierenden Ausschuss}, kurz AStA).
	\subsection*{Uniweite Vollversammlung (VV)}
		Mindestens einmal im Semester findet eine
		Vollversammlung aller Studierenden statt, d.h. theoretisch
		stürmen 13.500 Studierende ins Audimax. Obwohl alle kommen
		sollen, reicht schon ein winziger Bruchteil, damit
		die VV beschlussfähig ist. So kann hier über wichtige
		Themen abgestimmt werden, die alle Studierende betreffen.
		Beispielsweise wurde hier die Einführung des Semestertickets
		hier beschlossen. Leider kommt meist nichtmal dieser
		Bruchteil zustande, so dass die VV  nur Empfehlungen
		an das StuPa aussprechen kann. Wenn du informiert darüber 
		bleiben willst, was neben deinen Studiengang so an der
		Uni passiert, solltest du diese Versammlungen nicht verpassen.

		\subsubsection*{Vollversammlung der Informatik}
			Was vor vielen Jahren noch regelmäßig war, ist
			zwischenzeitlich etwas eingeschlafen. Seit dem
			Sommersemester 2010 gibt es aber wieder VV's der
			in der Informatik, auf denen der
			Fachgruppenrat
			wichtige Informationen verkündet und einen
			breiteren Dialog sucht, als es über die
			Fachgruppentreffen oder die Mailkommunikation
			möglich ist. Hier ist die Möglichkeit für jeden,
			sich mit minimalem Aufwand in die Gestaltung des
			Studienganges einzubringen. Dabei gilt: Wenn
			20\% der Informatikstudierenden anwesend sind  ist die VV offiziell beschlussfähig.
