\begin{multicols}{2}
\subsection{Kleine Gremienkunde}
	Des Deutschen liebstes Kind ist -- nein, nicht sein Auto! Die B"urokratie, denn ohne sie herrschte Chaos im Dunkel und Angst, Furcht und Schrecken all"uberall. Damit auch die Studierenden sich gut verwaltet f"uhlen d"urfen, gibt es nat"urlich ebenfalls an der TU Braunschweig eine Menge Gremien und Organe, die Entscheidungen f"allen (oder verschieben ;-)), Kompetenzen zuteilen (oder verschieben ;-)) und aufpassen, dass alles mehr oder weniger seinen demokratischen Gang geht.

	Damit ihr euch im Dschungel ein wenig besser orientieren k"onnt, wollen wir im Folgenden versuchen, die einzelnen Gremien und deren Aufgaben vorzustellen und euch zeigen, wie und in welchem Umfang ihr unmittelbar (durch Wahl) oder mittelbar (durch die Gew"ahlten) Einfluss auf die Hochschulpolitik nehmen k"onnt. F"ur die, die lieber Bildchen begucken, stellen wir eine Grafik zur Verf"ugung, die ihr auf der Seite vor diesem Artikel findet -- schenkt ihr Beachtung, sie hat es verdient!

	Lange Rede, gar kein Sinn, wir fangen an:

	\subsubsection*{Organe der Studierendenschaft}
		Dieser inzwischen von allen maskulinen und femininen Kennzeichen befreite Begriff vereint nichts anderes als alle StudentInnen der TU-BS unter sich (also auch DICH!). Die StudentInnen, die mehr oder weniger zuf"allig an der gleichen Fakult"at studieren, fasst man als \textbf{Fachschaft (FS)} zusammen, derer gibt es 10 an der guten alten Carolo-Wilhelmina. Eigentlich solltest du es inzwischen mitbekommen haben, aber du, verehrte/r LeserIn, geh"orst mit gro"ser Wahrscheinlichkeit zur Fachschaft der Fakultät 1. Gibt es innerhalb einer Fachschaft noch Unterschiede in den Studienrichtungen, so wird in \textbf{Fachgruppen (FG)} aufgeteilt, für dich ist das die Fachgruppe Informatik.

		Die Studierenden einer Fachschaft werden üblicherweise durch den \textbf{Fachschaftsrat (FSR)} vertreten. Da die Studiengänge der Fakultät 1 aber sehr unterschiedlich sind, findet der Hauptteil der Arbeit in den \textbf{Fachgruppenr"aten (FGR)} statt. Der FGR k"ummert sich um die Belange der Fachgruppe, beruft die Fachgruppen-Vollversammlungen ein, streitet sich mit der Fakult"at, wenn's mal wieder Meinungsverschiedenheiten wegen irgendwelcher Neuerungen gibt, organisiert die Orientierungseinheit f"ur die Erstsemester am Anfang des Wintersemesters, verwaltet Pr"ufungsprotokolle, und über das Internet \url{http://fginfo.cs.tu-bs.de/} und tr"agt das ganze Semester über Informationen aus den verschiedenen Gremien zusammen, und an euch weiter. F"ur dich ist der FGR der wichtigste Ansprechpartner, denn auch wenn wir deine Probleme mal nicht l"osen k"onnen, dann k"onnen wir dir wenigstens sagen, an wen oder was du dich wenden kannst. Damit auch zwischen den verschiedenen Fachschaften und Fachgruppen kommuniziert wird, gibt es das \textbf{Fachschaftenplenum}, was kein Gremium im eigentlichen Sinne ist, aber ein Forum zum Meinungs- und Interessenaustausch darstellt. Es trifft sich etwa einmal im Monat und ist f"ur jeden offen, der einen Einstieg in die Unipolitik sucht.

		Ganz basisdemokratisch ist auf allen Hierarchie\-ebenen der Studierendenschaft die jeweilige \textbf{Vollversammlung (VV)} das oberste Organ, allerdings nur mit empfehlendem Charakter. Sie findet ein- bis zweimal pro Jahr statt und dort wird über Aktuelles und Wichtiges informiert und/oder abgestimmt. Eine Vollversammlung aller Studierenden wird vom StuPa-Pr"asidium, eine Fachschafts- oder Fachgruppen-VV vom FSR oder FGR einberufen und geleitet.

		Womit wir bei Abk"urzungen w"aren, die noch nicht erkl"art wurden -- aber keine Bange, das kommt: Das \textbf{Studierendenparlament (StuPa, SP)} ist die unmittelbare Vertretung aller StudentInnen und wird von der Studierendenschaft direkt in jedem Semester gew"ahlt, und tagt \textbf{hochschul"offentlich}. Die etwa 40 Mitglieder des StuPa beschlie"sen studentische Angelegenheiten, verabschieden den studentischen Haushalt und w"ahlen den \textbf{Allgemeinen Studierendenausschuss (AStA)}, den \textbf{"Ubergeordneten Wahlausschuss ("UgWA)} und verschiedene weitere Aussch"usse. Das StuPa w"ahlt au"serdem sein eigenes Pr"asidium, welches die Sitzungen und (uniweiten) Vollversammlungen leitet und das StuPa nach au"sen hin vertritt.

		Von allen studentischen Aussch"ussen ist der \textbf{AStA} sicherlich der sichtbarste. Er ist das ausf"uhrende Organ der Studierendenschaft und vertritt alle Studierenden nach au"sen, z.B. bei Verhandlungen mit der BVAG wegen des Semestertickets. Seine Aufgaben werden vom StuPa festgelegt und beinhalten z.B. Serviceangebote (Kopieren, Binden, Internationaler Studiausweis) oder Informationsquellen zu den unterschiedlichsten Themen. Um sich zu entlasten, kann er ReferentInnen bestellen, die sich um einzelne Bereiche mehr oder weniger hauptamtlich k"ummern. Das zweite vom StuPa gew"ahlte Gremium ist der \textbf{"Ubergeordnete Wahlausschuss ("UgWA)}, der die studentischen Wahlen organisiert und "uberwacht.

	\subsubsection*{Kollegialorgane}
		Neben den bis jetzt vorgestellten Organen der Verfassten Studierendenschaft gibt es nat"urlich auch noch Schnittstellen zwischen den Studis und den anderen an der Universit"at vertretenen Personengruppen, den MTVlern (MitarbeiterInnen aus Technik und Verwaltung), den WiMis (Wissenschaftliche MitarbeiterInnen, AssistentInnen) und nat"urlich den Lehrenden (ProfessorInnen). Hier ist das oberste Organ innerhalb der Fakult"aten der \textbf{Fakult"atsrat (FKR)}, dem 7 Professoren, 2 Studis, 2 MTVler und 2 WiMis angeh"oren. Hier wird all das entschieden, was andere Gremien oder das Dekanat erarbeitet haben, bspw. Änderungen an der BPO. Wird eine Entscheidung getroffen, so ist diese sozusagen offiziell geworden und kann umgesetzt werden. Da auf Grund der Stimmenverteilung (s.o.) die Professoren immer eine Mehrheit haben, m"ussen wir in den Gremien, die vorher die inhaltliche Arbeit leisten, versuchen, unsere und eure Vorstellungen einzubringen. Die studentischen Vertreter werden einmal im Jahr, jeweils im Wintersemester, direkt gew"ahlt. Da wie gesagt die Mathematik und die Informatik doch durchaus unterschiedliche Studieng"ange sind, gibt es einen nicht formellen \emph{kleinen Fakult"atsrat}, die \textbf{Informatik-Kommission}. Die Informatik-Kommission, die im Verh"altnis 3 : 1 : 1 : 1 besetzt ist, ber"at informatikspezifische Dinge und bereitet sie f"ur den Fakult"atsrat vor, damit die Entscheidungen im FKR schneller gef"allt werden k"onnen und sich die Mathematiker nicht so langweilen ;-).

		Das formal oberste Gremium der Uni ist der \textbf{Senat}, der sich mit allgemeinen Sachen befasst, die über der Zust"andigkeit der Fakult"aten liegen (als wichtiger Punkt ist hier die Verteilung des universit"aren Haushaltes zu nennen). Wie in den FKR ist hier die Stimmengewichtung 7 : 2 : 2 : 2, auch seine Mitglieder werden j"ahrlich gew"ahlt. Wie der AStA hat auch der Senat die M"oglichkeit, seine Arbeit unterst"utzende Kommissionen einzusetzen.

	\subsubsection*{Kommissionen und Aussch"usse}
		Da wir so oft Kommissionen und Aussch"usse erw"ahnt haben, seien die drei wichtigsten hier kurz vorgestellt: zun"achst ist da die \textbf{Studienkommission (StuKo)} zu erw"ahnen, die mit dem neuen Nieders"achsischen Hochschulgesetz (NHG) im vergangenen Jahr eingef"uhrt wurde. Sie ist das einzige gemischte Gremium, in dem die Studierenden die Mehrheit haben: 1 : 2 : 0 : 1 lautet die Verteilung der stimmberechtigen Mitglieder. Die Studienkommission erarbeitet vor allem Vorschl"age f"ur die Verbesserung der Qualit"at in der Lehre, so werden z.B. Vorschl"age zur Änderung der Studienordnung und der BPO diskutiert. Die Studienkommission muss vor allen Entscheidungen des Fakult"atsrates, welche die Lehre, das Studium oder Pr"ufungen betreffen, angeh"ort werden. Eingesetzt wird die StuKo von den Fakult"atsr"aten, die studentischen Vertreter rekrutieren sich meist aus den FSR/FGRn oder deren Umfeld (obwohl theoretisch jede/r Interessierte mitarbeiten kann). Die Sitzungen sind hochschul"offentlich, d.h. auch nicht gew"ahlte Studierende k"onnen (und sollten) dort jederzeit ihre Stimme einbringen.

		Auch Professoren ist es einmal verg"onnt, sich in den Ruhestand zu begeben oder andere Hochschulluft zu schnuppern. Wenn dies ansteht, dann muss die freigewordene Stelle (logischerweise) in den meisten F"allen neu besetzt werden. Daf"ur wird eine \textbf{Berufungskommission} vom Senat eingesetzt, um die Nachfolge zu regeln. Hier werden die Kandidaten, nachdem eine Vorauswahl getroffen wurde, sozusagen auf Herz und Nieren überpr"uft, und zwar im Rahmen eines öffentlichen Vortrags, den sich jede/r Interessierte anh"oren kann. Die zwei studentischen Vertreter in der Kommission interessiert dabei vor allem, ob der/die KandidatIn f"ahig ist, eine Vorlesung verst"andlich und klar strukturiert zu halten oder ob er sich in schweren wissenschaftlichen Formulierungen verliert, denn es gibt immer wieder Personen, die sichhaupts"achlich auf die Forschungs- und kaum auf die Lehraufgaben konzentrieren. Die Berufungskommission erstellt nach ausgiebigen Beratungen eine Liste, die, nachdem sie den Senat passiert hat, ans Kultusministerium (MWK) weitergeleitet wird, das dann nach dieser Liste entscheidet, mit wem es, vertreten durch den Uni-Pr"asidenten, der ja formal auch Angestellter des MWK ist, in Verhandlungen tritt.

		Ein ziemlich wichtiger, von den FKR eingesetzter Ausschuss ist der \textbf{Pr"ufungsausschuss (PA)}. Der PA besteht aus 5 Mitgliedern (3 : 1 : 0 : 1) und ist f"ur alle Fragen zust"andig, die im Zusammenhang mit Pr"ufungen auftreten k"onnen. Bei (fast) allen Problemen, die mit Pr"ufungen zusammenh"angen, kann man sich an den Pr"ufungsausschuss wenden - so k"onnen z.B. weitere Nebenf"acher auf Antrag der Studierenden vom Pr"ufungsausschuss genehmigt werden. 

		Daneben gibt es nat"urlich noch ungez"ahlte weitere kleine und gro"se Gremien, Aussch"usse, Kommissionen und damit verbunden viele viele P"ostchen, die immer wieder zu vergeben sind. Wenn ihr also Blut geleckt habt und nicht nur durch eure Beteiligung bei den Wahlen Einfluss auf die Hochschulpolitik nehmen wollt, dann meldet euch doch im Fachschaftsrat und arbeitet mit -- ihr seid herzlich willkommen!
\end{multicols}