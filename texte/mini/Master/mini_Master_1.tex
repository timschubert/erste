\section*{Das Wichtigste aus der 1-ten}
  Auf detailiertere Informationen in der vollständigen Ausgabe wird 
  verwiesen: \mehrInfo{Abschnitt, Seite}. Viel Spaß beim Lesen wünscht euch eure
  Fachgruppe.%\end{multicols}
\subsection*{Termine und Checkliste}
In der Tabelle steht das B für Bachelor und das M für Master. 
\mehrInfo{Abschnitt ,,Die ersten Tage'', Seite 3 und 4}. Dort findet
ihr auch eine Checkliste mit den wichtigsten zu erledigenden Dingen.
\end{multicols}
\begin{tabular}{|l|l|p{6.7cm}|c|c|c|}
\hline \textbf{Datum} 		& \textbf{Uhrzeit} 	& \textbf{Veranstaltung}						& \textbf{Ort} 	& \textbf{B}	& \textbf{M} 	\\
\hline 26.09. – 30.09.		& 1. Tag: 10:00	 	& Vorkurs Informatik							& PK 2.1		& B				& 				\\
	   17.10. – 21.10.		& 					& 												& 				& 				&    			\\
\hline Mo, 24.10. 			& 09:00 – 10:00		& Begrüßung	durch den Präsidenten				& Stadion		& B				& M				\\ 
\hline 						& 10:00 – 12:00	 	& Infobörse										& Altgebäude	& B				& M				\\
\hline   					& 10:30 – 12:00	 	& Begrüßung (M) \newline durch den Studiendekan	& IZ 160		& 				& M				\\
\hline 						& 13:30 – 15:00	 	& Begrüßung (B) \newline durch den Studiendekan	& PK 11.3		& B				& 				\\
\hline 						& 15:00 – 16:30		& Erste Vorlesung ,,Programmieren 1''			& SN 19.1		& B 			&				\\
\hline Di, 25.10.			& 10:00 – 11:45 	& Erstsemester-Frühstück 						& IZ Plaza 		& B 			& M 			\\ 
\hline 						& 12:00 – 13:30 * 	& FG-Einführung und \newline Stundenplan-Bauen 	& ??? * 		& B 			&  				\\%& ??? * & B &\\
\hline 						& 12:00 – 13:30 * 	& FG-Einführung und \newline Stundenplan-Bauen 	& IZ 160 * 		& 				& M				\\
\hline 						& 13:30 – 15:00 	& Rundgang mit den  Tutorengruppen 				& IZ 160 		& B 			& M				\\
\hline Mi, 26.10.			& 10:00 – 17:00		& Studium Generale								& Altgebäude	& B				& M 			\\
\hline Do, 03.11. 			& 19:00 			& Kneipentour der Fachgruppe 					& IZ 150 		& B 			& M				\\
\hline Mi, 09.11.	 		& 19:00 			& Spieleabend der Fachgruppe 					& vor  150 		& B 			& M				\\
\hline
\end{tabular} 

%%% Local Variables: 
%%% mode: latex
%%% TeX-master: "../../1-te"
%%% End: 
%termine}\\
%\begin{multicols}{2}
%\mehrInfo{Termine auf auf Seite \pageref{termine}}
% \subsection*{Checkliste für Erstsemester}
% \mehrInfo{Abschnitt ,,Die ersten Tage'', Seite 4}\\
% Dort findet ihr auch die
% richtigen Seitenzahlen und nicht nur Fragezeichen :)\\
% %\end{multicols}
% % !TEX root = ../../1-te.tex

\begin{tabular}{|p{3mm}|l|l|c|c|}
\hline \checkmark 
       & \textbf{Todo}             & \textbf{Zu erledigen bis}                                  & \textbf{Seite}               & \textbf{Muss?} \\ 
\hline & BAföG beantragen          & Spätestens Ende \iftoggle{winter}{Oktober}{April}          & \pageref{todobafoeg}         & optional \\ 
\hline & Wohnsitz Ummelden         & 1 Woche nach Umzug                                         & \pageref{todoummelden}       & ja \\ 
\hline & Mailinglisten             & So früh wie möglich                                        & \pageref{mailinglisten}      & ja \\ 
\hline & Studiengrobplanung        & Vor dem Stundenplanbauen                                   & \pageref{grob}               & ja \\ 
\hline & Auflagen klären           & So früh wie möglich, final: Ende 2. Semester               & \pageref{auflagen}           & nur Master \\ 
\hline & Persönlicher Stundenplan  & Siehe Terminzettel der Fachgruppe                          & \pageref{masterstundenplan}  & ja \\ 
\hline & Prüfungsbogen             & Apätestens \iftoggle{winter}{Dezember}{Mai}                & \pageref{todoanmeldung}      & ja \\ 
\hline & Prüfungsanmeldung         & Anmeldewoche (30.05 - 03.06) oder Online (09.05 - 20.06)   & \pageref{todoanmeldung}      & ja \\ 
\hline & Blog abonnieren           & So früh wie möglich                                        & \pageref{fachgruppe}         & ja \\ 
\hline & Prüfungsordnung lesen     & Zu den ersten Klausuren                                    & \pageref{po}                 & ja \\ 
\hline & TUcard validieren         & Zu Beginn und zu jedem neuen Semester                      & \pageref{tucard}             & ja \\
\hline & Bibliotheksausweis        & Vor der ersten Buchausleihe                                & \pageref{todobib}            & optional \\
\hline & Kopierkarte               & Wenn man was kopieren muss                                 & \pageref{todobib}            & optional \\ 
\hline
\end{tabular} 
\tocheck{2}{Exakte Daten Anmeldewoche einfügen, s.\url{https://www.tu-braunschweig.de/fk1/service/informatik/pa/}}
\begin{multicols}{2}
\subsection*{Tutorien}
Nach dem Erstsemesterfrühstück werden die Erstsemester nach Bachelor und
Master getrennt in Tutorengruppen aufgeteilt. Diese kleinen Gruppen erkunden den Campus. Auftretende Fragen sollten dem Tutor gestellt werden. Im 
\mehrInfo{Abschnitt Tutorien, Seite 5} findet ihr sowohl mehr Informationen, als auch Bilder und Emailadressen
einiger Tutoren.%\newpage
\subsection*{Studienplanung}
Pro Semester sollten ungefähr 30 Credits erreicht werden, damit man in
6 Semestern den Bachelor und in 4 den Master besteht.
Einen Stundenplan für das 1. Semester Bachelor und Studienpläne für den
weiteren Verlauf findet ihr auf den folgenden Seiten.
Dabei gilt: Die Pläne sind
  lediglich ein Vorschlag, ihr seid nicht daran gebunden. An folgenden
  Stellen findet ihr dazu mehr Informationen:\\
%Allgemeine Studienplanung \\
\mehrInfo{Allgemeine Studienplanung ab  Seite 7}\\
\mehrInfo{ Zum Bachelor ab Seite
  18 }\\
\mehrInfo{Zum Master ab Seite 29 \hfill}
