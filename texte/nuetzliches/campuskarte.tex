\subsection{Campuskarten und Raumnummern}
\label{campuskarte}
\ifpdf % Hier muss eine Abfrage rein, denn die Karte im Einband gibt es in der Onlinefassung nicht. Könnte man zwar stattdessen hier einfügen, mach ich aber gerade nicht.
Bereits auf Seite \pageref{plan} habt ihr einen einfachen Campusplan
gesehen. Es gibt aber noch diverse andere Online:\\
\else
Ihr findet im Netz diverse Campuskarten:\\
\fi
Eine aktuelle Campuskarte, die durchsucht werden kann findet sich im
TUgether Portal unter \nurl{https://tugether.tu-braunschweig.de/}. Ihr
könnt euch dort mit eurer Y-Nummer einloggen.\\\\

Ein Raumplan für das 1. und 2. OG des Informatikzentrums findet sich
unter\\ \nurl{http://www.ibr.cs.tu-bs.de/rooms/rooms.html}
Sollten euch die genannten Links zu unhandlich zum Abtippen sein, findet ihr
auch alle unter
\nurl{http://fginfo.cs.tu-bs.de/index.php/studium/lernraume/}.
Dort findet ihr auch virtuelle Campustouren, die  in einem Web 2.0 Seminar entstanden und
bei Google Maps gehostet sind.

Für die Suche nach einem Raum solltet ihr noch wissen, wie sich die Raumnummern bilden: Bei Nummern wie \textit{PK 15.1} sind die Buchstaben ein Kürzel für die Straße, in dem das Gebäude liegt. Die Zahl vor dem Punkt ist meist die Hausnummer, und nach dem Punkt eine willkürliche Durchnummerierung. Anders bei Kürzeln wie \textit{IZ 150}, bei denen IZ das Informatikzentrum an der Mühlenpfordstraße meint, die erste Stelle für die Etage steht (zwischen $0$ und $1$ kommt $G$, ist doch klar, oder?) und bei beiden letzten den Raum innerhalb der Etage. Die \textit{Plaza} ist der große Platz im ersten Stock bei den Aufzügen.
%ekeliger tabellenhack NICHT NACHMACHEN
%\begin{tabular}{c}
%\\  \\ \\ \\ \\ \\
%\end{tabular}
