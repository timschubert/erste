% !TEX root = ../../1-te.tex

\subsection{Campuskarten und Raumnummern}
\label{campuskarte}
Eine aktuelle Campuskarte, die durchsucht werden kann, findet sich unter \verUrl{5}{https://www.tu-braunschweig.de/suchoptionen/ortsfinder}.

%Ein Raumplan für das 1. und 2. OG des Informatikzentrums findet sich unter \verUrl{4}{http://www.ibr.cs.tu-bs.de/rooms/rooms.html}

%Sollten dir die genannten Links zu unhandlich zum Abtippen sein, findest du auch alle auf unserem Blog. Dort findest du auch virtuelle Campustouren, die in einem Web 2.0 Seminar entstanden und bei Google Maps gehostet sind.

Für die Suche nach einem Raum solltest du noch wissen, wie sich die Raumnummern bilden: Bei Nummern wie \textit{PK 15.1} sind die Buchstaben ein Kürzel für die Straße (Pockelstraße), in dem das Gebäude liegt. Die Zahl vor dem Punkt ist meist die Hausnummer, und nach dem Punkt eine willkürliche Durchnummerierung der Räume.\\ Anders bei Kürzeln wie \textit{IZ 150}. Bei denen steht IZ für das Informatikzentrum an der Mühlenpfordstraße. Die erste Stelle der Zahl steht für die Etage und die beiden letzten bezeichnen den Raum innerhalb der Etage. Die \textit{Plaza} ist der große Platz im ersten Stock bei den Aufzügen.
