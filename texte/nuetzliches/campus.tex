% !TEX root = ../../1-te.tex

\subsection{Campuskarten und Raumnummern}
\label{campuskarte}
Eine aktuelle Campuskarte, die durchsucht werden kann, findet sich unter \verUrl{2}{https://www.tu-braunschweig.de/suchoptionen/ortsfinder}.

Ein Raumplan für das 1. und 2. OG des Informatikzentrums findet sich
unter \verUrl{2}{http://www.ibr.cs.tu-bs.de/rooms/rooms.html}

%Sollten euch die genannten Links zu unhandlich zum Abtippen sein, findet ihr
%auch alle auf unserem Blog. Dort findet ihr auch virtuelle Campustouren, die  in einem Web 2.0 Seminar entstanden und
%%bei Google Maps gehostet sind.

Für die Suche nach einem Raum solltet ihr noch wissen, wie sich die Raumnummern bilden: Bei Nummern wie \textit{PK 15.1} sind die Buchstaben ein Kürzel für die Straße, in dem das Gebäude liegt. Die Zahl vor dem Punkt ist meist die Hausnummer, und nach dem Punkt eine willkürliche Durchnummerierung. Anders bei Kürzeln wie \textit{IZ 150}, bei denen IZ das Informatikzentrum an der Mühlenpfordstraße meint, die erste Stelle für die Etage steht und bei beiden letzten den Raum innerhalb der Etage. Die \textit{Plaza} ist der große Platz im ersten Stock bei den Aufzügen.
