% !TEX root = ../../1-te.tex

\subsection{Studienplanung im Bachelor}

Wie geht das eigentlich, studieren?

Wie du lernst, studierst, lebst; ob du brav mitschreibst oder öfter mal ausschläfst kannst und musst du selbst entscheiden. Wann du die vorgeschriebenen Lehrveranstaltungen belegst, liegt ebenfalls in deinem eigenen Ermessen, allerdings: Nachdem bis auf vier Ausnahmen klar festgelegt ist was du studieren musst, ergibt sich eine sinnvolle Reihenfolge, da beispielsweise fortgeschrittenes Programmieren ohne Kenntnis von Algorithmen schlicht nicht möglich ist. Nichtsdestotrotz hast du Spielraum, das Studium an deine persönliche Situation anzupassen.

Du wohnst noch zu Hause und brauchst nicht arbeiten? Prima, mach noch \iftoggle{winter}{Theoretische Informatik 1}{Technische Informatik 2} im 1. Semester. Du hast ein Kind und musst nebenbei auch noch arbeiten? Kein Problem, sprich dich mit deinem Mentor ab und mach ein Teilzeitstudium. Die konkreten Vorschriften zum Studium findest du in der Prüfungsordnung.

\tocheck{4}{Stimme die Credit-Anzahlen noch mit der BPO überein (sprich, gab es eine neue BPO)?}

In kurz: Grundsätzlich musst du Veranstaltungen im Wert von 180 Credit Points (CP) erfolgreich absolvieren, davon 130 CP im Bereich Informatik, 35 CP in Mathematik, 10 CP für dein Nebenfach\footnote{Bei Belegung des Nebenfachs \enquote{Betriebswirtschaftslehre} abweichend davon 12 CP} und 5 CP für Schlüsselqualifikationen. 

\tocheck{4}{Beschreibung der Musterstudienpläne aktuell?}

Um dir einen sinnvollen Weg durchs Studium zu ermöglichen, gibt es von der Fakultät den Musterstudienplan, der versucht, Überschneidungen der Veranstaltungen zu vermeiden. Es gibt aber auch noch Alternativstudienplan der Fachgruppe, die in bestimmten Situationen sinnvoll sein können. 

Auf den folgenden Seiten findest du die erwähnten Pläne. Der erste ist der Musterstudienplan der Fakultät. Der zweite Musterstudienplan wurde durch die Fachgruppe erstellt, um einige Probleme mit dem durch die Fakultät bereitgestellten Musterstudienplan zu adressieren. So liegt im Plan der Fakultät die Pflichtvorlesung \emph{Einführung in die IT-Sicherheit} im fünften Semster. Das führt dazu, dass sich das Studium bei Nichtbestehen des Moduls zwingend um ein Semester verlängert, da man die Bachelorarbeit erst nach Bestehen aller Pflichtmodule anmelden kann

Der zweite Musterstudienplan zieht einige Veranstaltungen nach vorne, sodass weniger Vorlesungen parallel zur Bachelorarbeit im sechten Semster liegen. Diesen Plan empfehlen wir dir allerdings nur, wenn du dir einen gewissen Mehraufwand gerade auch in den ersten Semstern zutraust. Weitere Informationen bekommst du auf dem Treffen zum Stundenplanbau nach dem Erstsemesterfrühstück oder bei einem Besuch im Fachgruppenraum der FG Informatik.

\tocheck{4}{Beschreibung des Musterstudienplans SoSe}

% So liegt im Plan der Fakultät z.\ B. ein Semester zwischen den Vorlesungen Programmieren I und Programmieren II. Wir halten es nicht für sinnvoll, aufeinander aufbauende Vorlesungen voneinander zu trennen und legen daher Programmieren II direkt ins 2. Semester. 

%Außerdem soll das Teamprojekt nach dem Musterstudienplan der Fakultät parallel zur Bachelorarbeit im 6. Semester gemacht werden. Da Projekte erfahrungsgemäß mit einem relativ hohen Arbeitsaufwand verbunden sind, empfehlen wir, das Projekt schon früher durchzuführen, damit man sich im 6. Semester voll auf die Bachelorarbeit konzentrieren kann. Weitere Informationen, oder Erfahrungen bekommt ihr auf dem Treffen zum Stundenplanbau, oder bei einem Besuch im Fachgruppenraum der FG Informatik.

% Der zweite Musterstudienplan gibt eine Orientierung, falls man das Fach Technische Informatik vorverlegen möchte. Der dritte Plan zeigt die Möglichkeit das Fach Theoretische Informatik vorzuverlegen. Bei beiden Plänen wurden zusätzliche Vorlesungen umgelegt, damit der Lernaufwand möglichst ausgeglichen bleibt. Theoretische Informatik und Technische Informatik bauen auf die Kenntnisse aus den Vorlesungen der ersten beiden Semester auf. Die Umlegung dieser Kurse nach vorne ist somit für diejenigen eine Möglichkeit, die schon vorher Kenntnisse gesammelt haben. Weitere Informationen, oder Erfahrungen bekommt ihr auf dem Treffen zum Stundenplanbau nach dem Erstsemesterfrühstück, oder bei einem Besuch im Fachgruppenraum der FG Informatik.

\nottoggle{winter}{Ein grundsätzliches Problem des Studienbeginns im Sommersemester ist, dass viele Fächer auf den Inhalten der Mathevorlesungen des Wintersemesters aufbauen. Je nach euren Vorkenntnissen kann es daher sinnvoll sein, stattdessen andere Fächer zu belegen. Da eine pauschale Aussage aber nicht möglich ist, sondern viel vom Einzelfall abhängt, führen wir mit euch eine Veranstaltung zur Studienplanung durch. Diese solltest du nicht verpassen!}{}

Du bist nicht mehr in der Schule, du hast nun Freiheiten. Nutze sie weise und studiere so, wie du es für richtig hältst!
