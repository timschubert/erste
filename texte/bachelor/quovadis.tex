%\newpage
%\end{multicols}\begin{multicols}{2}
\subsection{Quo vadis studens?}

Nun also ein Studium soll es sein, nur wie geht das eigentlich, studieren?
Das wichtigste hast du schon geschafft, wenn du diese Zeilen liest, nämlich die Einschreibung in dein gewähltes Studienfach.
Um ebendieses abzuschließen, macht dir die TU klare Vorgaben was du zu studieren hast, nicht allerdings wann und wie. Was heißt das für dich?
Fangen wir von hinten an:
Wie du lernst, studierst, lebst; ob du brav mitschreibst oder öfter mal ausschläfst kannst und musst du selbst entscheiden.
Wann du die vorgeschriebenen Lehrveranstaltungen belegst, liegt ebenfalls in deinem eigenen Ermessen, allerdings:
Nachdem, bis auf vier Ausnahmen, klar festgelegt ist was du studieren musst, ergibt sich eine sinnvolle Reihenfolge, da beispielsweise fortgeschrittenes Programmieren ohne Kenntnis von Algorithmen schlicht nicht möglich ist. Nichtsdestotrotz hast du Spielraum, das Studium an deine persönliche Situation anzupassen.
Du wohnst noch zu Hause und brauchst nicht arbeiten? Prima, mach noch Theoretische Informatik I im 1. Semester dazu.
Du hast ein Kind und musst nebenbei auch noch arbeiten? Kein Problem, sprich dich mit deinem Mentor ab und mach ein Teilzeitstudium.
Die konkreten Vorschriften zum Studium findest du in der Prüfungsordnung
%Link einfügen
.
In kurz: Grundsätzlich musst du Veranstaltungen im Wert von 180 Leistungspunkte (LP) erfolgreich absolvieren, davon 116–121 LP im Bereich Informatik, 35 LP in Mathematik, 14–19 LP für dein Nebenfach und 10 LP für Schlüsselqualifikationen.

Um dir einen sinnvollen Weg durchs Studium zu ermöglichen, gibt es von der Fakultät den „Musterstudienplan“, der versucht Überschneidungen der Veranstaltungen zu vermeiden.
Es gibt aber auch noch einen Alternativstudienplan der Fachgruppe, diesen empfehlen wir dir allerdings nur, wenn du dir den geringen Mehraufwand pro Semester zutraust. Dafür wirst du es sehr genießen, während des SEPs und der Bachelorarbeit nicht so viele Vorlesungen zu haben, die dir deine dann onehin knappe Zeit rauben.
Wir halten natürlich diesen alternativen Studienplan für ausgewogener und studierendenfreundlicher als den der Fakultät, aber auch er ist nur eine Empfehlung.
Ihr seid nicht mehr in der Schule, ihr habt nun Freiheiten, nutzt sie weise und studiert so, wie ihr es für richtig haltet.



%\end{multicols}
%\begin{wrapfigure}{l}{\linewidth}
%    \begin{center}
%          \includegraphics[width=\linewidth]
%	  {bilder/comics/dilbert.png}    \end{center}
%	\end{wrapfigure}
%	\begin{multicols}{2}

%%% Local Variables: 
%%% mode: latex
%%% TeX-master: "../../1-te"
%%% End: 
