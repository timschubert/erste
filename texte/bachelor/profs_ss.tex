\begin{multicols}{2}
\subsection{Eure Veranstaltungen im ersten Bachelor-Semester}
	Um euch einen kleinen Vorgeschmack auf die Leute zu geben, die euch im ersten Semester mit ihrem Wissen beglücken wollen, seien sie hier kurz aufgeführt:

\subsubsection{Einführung in die Logik}
	\textit{Prof. Jiri Adamek}
	Die Vorlesung behandelt die Grundlagen der formalen Logik, mit einen starken Fokus auf Aussagen- und Prädikatenlogik. Die Hausaufgaben sind dabei teilweise sehr zeitaufwändig, aber dafür eine gute Klausurvorbereitung. Dabei ist das Skript sehr hilfreich.

\subsubsection{Programmieren 2}
	\textit{Dr. Werner Struckmann}
	In Programmieren 1 (jährlich im Wintersemester) geht es um grundlegende Konzepte der Programmierung am Beispiel von Java. Darauf aufbauend wird in Programmieren 2 (jährlich im Sommersemester) die die Implementierung von Algorithmen und Datenstrukturen geübt. Hat man schon Vorkenntnisse in einer Programmiersprache (etwa durch eine Ausbildung zum Fachinformatiker o.Ä.) kann man sich durchaus auch schon im 1. Semester an Programmieren 2 versuchen, die Reihenfolge ist NICHT vorgeschrieben.

\subsubsection{Analysis}
	\textit{Dr. Wolfgang Marten}
	Hier geht es um Differential- und Integral- rechnung, sowie Grenzwerte. Die Übungen sind zwar nicht immer einfach, geben aber einen sehr guten Ausblick auf die Klausur.

\subsubsection{Computernetze 1}
	\textit{Prof. Lars Wolf}
	Hier lernt man die grundlegende Funktionsweise von Netzwerken kennen. Für die Klausur sollte man auf gar keinen Fall die Übungen verpassen. Interessiert man sich über die Vorlesung hinaus für das Thema, sollte man in die Bücher von Andrew S. Tanenbaum schauen.

\end{multicols}
