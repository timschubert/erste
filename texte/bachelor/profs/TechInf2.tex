% !TEX root = ../../../1-te.tex

\subsubsection{Technische Informatik 2}
	\textit{Prof. Rolf Ernst}
	Das Modul besteht aus Technischer Informatik 1 und 2 und endet
	mit einer Kombiklausur aus beiden Veranstatungen, die
	Reihenfolge in der man diese besucht ist egal, da die
	Inhalte relativ unabhängig voneinander sind.
	Die Vorlesung zu Technischen Informatik 2 orientiert sich
	weitgehend an dem Lehrbuch \enquote{Logic and Computer Design
	Fundamentals – 4th edition} von M. Mano und Ch. Kime, welches
	gleichzeitig als Skript gilt. Das Buch findet man in
	ausreichender Anzahl in der UB. Die kleinen Übungen sind als
	Klausurvorbereitung zu empfehlen, ersetzen aber nicht das eigene
	Nacharbeiten. Wichtig: Belegt man Technische Informatik 2 schon
	im 1. Semester, ist dies nur sinnvoll, wenn man auch Technische
	Informatik 1 im 2. Semester hört, um die Kombiklausur
	mitzuschreiben. Man sollte sich also gut überlegen, wie weit man
	sich da schon festlegen möchte! Andererseits hat man dann im
	zweiten Semester schon ein relativ dickes Modul abgeschlossen,
	was ja auch nicht verkehrt ist.