\subsubsection{Mathewahlpflicht}
%	\textit{N.N.} 
Ihr müsst insgesamt zwei Module zu je fünf Credits
	im Mathe-Wahlpflichtbereich einbringen. Dabei wird eine Vorlesung im Wintersemester und zwei
	Vorlesungen im Sommersemester angeboten:
	\begin{itemize}
	  \item Sommersemester: 
	    \begin{itemize} 
	      \item Algebra für Informatiker: Hier gehts um grundlegende
		algebraische Strukturen (Mengen, Gruppen, Monide etc). Diese sind insbesondere für die
		theoretische Informatik von großer Bedeutung.
	      \item Einführung in die Stochastik für Informatiker: Die
		Vorlesung behandelt die Grundlagen der
		Wahrscheinlichkeitstheorie (Laplace-Experimente,
		Erwartungswerte, Zufallsvariablen etc.). 
	    \end{itemize}
	  \item Wintersemester: 
	    \begin{itemize}
	      \item Einführung in die Numerik für Informatiker: Hier
		werden Verfahren zum Lösen numerischer Probleme
		behandelt. 
%	      \item Statistische Verfahren für Informatiker: Hier geht
%		es um statistische Probleme und wie man sie lösen kann.
%		Achtung: Die Vorlesung baut auf  ,,Einführung in
%		die Stochastik'' auf und setzt deren Inhalte voraus! Sie wird jedoch derzeit nicht angeboten.
	    \end{itemize}
	\end{itemize}
	Bei der Auswahl geht ihr am Besten so vor, dass ihr euch erstmal
	in alle gerade angebotenen reinsetzt und dann die behaltet, mit der ihr besser
	klarkommt. Generell gilt aber bei mathematischen Vorlesungen: Es
	gibt im Allgemeinen kein aktuelles Skript, wer nichts verpassen
	will, muss in der Vorlesung mitschreiben. Auch können die
	Hausaufgaben gerne mal umfangreicher werden, bereiten aber dafür
	sehr gut auf die Klausur vor. Dranbleiben und sich nicht
	entmutigen lassen ist alles :)