\subsection{Studienplan}
\label{bach_studienplan}
Wie ihr wahrscheinlich bereits in eurem Stundenplan festgestellt habt, m"usst ihr im ersten Semester f"unf "`Pflichtveranstaltungen'' h"oren.
Doch die Bezeichnung Pflichtverantstaltung sagt blo"s aus, dass ihr die Veranstaltung \emph{irgendwann} einmal h"oren m"usst, um euren Bachelor abschlie"sen zu k"onnen.
Die zeitliche Abfolge der Veranstaltungen d"urft ihr aber selbst festlegen.
Der von Frau Sehnert bereit gestellte Musterstudienplan (s. n"achste Seite) bietet hier eine gute Orientierungsm"oglcihkeit.
Ihr m"usst euch aber nicht daran halten. Niemand zwingt euch eine Veranstaltung zu h"oren oder h"alt euch davon ab.
Ihr k"onnt euch eigentlich in jede Vorlesung setzen, auch ohne hinterher an der Pr"ufung teilnehmen zu m"ussen - allerdings gibts dann auch keine Punkte dafür.
Hier bieten sich zum Beispiel Module aus dem Wahlplichtbereich Informatik an, die eventuell nur alle 2 Jahre angeboten werden und "uber mehrere Semester gehen.
Bei den (Pflicht-)Modulen der Informatik m"usst ihr jedoch beachten, dass einige Module auf anderen aufbauen.
Zum Beispiel sollten Programmierengrundlagen in den ersten zwei Semestern erarbeitet werden und mit Theoretische Informatik II werdet ihr euch schwer tun, wenn ihr TheoInf I nicht geh"ort habt.

Damit sich euer Studium nicht unn"otig verl"angert, solltet ihr aber darauf achten, in jedem Semester 30 Leistungspunkte zu erwerben. 


% Wie ihr wahrscheinlich bereits in eurem Stundenplan festgestellt habt,
% m"usst ihr im ersten Semester drei "`Pflichtveranstaltungen'' hören.  
% Doch die Bezeichnung Pflichtverantstaltung sagt blo"s aus, dass ihr die Veranstaltung \emph{irgendwann} einmal h"oren m"usst, um euren Bachelor abschlie"sen zu k"onnen.
% Die zeitliche Abfolge der Veranstaltungen d"urft ihr aber selbst festlegen.

% Der von Frau Sehnert bereit gestellte Musterstudienplan (s. Seite \pageref{musterstudienplan}) bietet hier eine gute Orientierungsm"oglcihkeit.
% Ihr m"usst euch aber nicht daran halten. Niemand zwingt euch eine Veranstaltung zu h"oren oder h"alt euch davon ab.
% Ihr k"onnt euch eigentlich in jede Vorlesung setzen, auch ohne
% hinterher an der Pr"ufung teilnehmen zu m"ussen - allerdings gibts
% dann auch keine Punkte dafür. 

% Hier bieten sich zum Beispiel Module aus dem Wahlplichtbereich Informatik an, die eventuell nur alle 2 Jahre angeboten werden und "uber mehrere Semester gehen.

% Bei den (Pflicht-)Modulen der Informatik m"usst ihr jedoch beachten, dass einige Module auf anderen aufbauen.
% Zum Beispiel sollten Programmierengrundlagen in den ersten  Semestern
% erarbeitet werden und mit Theoretische Informatik II werdet ihr euch
% schwer tun, wenn ihr TheoInf I nicht geh"ort habt.
% Je nach Vorkenntnissen kann es also sein,
% dass ihr mit einer anderen Reihenfolge besser bedient seid. 

% Auch kann
% es sinnvoll sein, sich einige Semester voller zu packen, um dafür in
% anderen Semestern (etwa mit dem Software-Entwicklungs-Praktikum oder
% der Bachelorarbeit) mehr Luft zu haben. 

% Da ihr außerdem bereits im
% 1. Semester euch zwischen zwei Mathe-Wahlplfichtfächern entscheiden
% müsst, ist es definitiv sinnvoll, sich selbst einen eigenen
% Studienplan zu basteln. 

% Wie das aussehen kann zeigt unser alternativer
% Plan auf Seite \pageref{studienplan_neu}.
% Ihr werdet bemerken, dass
% dort das 2. Semester recht voll gepackt ist im Vergleich zum
% offiziellen Plan, während dafür die letzten Semester eher dünn besetzt
% sind. Wir haben uns gedacht, dass es für die Studierenden angenehmer
% ist, sich die hinteren Semester zu entlasten. Andererseits ist es
% nicht so schlimm, am Anfang die eine oder andere Vorlesung
% wegzulassen, da man dann noch genug Zeit zum Nachholen hat, ohne das
% gesamte Studium zu verzögern. Am Besten überlegt ihr euch selbst immer,
% welche Vorlesungen ihr machen wollt (können auch gerne ein paar mehr
% als nötig sein) und lasst dann welche weg, wenn etwas nicht ganz
% klappt. Dazu haben wir auch den Erfahrungsbericht eines Studis auf
% Seite \pageref{studienplan_bericht}. 
% %Grundsätzlich solltet ihr euch nie zu 100 \% an einen Plan
% %halten, sondern euch einen eigenen basteln, der euch am Besten
% %entgegen kommt.
% \\
% Damit sich euer Studium nicht unn"otig verl"angert, solltet ihr aber darauf achten, in jedem Semester in etwa 30 Leistungspunkte zu erwerben. 


%%% Local Variables: 
%%% mode: latex
%%% TeX-master: "../../1-te"
%%% End: 
