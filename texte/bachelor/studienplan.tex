% !TEX root = ../../1-te.tex

\subsection{Studienplan}
	\label{bach_studienplan}
	\tocheck{3}{Stimmt die Anzahl Pflichtveranstaltungen noch? Mit aktuellem Stundenplan abgleichen!}
	Wie ihr wahrscheinlich bereits in eurem Stundenplan festgestellt habt, müsst ihr im ersten Semester \iftoggle{winter}{vier}{fünf} Pflichtveranstaltungen hören. Doch die Bezeichnung Pflichtverantstaltung sagt bloß aus, dass ihr die Veranstaltung \emph{irgendwann} einmal hören müsst, um euren Bachelor abzuschließen. Die zeitliche Abfolge der Veranstaltungen dürft ihr aber selbst festlegen. Der Fakultäts-Musterstudienplan bietet hier eine gute Orientierungsmöglichkeit. Ihr müsst euch aber nicht daran halten. Niemand zwingt euch eine Veranstaltung zu hören oder hält euch davon ab. Ihr könnt euch eigentlich in jede Vorlesung setzen, auch ohne hinterher an der Prüfung teilnehmen zu müssen -- allerdings gibt es dann auch keine Punkte dafür. Hier bieten sich zum Beispiel Module aus dem Wahlplichtbereich Informatik an, die eventuell nur alle 2 Jahre angeboten werden und über mehrere Semester gehen. Bei den (Pflicht-)Modulen der Informatik müsst ihr jedoch beachten, dass einige Module auf anderen aufbauen. Zum Beispiel sollten Programmiergrundlagen in den ersten zwei Semestern erarbeitet werden und mit Theoretische Informatik II werdet ihr euch schwer tun, wenn ihr TheoInf I nicht gehört habt.

	Damit sich euer Studium nicht unnötig verlängert, solltet ihr darauf achten, in jedem Semester rund 30 Leistungspunkte zu erwerben. 
