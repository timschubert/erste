\begin{multicols}{2}
\subsection{Studienplan}
	\label{bach_studienplan}
	Wie ihr wahrscheinlich bereits in eurem Stundenplan festgestellt habt, müsst ihr im ersten Semester fünf Pflichtveranstaltungen hören. Doch die Bezeichnung Pflichtverantstaltung sagt bloß aus, dass ihr die Veranstaltung \emph{irgendwann} einmal hören müsst, um euren Bachelor abschließen zu können. Die zeitliche Abfolge der Veranstaltungen dürft ihr aber selbst festlegen. Der von Frau Sehnert bereit gestellte Musterstudienplan (s. nächste Seite) bietet hier eine gute Orientierungsmöglcihkeit. Ihr müsst euch aber nicht daran halten. Niemand zwingt euch eine Veranstaltung zu hören oder hält euch davon ab. Ihr könnt euch eigentlich in jede Vorlesung setzen, auch ohne hinterher an der Prüfung teilnehmen zu müssen - allerdings gibts dann auch keine Punkte dafür. Hier bieten sich zum Beispiel Module aus dem Wahlplichtbereich Informatik an, die eventuell nur alle 2 Jahre angeboten werden und über mehrere Semester gehen. Bei den (Pflicht-)Modulen der Informatik müsst ihr jedoch beachten, dass einige Module auf anderen aufbauen. Zum Beispiel sollten Programmierengrundlagen in den ersten zwei Semestern erarbeitet werden und mit Theoretische Informatik II werdet ihr euch schwer tun, wenn ihr TheoInf I nicht gehört habt.

	Damit sich euer Studium nicht unnötig verlängert, solltet ihr aber darauf achten, in jedem Semester 30 Leistungspunkte zu erwerben. 
\end{multicols}