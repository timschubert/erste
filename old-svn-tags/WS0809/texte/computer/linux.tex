\subsection{Linux-Bezug an der TU-BS}


Das Betriebssystem Linux ist den meisten inzwischen ein Begriff. Dass man Linux 
jedoch kostenlos an verschiedenen Stellen der TU-BS beziehen kann, ist leider 
weniger bekannt.

F"ur Studierende ohne breitbandigen Netzzugang sind sicherlich die CDs 
n"utzlich, die sich jede/r im
IT Service-Desk\nurlfootnote{http://www.tu-braunschweig.de/it/service-desk}
im Gauß-IT-Zentrum, \nroom{Raum 017}, ausleihen kann. Dort stehen eigentlich
immer die neusten Versionen von SuSE, Mandrake, Fedora, Gentoo, Debian und Knoppix
sowie diverse "altere Distributionen zur Verf"ugung. Dank eines DVD-Brenners
k"onnen inzwischen auch --~soweit vorhanden (SuSE, Knoppix)~-- die
DVD-Versionen verliehen werden. Auf der sicheren Seite ist, wer vorher einen
Abholtermin vereinbart, damit die gew"unschte Distribution garantiert greifbar
ist: 0531/391-5555.

F"ur Studierende mit Breitband-Internetzugang sind vermutlich die diversen 
Mirror-Server interessant. Auch hier stehen wieder die gr"o"seren 
Distributionen im Uni-Netz bereit:
	  
\begin{description}
\item[\nurl{ftp://ftp.rz.tu-bs.de/}]~\\Enth"alt Openoffice-, Mozilla-,
Gentoo-, Slackware- und Ubuntumirror, CCC Vortr"age
\item[\nurl{ftp://debian.tu-bs.de/}]~\\Debian-, Kanotix- und Knoppixmirror
\item[\nurl{ftp://ftp.ibr.cs.tu-bs.de/}]~\\Mehr CCC Vortr"age, diverse freie
Software (gr"o"stenteils f"ur Unix/Linux)
\end{description}


