\subsection{Microsoft Academic Alliance}

Die TU hat seit 2003 eine
Campuslizenz\nurlfootnote{http://msdn.microsoft.com/en-us/default.aspx}
von Microsoft erworben, in deren Rahmen du Microsoftprodukte kostenlos
beziehen kannst.\\
Zur Auswahl stehen die meisten Betriebssysteme, Entwicklungswerkzeuge und
diverse Serversoftware\footnote{\sloppy Eine komplette Liste der Software
findet sich unter \nurl{http://msdn.microsoft.com/en-us/subscriptions/downloads/default.aspx}}.
Die Office-Suite ist explizit \textbf{nicht} enthalten, kann aber separat
gekauft werden.\footnote{Mehr Infos unter
\nurl{http://www.tu-bs.de/it/services/software/lizenz/vertraege/ms-select/}}\\
Die Software darf zu nicht-kommerziellen Zwecken in Forschung und Lehre
eingesetzt werden, jedoch keine Infrastrukturaufgaben
erf"ullen\footnote{Die Nutzungsbedingungen sind nachzulesen unter
\nurl{http://msdn.microsoft.com/en-us/academic/bb250609.aspx}}.
Infos gibt es unter
\nurl{http://www.tu-bs.de/it/services/software/lizenz/vertraege/msdn-aa/}.
Microsoft Windows XP kannst du bei den Operateuren im Rechenzentrum in
\nroom{Raum 015}
f"ur eine Schutzgeb"uhr von \EUR{5} erwerben, die "ubrige Software kannst du
dort ausleihen oder unter
\nurl{http://www.tu-bs.de/it/services/software/doku/msdn-aa/}
downloaden.