\subsection{Interview mit Dr. Struckmann}

\nquestion{Welche Vorlesungen halten Sie?}

Vorlesungen, die mit Programmierung in Theorie und Praxis zu tun haben.
Das sind \emph{Programmieren 1} und \emph{Programmieren 2} sowie \emph{Programmieren f"ur Fortgeschrittene}, in der vertiefende Aspekte der Programmierausbildung behandelt werden.
Dies sind die Vorlesungen f"ur Bachelor-Studierende.
F"ur Master-Studierende halte ich unter anderem die Vorlesungen \emph{Semantik von Programmiersprachen}, in der Methoden zur Definition der Syntax und Semantik von Programmiersprachen vorgestellt werden.

\nquestion{Wie sind Sie zur Informatik gekommen?}

Ich habe Mathematik mit dem Nebenfach Informatik studiert.
An der Informatik interessiert mich unter anderem besonders die Theoretische Informatik, in der es grundlegend um die L"osbarkeit von Problemen durch die Informatik geht.
Nat"urlich interessiert mich aber auch die praktische Seite, also eigentlich interessiert mich alles was das weite Feld der Informatik betrifft.

\nquestion{Beschreiben Sie den Stereotyp "`Informatiker"'?}

Den Stereotyp "`Informatiker"' kenne ich nicht.
Ich denke, dass es solche Informatiker nicht gibt.
Dies mag einfach daran liegen, dass die Informatik so viele Themenbereiche abdeckt, von der praktischen bis zur theoretischen Seite.
Vom Praktiker bis zum Theoretiker sind alle Typen in der Informatik vereint.

\nquestion{Was ist f"ur Sie ein richtiges Studentenleben?}

Fr"uher htte ich als Antwort gegeben, dass zu einem richtigen Studentenleben gehrt, die Semesterferien zu genie"sen und in der Vorlesungszeit hart zu arbeiten.
Ich war ich in den Semesterferien Bademeister und habe mir ein wenig Geld dazu verdient.
Heutzutage aber bleibt ja nicht mehr viel Zeit in der "`vorlesungsfreien Zeit"', da man alle zwei Wochen eine Klausur schreiben muss, auf die man sich vorzubereiten hat.
F"ur Studienanf"anger bedeutet dies eine erhebliche Belastung.
Als Studienberater kann ich nicht guten Gewissens dazu raten, nebenbei zu arbeiten.
Durch die Studienberatung habe ich Kontakt zu vielen Studierenden, die mir von ihren Schwierigkeiten berichten, das Studium zu finanzieren.
Ich kann jetzt nat"urlich aufgrund meiner Schweigepflicht keine Beispiele nennen, aber das Finanzierungsproblem ist Realitt.

\nquestion{Erz"ahlen Sie eine lustige Anekdote aus Ihrem Studium!}

W"ahrend meiner Studienzeit befand sich das Rechenzentrum (jetzt: Gau"s-IT-Zentrum) noch im Altgeb"aude in der ersten Etage.
Irgendwann sollte der Speicher des Gro"srechners um einige hundert Kilo-Byte erweitert werden.
Daf"ur musste ein Fenster ausgebaut werden.
Der Speicher wurde dann mit einem Kran in das erste Stockwerk gehoben.

\nquestion{Was k"onnen Sie den Studierenden f"ur das erste Semester mit auf den Weg geben?}

Im Studium spielt das eigenverantwortliche Lernen eine wichtige Rolle.
Man muss selbst einschtzen k"onnen, ob man einen Sachverhalt wirklich verstanden hat.
Ggf. muss man beim Lehrenden oder Hiwi nachfragen.
Zu Programmieren 1 und 2 finden beispielsweise die "Ubungen am Rechner w"ochentlich statt, w"ahrend Hausaufgaben nur alle zwei Wochen abgegeben werden m"ussen.
Beim ersten Termin kommen meist nur wenige Studierende.
Wenn die Abgabe der Hausaufgabe aber n"aher r"uckt, treffen viele Studierende ein und versuchen auf den letzten Dr"ucker die Aufgaben zu bearbeiten.
Hier empfehle ich konkret, w"ochentlich an den Rechner"ubungen teilzunehmen und
den Hiwis auch Fragen zu stellen, die nicht direkt mit der Bearbeitung der Hausaufgaben zu tun haben.
Meine Empfehlung f"ur Studierende im ersten Semester lautet also: 
Nehmen Sie regelm"a"sig an den Veranstaltungen teil und z"ogern Sie nicht, den "Ubungsleitern Fragen zu stellen.
