\label{elekinf}\subsection{Elektronisch informiert}

Die wichtigsten Aufgaben der Studierenden sind der Besuch von
Lehrveranstaltungen, Zeitmanagement f"ur Studium und Freizeit und
Informationsbeschaffung. In diesem Artikel geht es um den letzten Punkt, und da
wir nun mal Informatik studieren, soll die Informationsbeschaffung "uber das
Internet erfolgen.

\begin{figure}[h]
  \centering\includegraphics%[width = \linewidth]
  {bilder/comics/stein1.png}
\end{figure}

\subsubsection*{Adressen im World Wide Web}

Die Internet-Seiten der TU sind so riesig (oder un"ubersichtlich), dass man sie
auf den ersten Blick gar nicht begreifen kann. Ein paar Seiten der TU und
etliche Links aus dem restlichen WWW seien hier genannt.

 \begin{description}
 \item[TU-Homepage]~\\\nurl{http://tu-braunschweig.de/}
 \item[Hauptseite der Informatik]~\\\nurl{http://www.cs.tu-bs.de/}
 \item[Fachgruppenrat Informatik]~\\\nurl{http://fginfo.cs.tu-bs.de/}
 \item[Fachbereichssekretariat / Pr"ufungsamt]~\\
 \nurl{http://tu-braunschweig.de/fk1/service/informatik}
 \item[Mensa-Speisepl"ane]~\\
 \nurl{http://sw-bs.de/braunschweig/essen/}
 \item[BAf"oG-Amt]~\\\nurl{http://sw-bs.de/braunschweig/finanzen/}
 \item[Gau"s-IT-Zentrum]~\\\nurl{http://tu-braunschweig.de/it}
 \item[Unisport]~\\\nurl{http://www.unisport.tu-bs.de/}
 \item[Sprachenzentrum]~\\\nurl{http://www.sz.tu-bs.de/}
 \item[Immatrikulationsamt]~\\\nurl{http://tu-braunschweig.de/i-amt/}
 \item[Stadt Braunschweig]~\\\nurl{http://braunschweig.de/}
 \item[Stadtplan f"ur Braunschweig]~\\\nurl{http://stadtplan.braunschweig.de/}
 \item[Campus-BS.de Portal]~\\\nurl{http://www.campus-bs.de/}
 \item[Stadtmagazine]~\\\nurl{http://www.subway-net.de/}
 \item[Schwimmb"ader in Braunschweig]~\\\nurl{http://www.stadtbad-bs.de/}
 \item[Kennelbad (Freibad \& Open-Air-Kino)]~\\\nurl{http://www.kennel-bad.de/}
 \item[Kinos]~\\\nurl{http://www.cinemaxx.de/}\\
   \nurl{http://www.bs-net.de/kino/}\\
   \nurl{http://www.schunterkino.de/}
 \item[MonkeyIsland]~\\\small{\nurl{http://gruppen.tu-bs.de/monkeyisland/}}
 \item[Schuntille]~\\\nurl{http://www.schuntille.de/}
 \item[Michaelishof]~\\\nurl{http://www.michaelishof.de/kneipe/}
 \item[Atelco, Karrenf"uhrerstr. 1-3]~\\\nurl{http://www.atelco.de/}
 \item[EGA.Com, Bohlweg 55]~\\\nurl{http://www.egacom.de/}
 \item[Kosatec, Kleine Burg 14]~\\\nurl{http://www.kosatec.de/}
 \item[SHV-Computer, B"ultenweg 81]~\\\nurl{http://www.shv-computer.de/}
 \item[Skycom, Gifhorner Stra"se 148]~\\\nurl{http://www.skycompc.de/}
 \item[Vobis, Otto-von-Guericke-Stra"se 2]~\\\nurl{http://www.vobis.de/}
 \item[Art of Systems, Wendenstrasse 58]~\\\nurl{http://www.artsys.de/}
 \end{description}

\subsubsection*{Mailinglisten}

Die wichtigste Mailingliste f"ur Informatikstudierende ist die Liste
\textbf{cs-studs}. Da bei den Wirtschaftsinformatikern oftmals auch
informatikrelevante Themen diskutiert werden, lohnt sich m"oglicherweise auch
ein Blick in \textbf{winfo-studs}. Wenn ihr an Stellenangeboten und Werbung aus
der freien Wirtschaft interessiert seid, steht die Mailingliste
\textbf{firmenkontakt} zu eurer Verf"ugung. Die Informatik-Kolloquien, das sind
Vortr"age von "ublicherweise externen Referenten zu Informatik-Themen, werden
auf der Mailingliste \textbf{kolloq} angek"undigt. Alle bisher genannten
Mailinglisten sind "uber \nurl{http://www.cs.tu-bs.de/mailinglisten.html}
erreichbar.

% \subsubsection*{Newsgroups}
% leider komplett tod
% Die auf dem Newsserver \nurl{news://news.tu-bs.de} (nur aus dem TU-Netz heraus
% erreichbar) liegenden Newsgroups werden leider nur sehr m"a"sig genutzt.
% Mitteilungen des Rechenzentrums kommen "uber \textbf{tubs.general},
% Mitteilungen speziell f"ur WLAN-Nutzer auf \textbf{tubs.wlan.d}. Die Newsgroups
% \textbf{tubs.studium} und \textbf{tubs.studium-informatik} k"onnte man mal
% wiederbeleben. Sehr reger Betrieb herrscht auf
% \textbf{braunschweig.allgemeines} und der f"ur Schn"appchenj"ager idealen
% \textbf{braunschweig.kaufrausch}.

\subsubsection*{IRC}

Im Freenode IRC (\nurl{http://freenode.net}) gibt es den Channel \nurl{##cs-studs}. Hier
sind immer ein paar BraunschweigerInnen und gro"sse Teile des Fachgruppe online. Die Gespr"achsthemen haben (im weitesten
Sinne ;) mit dem Studium zu tun.

\subsubsection*{Clevershit}

Auf jeden Fall einen Besuch wert und eine gute Hilfe bei allem, was das Studium betrifft, ist das von Studenten im letzten Jahr ins Leben gerufene Portal \mbox{\nurl{http://www.clevershit.de}}.\\
Diese von Studenten f"ur Studenten erstellte und gef"uhrte Plattform bietet eine gute Anlaufstelle f"ur Fragen jeglicher Art. In der Wiki der Seite gibt es eine Materialsammlung zu allen Veranstaltungen der ersten Semester. Im gut besuchten Forum werden stets aktuelle Informationen und "Anderungen zu den Vorlesungen weiter gegeben, Hausaufgaben und Klausuren diskutiert oder einfach etwas Smalltalk gehalten.

\subsubsection*{Sag's uns}
\emph{Sag's uns} ist ein Blog, der im Auftrag des Pr"asidiums und in Kooperation mit dem Institut f"ur Wirtschaftsinformatik, insbesondere Informationsmanagement, von Studierenden f"ur Studierende entwickelt wurde und Anfang des Jahres 2009 an den Start ging.

Eure Ideen, Kritiken, Anregungen und Belobigungen werden hier transparent von zentraler Stelle aus schnellstm"oglich bearbeitet und moderiert, die zust"andigen Einrichtungen der TU werden zur Absprache und Beantwortung einbezogen. 
Wir freuen uns "uber unsere Studierenden und Mitarbeiter, die dazu beitragen, dass der Blog sehr effizient und zielf"uhrend eingesetzt und auch angenommen wird.

Gute Ideen und L"osungen werden bei \emph{Sag's uns} ver"offentlicht, so dass auch andere davon profitieren k"onnen.
Dar"uber hinaus kannst du die Ideen anderer bewerten – und umgekehrt. Nat"urlich kannst du auch Anfragen stellen, ohne dass sie ver"offentlicht werden, ebenso wie du anonym bleiben kannst, wenn dir das lieber ist.

Unter \nurl{http://tu-braunschweig.de/sagsuns} kannst du loswerden, was dir an der Universit"at wichtig ist.

Noch Fragen? Die Referentin f"ur Studienqualit"at, Frau Dipl.-P"ad. Anja Reisch, steht Dir als Ansprechpartnerin in der Gesch"aftsstelle des Pr"asidiums zur Verf"ugung (sagsuns@tu-braunschweig.de; 0531 391 4109).

\begin{description}
\item[Allgemeines Vorlesungsverzeichnis:] ~\\
{\footnotesize\url{http://vorlesungen.tu-bs.de}}
\item[Uni-Bibliothek:] ~\\
{\footnotesize\url{http://www.biblio.tu-bs.de}}
\item[Druckkosten:] ~\\
{\footnotesize\url{http://www.tu-braunschweig.de/it/services/drucken/kosten}}
\item[Don't Panic online] ~\\
{\footnotesize\url{http://www.tu-braunschweig.de/Medien-DB/it/dontpanic.pdf}}
\item[Putty Homepage] ~\\
{\footnotesize\url{http://www.putty.org}}
\item[WinSCP Homepage] ~\\
{\footnotesize\url{http://winscp.net}}
\end{description}
%%% Local Variables: 
%%% mode: latex
%%% TeX-master: "../../1-te"
%%% End: 
