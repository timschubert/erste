\end{multicols}
\newpage
\section{Computer und so\ldots}
\begin{multicols}{2}

"`Informatik hat viel mit Computern zu tun!"' - Diesem Irrglauben erliegen zu
Anfang des Studiums einige, auch wenn sich inzwischen öfter rumspricht, dass
das Studium abstrakter ist. Das Informatikstudium ist nicht dafür da, euch beizubringen, wie man einen
Computer bedient. Somit sind diese Seiten eventuell das erste und letzte Mal, 
dass euch Infos zu diesem Thema direkt vorgesetzt werden. Natürlich können wir
hier nur ein paar Tipps geben und euch darauf hinweisen, wo ihr mehr Infos 
findet.

In Wirklichkeit wirst du den Computer im Studium kaum mehr brauchen als ein
Student der Germanistik oder Sozialwissenschaften, denn die einzigen Inhalte, 
die man direkt am Rechner lernen und umsetzen muss, sind die Hausaufgaben, 
die in Programmieren aufgegeben werden.

Dennoch sind Computer ein unersetzliches Werkzeug um durchs Studium zu kommen.
Aber heißt dass nun, dass ihr um zu Studieren einen eigenen, top-aktuellen Rechner
braucht, und dass ihr absolute PC-Freaks sein solltet? Dem wollen wir auf den 
nächsten Seiten ein wenig auf den Grund gehen.

% was macht man mit dem PC
% wo gibts Uni-Rechner
% eigener Rechner
% gitz
% linux
% microsoft
% links

%\begin{multicols}{2}
	\subsection{Wozu Computer?}
		\subsubsection{Vorlesungen Online}
			Zu den meisten Vorlesungen kann man die Skripte im Internet finden. Für einige Vorlesungen gibt es sogar Ton- oder Videomitschnitte.

			Es gibt auch immer engagierte Studierende, die ihre Vorlesungsmitschriften online stellen. Da diese sehr wahrscheinlich in deinem Semester sind, hilft es, wenn du dich in den Vorlesungen umhörst. Ansonsten ist \url{https://www.clevershit.de} die richtige Anlaufstelle für den Informationsaustausch zwischen Studenten.

		\subsubsection{Organisatorisches ohne Papier}
			Ansonsten gibt es eine Reihe von Informationen, die man nur über das Web bekommt, und mehr und mehr Formalitäten (z.B. die Prüfungsanmeldung) werden auch in die virtuelle Relatität verlagert.

			Desweiteren kannst du dir im Netz einen individuellen Stundenplan zusammenstellen, in Erfahrung bringen, wann die nächsten Klausuren stattfinden, lesen, was es in der Mensa zu essen gibt, endlich herausfinden, wann das Prüfungsamt geöffnet hat,  offene HiWi"~Stellen bei den Instituten finden und vieles mehr.

			Die Webseiten der TU sind ein großer Dschungel, durch den man sich am besten  mit Machete und Googlesuche kämpft. Um an der TU etwas zu finden, solltest du deinem eigentlichen Suchbegriff wahlweise \enquote{tu braunschweig} oder \enquote{site:tu-braunschweig.de} anhängen, und schon hast du gute Chancen zum Ziel zu kommen.

		\subsubsection{Mitschreiben am PC}
			Auf den ersten Blick mag es naheliegen, sich
			während der Vorlesungen Notizen  am Laptop
			anzufertigen. In der Praxis gibt es da aber eine
			Reihe von Problemen, vor denen wir  warnen
			möchten. Es hat schließlich seinen Grund, das
			nur 10\% der Studenten in der Vorlesung am
			Laptop sitzen und davon 90\% diesen nur nutzen,
			um zu zocken: Die meisten Tafelanschriften
			bestehen  aus verschachtelten Formeln,
			fremdartigen Buchstaben und verworrenen
			Zeichnungen. Diese in Echtzeit in den Laptop
			einzuhacken ist eine besondere Kunst, die du mit
			Notepad und Word gar nicht erst probieren
			brauchst. Eine Chance hast du vielleicht mit
			einem Tablet PC, oder wenn du
			\LaTeX\ bereits im Schlaf beherrscht -
			aber wer tut das schon zu Beginn des Studiums?

			In den Vorlesungen, in denen du nicht Tafelweise abschreiben musst, sondern nur hier und da mal etwas notieren, zeigt sich der PC schon als nützlicher. Wenn du ab und zu den Vortrag des Profs damit vergleichen möchtest, was er in sein Skript geschrieben hat, kann dir der mitgebrachte Laptop unter Umständen das Ausdrucken von ein paar hundert Seiten ersparen. Du wirst aber schnell merken, dass es in praktisch keinem der Hörsääle und Seminarräume Steckdosen gibt, und in manchen nichtmals ausreichende WLAN-Signalstärke.

		\subsubsection{Hausaufgaben am PC}
			In vielen Fächern musst du regelmäßig
			Hausaufgaben erledigen und einreichen. Keiner
			erwartet von dir, dass diese mit dem PC gemacht
			werden, es ist also völlig ok sie von Hand zu
			schreiben. Es hat aber auch gewisse Vorteile,
			sie am PC zu schreiben (z.B. mittels \LaTeX) und
			dann auszudrucken. Bei \LaTeX\ handelt es sich
			um ein Satzsystem für wissenschaftliche
			Texte wie Haus- oder Abschlussarbeiten.
			Erwähnenstwert ist die hervorragende
			Unterstützung für den Satz mathematischer
			Formeln und dass dabei mit Befehlen, ähnlich wie
			in HTML gearbeitet wird. Es gibt \LaTeX-Kurse die du im Schlüsselqualifiktationsbereich anrechnen lassen kannst, aber mit den Infos im WWW kann man sich das auch selbst beibringen. Je eher du damit anfängst, desto weniger Probleme hast du später, wenn du damit z.B. deine Abschlussarbeit aufsetzt.

		\subsection{Computer-Pools an der Uni}
			Es ist immer nützlich zu wissen, wo man mal schnell an einen Computer kann. Zumindest ab und zu wirst du die Computer in der Uni benutzen, besonders die Linuxarbeitsplätze in \textbf{PK4.5} oder \textbf{PK4.8}, an denen du die Hausaufgaben für Programmieren abgeben musst.

			\begin{itemize}
				\item[*] Im Erdgeschoss des Altbaus gibt es auf der rechten Seite zwei Computerräume, einen weiter vorne (\textbf{PK4.6}) und einen genau in der Ecke des Gebäudes (\textbf{PK4.5}). Zwei weitere Räume (\textbf{PK4.8} und die \textbf{Datenstation}) findest du im ersten Stock des Altbaus, auch wieder in der rechten Ecke. Die Rechner in \textbf{PK4.5} und \textbf{PK4.8} sind mit Linux ausgestattet. Im ersten Stock gibt es nun auch einen Windowsrechnerraum. Da kann man mal eine Word- oder Powerpoint-Datei ausdrucken, wenn man denn muss.

				\item[*] Reichlich Computer findet man schließlich im Gauß-IT-Zentrum~(GITZ) an der Hans-Sommer-Straße. Das ist der gedrungene, fast würfelförmige, dunkle Klotz hinter dem Elektrotechnik-Hochhaus (\emph{E-Tower}). Hier gibt es mehrere frei zugängliche Räume mit  Linux- und Windowsrechnern. Es gibt hier auch Räume für Medienbearbeitung, wo du etwa Video-Digitalisierer, ein Tonstudio und Rechner mit der Adobe Creative Suite Production Premium nutzen kannst.

				\item[*] Seit 2010 stellt das IBR (Institut für Betriebssysteme und Rechnerverbund) im Raum G40 des Informatikzentrums einen Rechnerraum mit vielen, schnellen Linux-Rechnern  zur Verfügung. Zu diesem CIP-Pool (Computer-Investitions-Programm) bekommt man mit seiner y-Nummer Zutritt. Wenn man Glück hat, funktioniert sogar einer der beiden Drucker in diesem Raum, so dass man zum Drucken nicht das IZ verlassen muss.
			\end{itemize}

		\subsection{Der eigene Rechner}
			Wenn du trotz aller Widrigkeiten planst, dir
			extra für's Studium einen (tragbaren) Rechner
			anzuschaffen, dann hast du hier gleich ein wenig
			Kaufberatung: Viel (Rechen- bzw.
			Grafik-)Leistung brauchst du im Studium  nur für
			sehr wenige spezielle Fachgebiete - das
			einfachste Netbook wird also vermutlich schon
			reichen. Wichtiger ist vielmehr die Akkulaufzeit
			und die WLAN-Empfangsstärke. %a die Här	

		\subsubsection{Welches System?}
			Dir wird auffallen, dass zwar alle Systeme geduldet sind, aber die Linux hier deutlich öfter über den Weg laufen wird als in der freien Wildbahn. Auch wir sind große Linux-Fans und haben deshalb ab Seite \pageref{linux} ein paar Infos dazu zusammengetragen.

			Aber trotz dieser nicht ganz unauffälligen Beeinflussung gilt: Beim Betriebssystem hast du freie Wahl. Sämtliche Software, die du für's Studium brauchen  könntest, gibt es für alle großen Systeme, meist sogar gratis. Für Linux ist eh  praktisch alles frei erhältlich, für Windows spendiert Microsoft den Studenten auch alles außer Office (siehe Seite \pageref{msdnaa}), und auch Apple bringt dich dank satter Studentenrabatte durch Bachelor und Master. 

		\subsubsection{Wege ins Uni-Netz}
			Um den eigenen Rechner ins Netz zu bekommen, stehen die in der Uni WLAN und LAN offen. Zur Konfiguration siehe Seite \pageref{wlan}.

			Für manche Aktivitäten (z.B. den Zugriff auf Prüfungsprotokolle) musst du dich direkt im Uni-Netz befinden. Wenn du und dein Rechner aber gerade zuhause oder sonstwo sind, heißt dass nicht, dass du dich nun physisch auf den Weg machen musst. Mittels VPN kannst du dich virtuell ins Uni-Netz einklinken. Schau einfach mal auf den Seiten des GITZ nach, um mehr zu erfahren.
%\end{multicols}

\subsection*{\Large{Du bist computers"uchtig, wenn\ldots}}

\begin{enumerate}
\item \ldots du eine Viertelstunde brauchst, um durch deine Bookmarks zu scrollen.
\item \ldots du deinen Lautsprecher aufdrehst, bevor du das Zimmer verl"a"st, damit du das akustische Signal h"orst, wenn eine neue E-Mail eintrifft.
\item \ldots dein Hund eine eigene Homepage hat.
\item \ldots du deine Mutter nicht anrufen kannst, weil sie kein VoIP-Telefon hat.
\item \ldots du deine Mail abrufst, die Meldung kommt: "`No new messages"' - und du sie gleich nochmal abrufst.
\item \ldots du das Geschlecht von dreien deiner besten Freunde nicht kennst, weil sie neutrale Nicknames haben und du sie nie danach gefragt hast.
\item \ldots du morgens um 3 Uhr aufwachst, zum Klo gehst und auf dem R"uckweg am Computer halt machst, um deine Mailbox abzurufen.
\item \ldots du dich t"atowieren l"asst: "`Diesen K"orper betrachtet man am besten mit Mozilla 5.0 oder h"oher"'.
\item \ldots dein Partner sagt, dass das Gespr"ach f"ur eine Beziehung wichtig ist, also kaufst du einen zweiten Rechner und richtest ihm/ihr einen IRC-Client ein.
\item \ldots dir jemand einen Witz erz"ahlt und du mit *lol* antwortest.
\item \ldots du deinen Freunden von einer hei"sen Verabredung erz"ahlst und ihnen verschweigst, dass sie in einem Chatroom stattfindet.
\item \ldots du dir einen Laptop kaufst, um auch auf dem Klo surfen zu k"onnen.
\item \ldots du auf eine Webseite schaust, die voll mit Links von jemand anderem ist, und alle Links bereits in Lila erscheinen.
\item \ldots dich dein Provider bei technischen Schwierigkeiten um deine Hilfe bittet.
\item \ldots du bei \nurl{http://www.wetter.de/} nachschaust, anstatt aus dem Fenster.
\item \ldots Google bei dir anfragt, was noch in ihrer Suchmaschine fehlt.
\item \ldots du deinen Kopf zur Seite beugst, um zu l"acheln.
\item \ldots deine Kaffeemaschine eine eigene IP hat.
\item \ldots du versuchst Texte aus deinem handgeschriebenen Script per copy and paste in ein \LaTeX-Dokument einzuf"ugen.
\item \ldots du keine Kiste mit alten Computerteilen hast, weil z.B. der alte 386er noch als Anrufbeantorter genutzt wird.
\item \ldots du deine HiFi-Anlage "uber einen eigens daf"ur aufgesetzten Webserver steuerst.
\item \ldots du bei vier Webbrowserspielen unangefochten auf Platz 1 stehst.
\item \ldots du wei"st, was man unter\\ \nurl{http://www.google.de/search?&amp;q=5\%5E2\%2B23\%2D3\%21&amp;btnG=Suche&amp;meta=} findet.
\end{enumerate}

\begin{multicols}{2}
\subsection{Gauß-IT-Zentrum}

	Das Rechenzentrum der TU-Braunschweig heißt Gauß-IT-Zentrum oder kurz GITZ. Es bietet euch eine Vielzahl an Diensten an. Manche davon könnt ihr nur vor Ort nutzen, also in der Hans-Sommer-Str. 65, direkt hinter dem ,E-Tower'. Den wunderschönen ,braunen Würfel' findet Ihr z.B. im Standplan \url{http://stadtplan.braunschweig.de}.

	Andere Dienste sind auch in den Außenstellen, wie z.B. im Altgebäude zu finden, und das allermeiste könnt ihr über das Netz an der gesamten Uni oder sogar weltweit in Anspruch nehmen.

\subsection{GITZ-Account}
\label{todogitz}
	Unser Rechenzentrum, das Gauß-IT-Zentrum, stellt euch diverse Dienste zur Vefügung, wovon manche quasi lebenswichtig sind, andere eher nebensächlich. Aber für all diese Dienste braucht ihr eine GITZ-Account-Nummer und ein Passwort. Diese so genannte y-Nummer ist nicht das gleiche wie eure Immatrikulationsnummer. In der Regel bekommt ihr schon vor Semesterbeginn eine Nummer und ein vorläufiges Passwort per Post zugesendet. Dieses Passwort müsst ihr euch nicht mehrken, denn ihr braucht es nur einmal, nämlich um sich ein richtiges Passwort für die spätere Verwendung auszusuchen. Das solltet ihr auf jeden Fall möglichst früh von einem eigenen PC von zuhause aus machen (denn ohne das gemacht zu haben, stehen euch die Uni-PCs nicht zur Verfügung, und ihr kommt in der Uni auch noch nicht ins WLAN). Dann solltet ihr euch alle drei wichtigen Daten - Matrikelnummer, Y-Nummer und das neue Passwort gut einprägen (ihr braucht sie dann ständig zu den unmöglichsten Zeiten).

	Es kann auch passieren, dass ihr den besagten Brief vom GITZ  gar nicht bekommt, dann müsst ihr euch selbst um all das kümmern. Keine Sorge, das passiert halt ab und zu, ist aber nicht weiter schlimm.

	\subsubsection{Emailadresse}
		\label{todomailing}
		Zusammen mit eurem GITZ-Account bekommt ihr auch ein neues Email-Postfach mit bis zu drei Adressen (y00000000@tu-bs.de, vorname.nachname@tu-bs.de, v.nachname@tu-bs.de). Für die oben genannte Mailingliste, und diverse andere Zwecke, könnt ihr euch meist aussuchen, ob ihr eure vorherige private Emailadresse nutzt, oder die neue von der TU-Braunschweig. Aber egal wie ihr euch entscheidet, ab und zu erreichen euch auch Emails auf eurem TU-Braunschweig-Postfach, also schaut dort regelmäßig rein! Wer mit der TU-Mailadresse nichts zu tun haben möchte\footnote{Es gibt immer wieder mal technische Probleme damit, weshalb viele es bevorzugen, selbst für studienspezifische Dinge nicht die TU-Adresse zu verwenden.}, sollte sich zumindest eine Weiterleitung auf seine Hauptadresse einrichten.

	\subsubsection{IRC-Channel und Forum/Wiki}
		Viele Studierenden der Informatik, Nebenfachhörer und Fachgruppenmitglieder sind im IRC-Channel \texttt{\#\#cs-studs} (ja, der zweite ,,\#'' ist korrekt) auf \texttt{irc.freenode.net} unterwegs. Auch hier ist ein guter Ort, Fragen zu stellen.

		Unter \url{http://clevershit.de} findet ihr außerdem ein Forum und ein Wiki extra für Informatiker an der TU-Braunschweig, auf dem ihr Fragen stellen könnt und extrem viele nützliche Infos für's Studium findet. Um euch dort anzumelden, braucht ihr übrigens die TU-Braunschweig-Emailadresse, die ihr vom GITZ bekommt.

	\subsubsection{Drucken und Kopieren}
		\label{kopieren}
		Es gibt viele Gründe, etwas zu Drucken, von 1000-Seitigen Skripten über am Rechner angefertigte Hausaufgaben bis hin zu Formularen die ihr online erhaltet aber nur offline einreichen dürft. Zur Wahl stehen euch der heimische Drucker (falls vorhanden), diverse Copyshops im Uni-Viertel und die Drucker des GITZ.

		Dabei ist das GITZ mit Abstand konstengünstigste Alternative, da es euch sämtliche Aufträge zum Selbstkostenpreis erfüllt. Praktisch gesehen kannst du dort sogar kostenlos drucken, denn alle Druckaufträge werden von einem persönlichen Druckkostenkonto abgebucht, dass sich zu Beginn jedes Semesters auf magische Weise auf 15,00 Euro regeneriert. (Naja, da diese 15 Euro aus deinen 500 Euro Studienbeiträgen kommen, ist es da mit der Magie und der Kostenlosigkeit so eine Sache\ldots)

		Bislang war das Drucken im GITZ sehr nervenaufreibend, es sei denn man wartet gerne mehr als eine Stunde auf ein einzelnes Blatt Papier. Pünktlich zum Semesterwechsel wird nun das System umgestellt, und in Zukunft soll alles besser werden. Also gebt dem GITZ eine Chance, und probiert es mal aus. Es kostet euch ja schließlich nichts - außer Zeit. Die Drucker des GITZ findet ihr im GITZ-Gebäude, im Altgebäude und im Raum G40 im IZ.

		Kopieren\footnote{Das Kopieren hat weder mit Computern noch mit dem GITZ etwas zu tun, aber passt trotzdem so schön hierher} könnt ihr auch sehr kostengünstig an der Uni. In der Bibliothek stehen euch verschiedene Kopierer zur Verfügung, von denen manche Kleingeld schlucken und andere eine Kopierkarte erfordern, die ihr für 5 Euro am Schalter erwerben könnt. Ansonsten bleibt euch auch hier der Copyshop als Alternative.

	\subsubsection{WLAN}
		\label{wlan}
		WLAN wird vom Rechenzentrum in vielen Hörsälen (wie dem \textbf{Audimax} und \textbf{SN19.1}), im IZ, in der Universitätsbibliothek (UB) und im GITZ angeboten - also fast überall außer der Mensa. Notebookbesitzer finden auf folgender Webseite alle notwendigen Informationen, um das \emph{eduroam} nutzen zu können. \url{http://www.tu-braunschweig.de/it/dienste/11/1106}

		Das \emph{eduroam} ist ein international standardisierter Zugang, der an vielen europäischen Hochschulen funktioniert. Einmal eingerichtet kannst du also mit deinen TU-BS-Zugangsdaten problemlos an anderen Unis surfen.

		Die Anleitungen der TU-Braunschweig werden dir nahelegen, eine spezielle Software nachzuinstallieren. Es geht aber für alle aktuellen Betriebssysteme auch ohne, also nur mit Boardmitteln - um herauszufinden wie, schau einfach im Netz nach, was andere Unis zu \emph{eduroam} zu sagen haben. Für Windows XP (und eng verwandte Versionen) bietet z.B. die Uni Graz eine schöne Anleitung.

		Aber Vorsicht beim kabellosen Vergnügen. Unverschlüsselt übertragene Passwörter (z.B. bei ftp, http, pop3 und imap) können alle WLAN Benutzer in deinem Umkreis mithören. Also verwende immer über SSL gesicherte Protokolle, wenn du sensible Daten überträgst.

		Wer etwas schneller unterwegs sein will (oder wessen Empfang überhaupt nicht ausreicht), dem sei das normale Ethernet ans Herz gelegt. Ein Kabel dazu musst du dir selbst mitbringen. Dosen zum Anschließen gibt es in der Uni"~Bibliothek (z.T. versteckt unter runden Klappen im Boden, z.T. an der Fensterseite frei liegend) und im Rechenzentrum (im Laptopraum \textbf{R003} und in \textbf{R001} zwischen den Mappits).

	% Viel zu viele Fehler nonsens und geflame...
	% \subsubsection{Noch ein bisschen Text}
	% 	Wem der klassische Kommunikationsweg per Email \url{it-zentrum@tu-braunschweig.de} oder Internet \url{http://www.tu-braunschweig.de/it} zu schwierig erscheint, kann auch per Telefon (0531-391 5555) oder persönlich das \textbf{G}auß-\textbf{IT}-\textbf{Z}entrum aka Rechenzentrum besuchen. Wer den weiten Weg nicht scheut, der findet außer den Linux-Distributionen noch viele weitere nützliche Features und Gadgets, die hin und wieder das Leben und Studium vereinfachen. Angefangen mit \textbf{A} wie Antworten zu Problemen rund um Euren Account (y-Nummer, etc.) ¨Uber \textbf{B} wie Bücher über gängige IT-Themen wie Betriebssysteme, Netze oder Programmiersprachen. Eine übersicht dieser sehr günstigen und oft guten Zusammenstellungen findet Ihr auf \url{http://www.tu-braunschweig.de/it/service-desk/rrzn-handbuecher}. Weiter geht es mit \textbf{K} wie Kurse \url{https://www.tu-braunschweig.de/it/service-interaktiv/kurse} zu gängigen Programmen wie zum Beispiel Maya, Photoshop oder auch AutoCAD sowie PHP oder auch C-Programmierung und natürlich Java. Diese werden für Studierende zumeist kostenlos vom GITZ angeboten. Am besten Ihr schaut einfach selber unter \textbf{S} wie Dienstleistungen\footnote{bis vor kurzem hieß dieser Punkt noch \textit{Services}} \url{http://www.tu-braunschweig.de/it/dienste} und bekommt eine Übersicht der angebotenen Geräten, Scannern, Software und Kursen.

		% Der aufmerksame Leser der GITZ-Seiten ist bestimmt über den Abschnitt mit seinem Workspace gestolpert. Jeder Studi hat ungefähr 250~MB zur freien Ferfügung, er kann sich auch einen Ordner anlegen, der im Netz erreichbar ist, also für statische HTML-Seiten, oder per FTP Dateien von Zuhause auf den Uni-Account schieben, damit diese dann in der Uni abrufbar sind. 

		% Eine mehr oder wenig übersichtliche Linksammlung findet Ihr unter \url{http://www.tu-braunschweig.de/it/hotlinks}, so auch zum Beispiel \textbf{Z} wie Zusammenstellung der wichtigsten Befehle für Linux, das ,Don't Panic' \url{http://www.tu-braunschweig.de/Medien-DB/it/dontpanic.pdf} und wem all diese Informationen doch nicht weiter geholfen haben, der sollte mal \textit{man man} ausprobieren...
\end{multicols}

% !TEX root = ../../1-te.tex

\subsection{Linux}
	\label{linux}
	Als Informatiker befasst man sich oft mit abstrakten und allgemeinen Konzepten, die unabhängig von konkreten Betriebssystemen gültig sind. Aber sobald man sich an einen Rechner setzt, hat man es dann doch mit einem konkreten System zu tun, und innerhalb der Rechnerpools an der Uni ist dies meist die eine oder andere Linux-Version. Du wirst also im Studium nicht drumherum kommen, etwas Erfahrung damit zu sammeln.

	Auf deinem eigenen Rechner kannst du natürlich machen, was immer du möchtest, aber viele von uns bevorzugen auch dort Linux oder ein anderes Unix-artiges System. Der Umstieg ist gar nicht so schwer wie man denkt bzw. wie er vor 10 Jahren mal war, und dank Live CDs, Dual Boot und Virtualisierung kannst du sogar Linux und dein bisheriges System parallel laufen lassen und somit ganz unverbindlich reinschnuppern.

	\subsubsection{SSH -- Zugriff aus der Ferne}
		Um vom heimischen PC aus Zugriff auf deinen Uniaccount zu haben, kannst du von Linux aus ssh benutzen. Für Windowsbenutzer gibt es drei nette kleine Tools, Putty, Xming und WinSCP. %Deinen Uniaccount erreichst du über den Server \verUrl{0}{rzstudio.rz.tu-bs.de}.

\todo[inline]{winSCP geht nicht mehr, Zugriff nur noch per Samba}
		\begin{description}
			\item[Putty] stellt dir eine Shell auf dem UNIX-Rechner bereit. Damit kannst du so auf deinem Rechner arbeiten, als würdest du direkt
			  auf dem Server arbeiten (tust du ja auch).  Download: \verUrl{4}{http://www.putty.org/}
			\item[Xming] Um auch grafische Programme starten zu können, musst du noch einen X-Server für Windows
			  installieren, z.B. Xming. Download: \verUrl{4}{http://sourceforge.net/projects/xming/}
			\item[WinSCP] ist ein Tool, das einem FTP-Client ähnelt. Mit diesem kannst du Dateien von und zu deinem Uniaccount kopieren. Der Vorteil ist, dass die Übertragung verschlüsselt ist und Passwörter somit nicht abgehört werden können. Download: \verUrl{4}{http://winscp.net/}
		\end{description}

		Zu allen in diesem Text angesprochenen und noch zu vielen anderen Computerproblemen gibt es mehr Informationen im Heft \emph{Don't Panic}, das kostenlos im Rechenzentrum erhältlich ist und dir sehr wahrscheinlich auch per Post zugeschickt wurde.

	\subsubsection{Linux-Bezug an der TU}
		Fast alle Linux-Distributionen und Softwarepakete für Linux sind freie Software und somit kostenlos erhältlich.

		Für Studierende mit Breitband-Internetzugang sind vermutlich die diversen Mirror-Server an der Uni interessant. Hier stehen die größeren Distributionen bereit:

		\begin{description}
			\item[\verUrl{4}{http://www.knopper.net/knoppix-mirrors/}]~\\Enthält Openoffice-, Mozilla-, Gentoo-, Slackware- und Ubuntumirror, CCC-Vorträge
			\item[\verUrl{4}{http://debian.tu-bs.de/}]~\\Debian-, Kanotix- und Knoppixmirror
			\item[\verUrl{4}{https://www.ibr.cs.tu-bs.de/kb/services.html}] Mehr CCC-Vorträge, diverse freie Software (größtenteils für Unix/Linux)
		\end{description}

%		Für Studierende ohne breitbandigen Netzzugang sind sicherlich die CDs nützlich, die sich jede/r im IT Service-Desk\footnote{\verUrl{4}{http://www.tu-braunschweig.de/it/service-desk}} im Gauß-IT-Zentrum, \textbf{Raum 017}, ausleihen kann. Dort stehen eigentlich immer die neusten Versionen von SuSE, Mandrake, Fedora, Gentoo, Debian und Knoppix sowie diverse ältere Distributionen zur Verfügung. Dank eines DVD-Brenners können inzwischen auch --~soweit vorhanden (SuSE, Knoppix)~-- die DVD-Versionen verliehen werden. Auf der sicheren Seite ist, wer vorher einen Abholtermin vereinbart, damit die gewünschte Distribution garantiert greifbar ist: 0531/391-5555.

%\end{multicols}%o
%end{center}
%\begin{multicols}{2}
\begin{multicols}{2}
\subsection{Microsoft Academic Alliance}
	\label{msdnaa}
	Die TU hat seit 2003 eine Campuslizenz\footnote{http://msdn.microsoft.com/en-us/default.aspx} von Microsoft erworben, in deren Rahmen du Microsoftprodukte kostenlos beziehen kannst.\\ 
	Zur Auswahl stehen die meisten Betriebssysteme, Entwicklungswerkzeuge und diverse Serversoftware\footnote{\sloppy Eine komplette Liste der Software findet sich unter \url{http://msdn.microsoft.com/en-us/subscriptions/downloads/default.aspx}}. Die Office-Suite ist explizit \textbf{nicht} enthalten.

	Die Software darf zu nicht-kommerziellen Zwecken in Forschung und Lehre eingesetzt werden, jedoch keine Infrastrukturaufgaben erf"ullen\footnote{Die Nutzungsbedingungen sind nachzulesen unter \url{http://msdn.microsoft.com/en-us/academic/bb250609.aspx}}. Infos gibt es unter \url{https://www.tu-braunschweig.de/it/service-interaktiv/software/doku/msdn-aa}.

	Etwas paradox ist dabei, dass du ein laufendes Windows brauchst, um Software (also auch Windows selbst) herunterzuladen. Du kannst Microsoft Windows XP aber auch bei den Operateuren im Rechenzentrum in \textbf{Raum 015} f"ur eine Schutzgeb"uhr von 5 Euro erwerben, die "ubrige Software kannst du dort ausleihen oder unter \url{https://www.tu-braunschweig.de/it/service-interaktiv/software/doku/msdn-aa} downloaden.
\end{multicols}

% !TEX root = ../../1-te.tex

\subsection{Elektronisch informiert}
	\label{elekinf}
	Die wichtigsten Aufgaben der Studierenden sind der Besuch von Lehrveranstaltungen, Zeitmanagement für Studium und Freizeit und Informationsbeschaffung. In diesem Artikel geht es um den letzten Punkt. Da wir nun mal Informatik studieren, soll die Informationsbeschaffung über das Internet erfolgen.

	\subsubsection*{Mailinglisten}
	\label{mailinglisten}
		Die wichtigste Mailingliste für Informatikstudierende ist die Liste \textbf{cs-studs}. Sie ist \emph{die} Informationsquelle. Hier werden Ankündigungen zu Lehrveranstaltungen gemacht, die Fachgruppe kündigt hier Spiele- und Grillabende an und es gibt oft Angebote zu Hiwistellen oder offenen Teamprojekten, Bachelorarbeiten etc. und selbstverständlich ist dies auch ein guter Ort, um Fragen zum Studium loszuwerden.

		Wer längere Diskussionen sucht, kann diese auf der Liste \textbf{cs-studs-discuss} finden bzw. führen. Diese Liste ist noch relativ neu und damit liegt es auch an euch, ihr Leben einzuhauchen.

		Da bei den Wirtschaftsinformatikern oftmals auch informatikrelevante Themen diskutiert werden, lohnt sich möglicherweise auch ein Blick in \textbf{winfo-studs}. 
		Wer an Stellenangeboten und Werbung aus der freien
		Wirtschaft interessiert ist, sollte Mailingliste
		\textbf{firmenkontakt} abonieren. Die
		Informatik-Kolloquien, das sind Vorträge von
		üblicherweise externen Referent/innen zu Informatik-Themen,
		werden auf der Mailingliste \textbf{kolloq} angekündigt.
		Alle bisher genannten Mailinglisten sind über
		\verUrl{2}{http://www.cs.tu-bs.de/mailinglisten.html}
		erreichbar. Unter
		\verUrl{3}{https://mail.ibr.cs.tu-bs.de/mailman/listinfo/}
		findest du eine umfassendere Liste der angebotenen Mailinglisten in der Informatik.

	\subsubsection*{IRC Channel}
		Im Freenode IRC Chat (\verUrl{3}{http://freenode.net}) gibt es den Channel \url{###cs-studs}. Hier sind immer ein paar BraunschweigerInnen und große Teile der Fachgruppe online. Die Gesprächsthemen haben (im weitesten Sinne ;) mit dem Studium zu tun.

	\subsubsection*{Clevershit}

		Auf jeden Fall einen Besuch wert und eine gute Hilfe bei allem, was das Studium betrifft, ist das von Studierenden ins Leben gerufene Portal \mbox{\verUrl{2}{https://clevershit.de/}}.

		Dieses von Studierenden für Studierende erstellte und geführte Plattform bietet eine gute Anlaufstelle für Fragen jeglicher Art. Es gibt eine Materialsammlung zu allen Veranstaltungen der ersten Semester.

\subsubsection*{Sonstige Informationen}
	\begin{description}
		\item[Allgemeines Vorlesungsverzeichnis:] ~\\
			{\footnotesize\verUrl{3}{http://vorlesungen.tu-bs.de}}
		\item[Uni-Bibliothek:] ~\\
			{\footnotesize\verUrl{3}{http://www.biblio.tu-bs.de}}
		\item[Druckkosten:] ~\\
			{\footnotesize\verUrl{3}{https://www.tu-braunschweig.de/it/service-interaktiv/druckkosten}}
		\item[Don't Panic online] ~\\
			{\footnotesize\verUrl{3}{http://www.tu-braunschweig.de/Medien-DB/it/dontpanic.pdf}}
	\end{description}

