\subsection{Lernräume}
%Hier fehlt noch Text, den man aber einfach von \url{http://fginfo.cs.tu-bs.de/index.php/studium/lernraume/} nehmen kann.
Hier wollen wir euch eine aktuelle Übersicht über Lernräume an der TU
Braunschweig geben. Die Liste ist im Moment nicht vollständig, sie
wird aber demnächst erweitert und ist dann auf
\url{http://fginfo.cs.tu-bs.de/index.php/studium/lernraume/} zu finden.
Alle Gebäude stehen, wenn nicht anders in Anlage 1 der Hausordnung der
TU Braunschweig erwähnt, von 7:30 bis 19:30 Uhr offen.
\end{multicols}
\paragraph{Informatikzentrum} \ \\
\begin{tabular}{|p{4cm}|p{4cm}|p{8cm}|}
  \hline Raum & Öffnungszeiten & Ausstattung  \\ 
  \hline Plaza des Informatikzentrums & normal &  Tische und Stühle,
  Steckdosen unter Bodenabdeckungen zu finden\\ %\hline
  \hline Studentenarbeitsraum IZ 159 & normal & Tische und Stühle,
  Steck\-dosen an der Wand, in naher Zukunft auch Lichtschutz,
  Präsenzbücher etc. \\ \hline
  Fachgruppenraum IZ 150 & nach Absprache mit Mitgliedern des Fachgruppenrates &
  Kaffemaschine, Sofas, Tische, Steckdosen in Massen sowie
  Ethernetkabel\\ \hline
  CIP Pool IZ G40 & normal & Rechner-Pool mit Linux-PCs, Tafel\\ \hline
\end{tabular}
\paragraph{Andere Lernräume} \ \\
 \begin{tabular}{|p{4cm}|p{4cm}|p{3.6cm}|p{4cm}|}%\hline 
  \hline Raum & Öffnungszeiten & Ausstattung & Anmerkung  \\  \hline
Grotrian  Zimmerstraße 24 & Normal  & Alte Tische und Stühle, vereinzelt
Tafeln&Wenn Mitglieder der verschiedenen Fachgruppen anwesend
sind hat das Grotrian meist länger offen. Da dies oft der Fall ist
kann man hier meist lange lernen. \\ \hline
Bibliothek &
 Mo - Fr: \ 07:00 - 24:00 Sa: 10:00 - 20:00& Niedrige
Tische und Stühle 
 &
 Man muss leise sein, daher praktisch  \\
\ & & \noindent Rechnerarbeits\-plätze, Kopierer&nicht zum  Lernen in der
Gruppe  geeignet \\ \hline
Mensa / Cafeteria & Mo -Do: 08 - 20:00 Uhr Fr: 08:00 - 15:00 &
Tische, Stühle, kein (!) WLAN, einzelner Rechner mit Netzzugang,
Verpflegung incl. Selbstbedienungs-Kaffeeautomat& Probleme: Nicht
durchgehend geöffnet, die Plätze sind
primär zu Essen gedacht, von Lernsessions zu den Stoßzeiten sollte man
also im eigenen und fremden Interesse absehen. \\ \hline
Bei dir zuhause & immer & Deine Sache & Achtung: Man lenkt sich leicht
ab :) \\ \hline
Das eine oder andere Cafe / Kneipe & kommt drauf an & wechselhaft &Siehe die beiden
vorherigen \\\hline
\end{tabular}
%Mehr Informationen zu Campuskarten findet ihr auf Seite:
%\pageref{campuskarte}
%Virtuelle Campustouren sind in einem Web 2.0 Seminar entstanden und
%bei Google Maps zu finden.
%Ein Raumplan für das 1. und 2. OG des Informatikzentrums findet sich unter \url{http://www.ibr.cs.tu-bs.de/rooms/rooms.html}
%\end{multicols}
% Local Variables: 
% mode: latex
% TeX-master: "../../1-te"
% End: 

\begin{multicols}{2}
